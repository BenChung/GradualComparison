\documentclass{sigplanconf}
\usepackage{xspace,hyperref,times,amsmath,xspace,listings,url,subfigure, framed}
\usepackage{graphicx,wrapfig,array,multirow,mathpartir,mathpartir,amsmath,amssymb}

\newcommand{\doi}[1]{doi:~\href{http://dx.doi.org/#1}{\Hurl{#1}}}
 \hypersetup{colorlinks=true,
   linkcolor=black,  citecolor=black,filecolor=magenta, urlcolor=black,
   pdftitle={R}, pdfauthor={Morandat Hill Osvald Vitek},
   pdfkeywords={Dynamic Languages, R, Reflection, Dynamic Analysis}}

\newcommand{\rl}[1]{{{\small{[{\sc #1}]}}}\xspace}
\newcommand{\lang}[1]{#1}% {\sf\scshape#1}\xspace}
\newcommand{\JAVA}{\lang{Java}\xspace}
\newcommand{\CommonLisp}{\lang{CommonLisp}\xspace}
\newcommand{\Java}{\lang{Java}\xspace}
\newcommand{\Scheme}{\lang{Scheme}\xspace}
\newcommand{\JavaScript}{\lang{Java\-Script}\xspace}
\newcommand{\Python}{\lang{Python}\xspace}
\newcommand{\Ruby}{\lang{Ruby}\xspace}
\newcommand{\Groovy}{\lang{Groovy}\xspace}

\newcommand{\NOTE}[3][NOTE]{\marginpar{\textcolor{#2}{\textbf{#1:}\scriptsize\sf#3}}}
\newcommand{\TODO}[2][Unassigned]{\NOTE[#1]{black}{#2}}
\newcommand{\FM}[1]{\NOTE[FM]{blue}{#1}}
\newcommand{\BH}[1]{\NOTE[BH]{magenta}{#1}}
\newcommand{\JV}[1]{\NOTE[JV]{red}{#1}}
\newcommand{\PROOF}[1]{#1}

\newcommand{\NUM}[2][]{#2#1\xspace}
\newcommand{\PC}[1]{#1\%\xspace}

\def\NEEDNUMBER{XXX\xspace}
\def\UrlFont{\fontfamily{cmtt}\selectfont}
\lstset{tabsize=2,columns=flexible,%
	basicstyle=\small\ttfamily,keywordstyle=\bfseries,%
	commentstyle=\rmfamily\itshape,indexstyle=[1]\indexlst,%
	showstringspaces=false,%
	lineskip=1pt,
	showspaces=false,belowcaptionskip=\baselineskip,framexleftmargin=5mm,%
	texcl=true,xleftmargin=15pt, % make room for line numbers
	breaklines=true,
	breakatwhitespace=true,
	escapeinside={(*}{*)},
}
\newcommand{\code}[1]{\lstinline[keywordstyle=]!#1!\xspace}
\newcommand{\etal}{{\em et al.}}
\def\cross{\ding{55}}\def\naive{na\"\i ve }
\newcommand{\app}[1]{{\small\textsf{#1}}}
\newcommand{\IGNORE}[1]{}\newcommand{\hide}[1]{}
\newcommand{\Section}[1]{Sect.~\ref{#1}\xspace}
\newcommand{\Figure}[1]{Fig.~\ref{#1}\xspace}
\newcommand{\figref}[1]{Fig.~\ref{fig:#1}\xspace}
\newcommand{\tabref}[1]{Table~\ref{tab:#1}\xspace}

\newcommand{\kewif}{\texttt{\textbf{if}}}
\newcommand{\kewthen}{\texttt{\textbf{then}}}
\newcommand{\kewelse}{\texttt{\textbf{else}}}
\newcommand{\kewnew}{\texttt{\textbf{new}}}
\newcommand{\kewclass}{\texttt{\textbf{class}}}
\newcommand{\kewimplements}{\texttt{\textbf{implements}}}

\newcommand{\ifabs}[3]{\kewif \; #1 \;\kewthen\; #2 \;\kewelse\; #3}
\newcommand{\opplus}[2]{#1 + #2}
\newcommand{\newclass}[2]{\kewnew\;\texttt{#1}(#2)}

\newcommand{\strt}{\texttt{str}}
\newcommand{\intt}{\texttt{int}}
\newcommand{\anyt}{\texttt{any}}
\newcommand{\vbar}{\;|\;}
\newcommand{\spec}{\vartriangleright}

\newcommand{\opdef}[2]{\framebox[1.1\width]{#1} ~ #2\\}

%!TEX output_directory=pdfs
\documentclass[a4paper,UKenglish,final]{tex/lipics-v2016}
\usepackage{xspace,listings,url,framed,amssymb,
            amsmath,tex/mathpartir,hyperref,stmaryrd, graphicx, mathtools}
%% Formatting
\newcommand{\EM}[1]{\ensuremath{#1}\xspace}
\newcommand{\xt}[1]{{\sf{#1}}}
\newcommand{\bt}[1]{\xt{\bf #1}}
\renewcommand{\b}[1]{\EM{\overline{#1}}}
\newcommand{\EMxt}[1]{\EM{\xt{#1}}}
\newcommand{\EMbt}[1]{\EM{\bt{#1}}}

%% Variables
\newcommand{\x}   {\EMxt x}
\newcommand{\n}   {\EMxt n}
\newcommand{\e}   {\EMxt e}
\newcommand{\m}   {\EMxt m}
\newcommand{\s}   {\EM{\sigma}}
\renewcommand{\t} {\EMxt t}
\newcommand{\ta}  {\EM{\tau}}
\renewcommand{\a} {\EMxt a}
\newcommand{\K}   {\EMxt K}
\renewcommand{\k} {\EMxt k}
\newcommand{\Kp}  {{\EMxt{K'}}}
\newcommand{\Kpp}  {{\EMxt{K''}}}
\newcommand{\Kppp}  {{\EMxt{K'''}}}
\newcommand{\ep}  {{{\EMxt{e'}}}}
\newcommand{\epp}  {{{\EMxt{e''}}}}
\renewcommand{\sp}{{{\EM{\s'}}}}
\newcommand{\spp}{{{\EM{\s''}}}}
\newcommand{\ap}  {\EM{\a'}}
\newcommand{\aE}[1]  {\EM{\a_{#1}}}
\newcommand{\app}  {\EM{\a''}}
\newcommand{\tp}  {\EM{ \t'}}
\newcommand{\tpp}  {\EM{ \t''}}
\newcommand{\C}   {\EMxt C}
\newcommand{\Cp}  {\EMxt{C'}}
\newcommand{\EC}   {\EMxt E}
\newcommand{\fd}  {\EMxt{fd}}
\newcommand{\md}  {\EMxt{md}}
\newcommand{\mdpp}  {\EM{\md'}}
\newcommand{\mt}  {\EMxt{mt}}
\newcommand{\mtp}  {\EMxt{mt'}}
\newcommand{\mtpp}  {\EMxt{mt''}}
\newcommand{\M}{\EMxt M}
\newcommand{\MN}  {\EMxt{M\,K}}
\newcommand{\MNargs}[1]  {\EMxt {M #1~K}}
\newcommand{\f}   {\EMxt f}
\newcommand{\fp}   {\EMxt{ f'}}
\newcommand{\E}   {\EM{\Gamma}}
\newcommand{\EE}   {\EM{\mathcal{E}}}
\newcommand{\any} {\EM{\star}}
\newcommand{\this}{\EMxt{this}}
\newcommand{\that}{\EMxt{that}}
\newcommand{\none}{\EM{\cdot}}
\newcommand{\D}   {\EMxt D}
\newcommand{\Dp}   {\EMxt{D'}}
\newcommand{\p}   {\EMxt p}
\newcommand{\np}{\n'}

\newcommand{\Get}[2]   {\EM{#1.#2()}}
\newcommand{\Set}[3]   {\EM{#1.#2(#3)}}
\newcommand{\Call}[3]  {\EM{#1.#2(#3)}}
\newcommand{\DynCall}[3]  {\EM{#1@#2(#3)}}

\newcommand{\New}[2]   {\EM{\new\;#1(#2)}}
\newcommand{\SubCast}[2]{\EM{<\hspace{-.6mm}{#1}\hspace{-.6mm}>\hspace{-1mm}\;{#2}}}
\newcommand{\ShaCast}[2]{\EM{\prec #1 \succ #2}}
\newcommand{\MonCast}[2]{\EM{\triangleleft\; #1 \triangleright #2}}
\newcommand{\BehCast}[2]{\EM{\blacktriangleleft #1 \blacktriangleright #2}}
\newcommand{\new}      {\EM{\bt{new}}}
\newcommand{\HT}[2]    {\EM{{#1}\!:{#2}}}
\newcommand{\Mdef}[5]  {\EM{ \HT{ #1( \HT{#2}{#3})}{#4}\;\{{#5}\}}}
\newcommand{\Mdefz}[3] {\EM{ \HT{ #1()}{#2}\;\{{#3}\}}}
\newcommand{\Mdefa}[4]  {\EM{ \HT{ #1( #2 )}{#3}~\{{#4}\}}}
\newcommand{\obj}[2]   { \EM{ #1\{#2\}}}
\newcommand{\alloc}[4] {\EM{#1\;#2  = \xt{alloc}(#3, #4)}}
\newcommand{\cast}[8]  {\EM{#6\;#7\;#8=\xt{#5 cast}(#1, #2, #3, #4)}}
\newcommand{\behcast}[7]  {\EM{\xt{behcast}(#1, #2, #3, #4)=#5\,#6\,#7}}
\newcommand{\moncast}[6]  {\EM{\xt{moncast}(#1, #2, #3, #4)=#5\,#6}}

\newcommand{\Alt}[1]   { &\B #1 \\}
\newcommand{\B}        {\EM{~|~}}
\newcommand{\bang}     {\EM{\xt{!}}}

\newcommand{\dispatch}[5] {\EM{#1\;#2 = \xt{disp}(#3,#4,#5)}}
\newcommand{\readf}[4]{\EM{\xt{read}(#1,#2,#3,#4)}}
\newcommand{\convert}[1]{\EM{\xt{cnvtMD}(#1)}}
\newcommand{\convertFD}[1]{\EM{\xt{cnvtFD}(#1)}}
\newcommand{\readfield}[4]{\EM{#1 = \xt{read}(#2,#3,#4)}}
\newcommand{\setf}[5] {\EM{\xt{write}(#1,#2,#3,#4,#5)}}
\newcommand{\Reduce}[6]   {\EM{{#1}~{#2}~{#3} \rightarrow {#4}~{#5}~{#6}}}
\newcommand{\ReduceA}[6]  {\EM{#1~#2~#3 } & \EM{\rightarrow #4~#5~#6}}
\newcommand{\Class}[3]    {\EM{\bt{class}\;#1\,\{\,#2~#3\,\}}}
\newcommand{\Ftype}[2]    {\EM{ \HT{#1}{#2} }}
\newcommand{\Fdef}[2]    {\EM{ \HT{#1}{#2} }}
\newcommand{\Mtype}[3]    {\EM{ \HT{#1(#2)}{#3}}}
\newcommand{\Type}[1]     {\EM{\{#1\}}}

\newcommand{\opdef}[2]    {\framebox[1.1\width]{#1} ~ #2\\}
\newcommand{\Map}[2]     {\EM{ #1[#2] }}
\newcommand{\Bind}[2]     {\EM{#1 \mapsto #2}}

\newcommand{\Sub}{\EM{<:}}
\newcommand{\OK}{\EM{~\checkmark}}
\newcommand{\SubE}[1]{\EM{<:_{#1}}}
\newcommand{\names}[1]{\EM{\xt{names}(#1)}}
\newcommand{\untyped}[1]{\EM{\xt{untyped}(#1)}}

\newcommand{\mnames}[1]{\EM{\xt{methName}(#1)}}
\newcommand{\fnames}[1]{\EM{\xt{fieldName}(#1)}}


\newcommand{\ConsSub}{\EM{\lesssim}}

\newcommand{\CondRule}[3]{ #3 &~ #2 \\}
\newcommand{\SuchRule}[3]{ #3 &~{\emph{s.t.}} #2 \\}
\newcommand{\EnvType}[5]{ \EM{#1\,#2\,#3\vdash #4 : #5}}

\newcommand{\IRule}[4][]{\inferrule*[lab={\tiny #2},#1]{#3}{#4}}
\newcommand{\HasType}[3]{ \EM{#1 (#2) = #3}}
\newcommand{\wrapper}[1]{\EM{\xt{wrap}(#1)}}
\newcommand{\spec}[4]{\EM{\xt{spec}(#1,#2,#3,#4)}}

\newcommand{\castfn}[4]{\text{cast}(#1,#2,#3,#4)}
\newcommand{\GenCast}[5]{#1~#2 \vdash #3 \hookrightarrow #4 \Uparrow #5 }
\newcommand{\AnaCast}[5]{#1~#2 \vdash #3 \Downarrow #5 \hookrightarrow #4}
\newcommand{\TransClass}[2]{\EM{ #1 \rightharpoonup #2 }}
\newcommand{\inv}[2]{\xt{invoke}(#1, #2)}
\newcommand{\classoff}[2]{\EM{\xt{mtypes}(#1,#2)}}
\newcommand{\classoffs}[3]{\EM{\xt{mtypes}(#1,#2,#3)}}
\newcommand{\mtype}[3]{\EM{\xt{mtype}(#1,#2,#3)}}
\newcommand{\wftype}[3]{\EM{\xt{wftype}(#1,#2,#3)}}

\newcommand{\field}[2]{\EM{\xt{field}(#1,#2)}}
\newcommand{\In}{\EM{\in}}

\newcommand{\T}{\EM{\xt T}}
\newcommand{\Cast}{Cast }
\newcommand{\fb}{\EM{\xt{f!}}}

\newcommand{\AND}{\EM{\wedge}}
\newcommand{\App}[2]{\EM{#1(#2)}}

\newcommand{\StrSub}[4]{\EM{#1~#2\vdash #3\Sub #4}}
\newcommand{\tmeet}[4]{\xt{tmeet}(#1,#2,#3,#4)}
\newcommand{\mmeet}[4]{\xt{mmeet}(#1,#2,#3,#4)}
\newcommand{\mtypes}[2]{\xt{mtypes}(#1,#2)}


\newcommand{\WFtype}[2]{\EM{#1\vdash#2 \OK}}
\newcommand{\WF}[4]{\EM{#1\,#2\,#3\vdash#4 \OK}}
\newcommand{\WFp}[3]{#1~#2~#3\OK}

\renewcommand{\P}{\EMxt P}
\newcommand{\Pp}{\EMxt{P'}}


\newcommand{\retype}[5]{\xt{retype}(#1,#2,#3,#4,#5)}
\newcommand{\htype}[3]{\EM{\xt{htype}(#1,#2,#3)}}
\newcommand{\ftypes}[4]{\xt{ftypes}(#1,#2,#3,#4)}
\newcommand{\typeof}[1]{\xt{typeOf}(#1)}
\newcommand{\classgen}[1]{\xt{classgen}(#1)}
\renewcommand{\S}{\EM{\tau}}
\newcommand{\Sp}{\EM{\tau'}}
\newcommand{\Spp}{\EM{\tau''}}
\newcommand{\EQ}{\EM{\equiv}}

\newcommand{\Dom}[1]{\EM{\xt{dom}(#1)}}
\newcommand{\fresh}[1]{\EM{#1~\xt{fresh}}}

\newcommand{\progtrans}[2]{#1 ~\hookrightarrow_p~ #2}
\newcommand{\classtrans}[3]{#1 \vdash #2 ~\hookrightarrow_c~ #3}
\newcommand{\methtrans}[4]{#1~#2 \vdash #3 ~\hookrightarrow_m~ #4}
\newcommand{\statictype}[2]{\xt{static}(#1,#2)}

\begin{document}

\section*{Target Langage}

The target language (TL) is built around a parred down statically typed
object calculus. The calculus has non-recursive structural
substyping\footnote{\bf I am not sure that this is the right way to describe
  what we have -- check it is. JV} without inheritance.\footnote{Structural
  subtyping is needed for type systems that have a notion of
  ``consistency''. Inheritance does not add anything other than code reuse.}
The calculus does not have null values.\footnote{Null adds a source of error
  but otherwise should not be too interesting. We should mention what is the
  type of null.} Fields are accessed by automatically provided getter and
setter methods, respectively denoted \Get\x\f and \Set\x\f\e.\footnote{This
  is because we want an easy way to interpose on field access in wrappers.}
Fields appear in type signatures only through the presence of the getter and
setter methods. Method calls are denoted \Call\x\m\e.\footnote{The single
  argument limitation is purely to lighten the notation.} Following C\#
4.0~\cite{} the TL allows a dynamic type, denoted \any, and
dynamically-typed method invocation, denoted
\DynCall\x\m\e. Dynamically-typed method invocation treats arguments and
return value as \any. A new object is created with a class name and a
sequence of arguments in the order of definition of fields, denoted
\New\C{\b\e}.  Meta-variable \x ranges over argument names, \a over memory
locations, \f over field names, \m over method names, \C over class
names. The language has four different kinds of cast expressions: it has two
structural casts and two generative casts. The structural subtype cast,
denoted \SubCast\t\a, asserts that the object at location \a is of type
\t.  The structural shallow cast, denoted \ShaCast\t\a, asserts that the
object at location \a has methods with names matching those of \t. This does
not make any guarantee about the type of arguments.  The generative
behavioral cast, denoted \BehCast\t\a, will ensure that either \a behaves
as a \t or that it get stuck. The generative monotonic cast, denoted
\MonCast\t\a, is a behavioral cast that, in addition, imposes constraints
on fields.

\this is a distinguished variable name that denotes the current object.
\that is a distinguished field name that denotes the target of a wrapper.

\D is a meta-variable used to range over dynamically generated class names.

\begin{minipage}{7cm}\begin{tabular}{l@{~~}l@{}l@{}l@{}ll}
\p &::= \e \b\k \\
\e &::=  \x &\B \this &\B \that \\
   &\B \Get\e\f    &\B \Set\e\f\e    &\B \Call\e\m\e  &\B \DynCall\e\m\e \\
   &\B \New\C{\b\e} \\
   &\B \SubCast\t\e  &\B \ShaCast\t\e    &\B \BehCast\t\e &\B  \MonCast\t\e \\
   &\B \a \\ 
% \Mxt{M} &::= ~~\none ~~ \B \b{\t \Sub \t} \\
% \Mxt{K} &::= ~\none ~~ \B \b{{\k}}     
\end{tabular}\end{minipage}
~~
\begin{minipage}{6cm}\begin{tabular}{l@{~~}l@{}l}
\k &::= \Class \C {\b{\Ftype\f\t}}{\b\md} \\
\end{tabular}
\begin{tabular}{l@{~~}l@{}l@{}l}
\md &::= \Mdef\m\x\t\t\e \\
    &\B  \Mdef\f\x\t\t\e ~\B~ \Mdefz\f\t\e \\
\mt &::= \Mtype\m\t\t ~\B~  \Mtype\f\t\t  ~\B~ \Mtype\f{}\t  \\
%     &\B  $\mu$\ta.\mt \\
\E  &::= \Ftype\x\t  \B \none\\
\t  &::= ~ \any  \B   \C  \B   \{\b\mt\} \\ 
% \B \ta \\
\s  &::= ~~\none ~~ \B ~~
    {\Bind{\a}{\obj\C{\b\a}}}~\s
\end{tabular}\end{minipage}

The semantics of TL is defined by a small step operational semantics with
evaluation contexts.  The context are deterministic. Program error is
denoted by a stuck term. Meta-variable \k ranges over class definitions.
The semantics operate over an explicit class table, denoted \K, which is a
sequence of class definitions. A heap, denoted \s, maps memory locations to
objects. We use the notation \Heap\s{\Bind\a{\obj\C{\b\a}}} to denote the
heap \s extended by the binding of location \a to object \obj\C{\b\a}.  A
configuration \K\e\s evaluates in one step to a new configuration, denoted
\Reduce \K\e\s \Kp\ep\sp. Execution terminate if \ep\xspace is a value, \a,
or if there is no applicable reduction, in which case the program is
stuck.\footnote{Intuitively we would expect that the program can only get
  stuck at dynamic calls or at structural casts. But generative casts add
  some complexity to that statement.}

New object creation picks a fresh memory location \ap and binds it to the
newly created. Operations on fields are require a typed receiver, for
example in the expression \Get\this\f, \this is always of the type of the
current class. Field access through a getter method, works as follows. If
the receiver's class has a getter method, that method is evaluated,
otherwise the field corresponding the getter's name is updated. Methods are
segregated into typed method (methods whose argument is not \any) and
untyped method (methods whose argument is \any). The former can be called by
statically resolved method, the latter must be called by dynamically
resolved methods.

The auxiliary function \texttt{names} return the list of function names in a class
or type, functions that have \any as return values are prefixed with the
symbol \any. The auxiliary function \texttt{typeof} return the type of the object 
at the location \a in the heap \s. The auxiliary function \texttt{read} return
the location \ap pointing to the field \f of the object at location \a. The 
auxiliary function \texttt{write} return the heap \sp with the field \f of
the object at location \a updated to the location \ap.


\hrulefill

\begin{minipage}{8cm}
  \opdef{\Reduce \K\e\s \Kp\ep\sp}
        {\e\s evaluates to \ep\xspace in a step}\\[-1mm]
\begin{tabular}{@{}l@{}l@{~}l@{~}l}
\CondRule{E2}{ %% new C -> a
   \sp = \Heap\s{\Bind\ap{\obj\C{\b\a}}} $\wedge ~$ \ap fresh
}{ 
    \ReduceA \K{\New\C{\b\a}}\s \K\ap\sp
}
\CondRule{E5}{ %% a.f() -> e
    \Mdefz\f\t\e \In \methods\s\a\K 
}{
   \ReduceA \K{\Get\a\f}\s \K{[\a/\this]\e}\s
}
\CondRule{E4}{ %% a.f(a) -> e
    \Mdef\f\x\t\t\e \In \methods\s\a\K 
}{
   \ReduceA \K{\Set\a\f\ap}\s \K{[\a/\this~{\ap/\x}]\e}\s
}
\CondRule{E6}{  %% a.f() -> e
 \ap = \readf \s\a{\f} 
}{
  \ReduceA \K{\Get\a{\f}}\s  \K\ap\s
}
\CondRule{E7}{ %% a.f(e) -> e
 \sp = \setf \s\a{\f}\ap

}{
  \ReduceA \K{\Set\a{\f}\ap}\s \K\ap\sp
}
\CondRule{E3}{ %% a.m(a) -> e
  \Mdef\m\x\t\tp\e \In \methods\s\a\K  $\wedge~ \t \not= \any$
}{
 \ReduceA \K{\Call\a\m\ap}\s \K{[\a/\this~{\ap/\x}]\e}\s
}
\CondRule{E3}{ %% a@m(a) -> e
    \Mdef\m\x\any\any\e \In \methods\s\a\K 
}{
  \ReduceA \K{\DynCall\a\m\ap}\s \K{[\a/\this~{\ap/\x}]\e}\s
}
\CondRule{E10}{  %% Subtypecast
   \BasicType {\none} {\classoffs\s\a\K \Sub \t} $~\vee~$ \t $= \any$
}{ 
  \ReduceA \K{\SubCast \t\a}\s \K\a\s
}
\CondRule{E11}{
 \names{\classoffs\s\a\K}  $\supseteq$  \names\t
}{ 
  \ReduceA \K{\ShaCast \t\a}\s \K\a\s
}
\CondRule{E8}{
  \behcast \a\t\s\K  \Kp\ap\sp
}{ 
  \ReduceA  \K{\BehCast \t\a}\s \Kp\ap\sp
}
\CondRule{E9}{
  \moncast \a\t\s\K  \Kp\sp
}{ 
  \ReduceA \K{\MonCast\t\a}\s \Kp\a\sp
}
\CondRule{E1}{ %% e -> e'
  \Reduce \K\e\s \Kp\ep\sp
}{
 \ReduceA \K{\Ctx\e}\s \Kp{\Ctx\ep}\sp
}
\end{tabular}\end{minipage}

%%%%%%%%%%%%%%%%%%% CONTEXTS %%%%%%%%%%%%%%%%%%%%%%%%%%%%%%%%%%%%%%%%%%%%
~\\[3mm]

\begin{minipage}{4cm}\begin{tabular}{l@{~~}l@{~}l}
\E &::= \Get\square\f     ~\B~
        \Set\square\f\e   ~\B~
        \Set\a\f\square   ~\B~  
        \Call\square\m\e   ~\B~
        \Call\a\m{\square} \\
   &\B~
        \DynCall\square\m\e   ~\B~
        \DynCall\a\m\square   ~\B~\\
   &\B~
       \SubCast\t\square  ~\B~
       \ShaCast\t\square  ~\B~
       \BehCast\t\square \\ &\B~
       \MonCast\t\square  ~\B~
       \New\C{\b \a\,\square\,\b\e}
\end{tabular}
\end{minipage}


Types are represented as a set of methods. 

\hrulefill

\opdef{\BasicType \MN {$~\t \Sub \tp$}}{\t is a subtype of \tp}
\begin{mathpar}
\IRule{SRef}{
}{
 \BasicType \MN {\t \Sub \t}
}

\IRule{STop}{
}{
 \BasicType \MN {\t \Sub {\Type{}}}
}

\IRule{SAssump}{
\t \Sub \tp \in \M{M} 
}{
 \BasicType \MN {\t \Sub \tp}
}

\IRule{SRec}{
 \classoff\C\K = \Type{\mt~\b{\mt_1}\,} \qquad
 \classoff\Cp\K = \Type{\mt'~\b{\mt_2}\,} \\
 \BasicType {\MNargs{\Type{\mt~\b{\mt_1}\,} \Sub  \Type{\mt'~\b{\mt_2}\,}}} {\mt\Sub\mt'} \\
 \BasicType {\MNargs{\Type{\mt~\b{\mt_1}\,} \Sub  \Type{\mt'~\b{\mt_2}\,}}} {\Type{ \b{\mt_1}} \Sub \Type{\b{\mt_2}}}
}{
 \BasicType \MN {\C} \Sub
  \Cp 
}

\IRule{SMeth}{
   \BasicType \MN {\t_2 \Sub \t_1} \\ \BasicType \MN {\t_1' \Sub \t_2'} 
}{
  \BasicType \MN {\Mtype\m{\t_1}{\t_1'} \Sub \Mtype\m{\t_2}{\t_2'}}
}

\IRule{SFd1}{
}{
 \BasicType \MN {\Mtype\f{}\t \Sub  \Mtype\f{}\t}
}

\IRule{SFd2}{
}{
 \BasicType \MN {\Mtype\f\t\t \Sub \Mtype\f\t\t}
}

\end{mathpar}

\hrulefill

\opdef{\BasicType \Sk {$~\Class \C {\b{\Ftype\f\t}}{\b\md} \OK$}}{Well-formed class}
\begin{mathpar}
\IRule{WFClass}{
 \b{\names{\md}} $~unique$ \\
 \b\f $~unique$ \\
 \BasicType {\none ~\none ~\K} {\b\md \OK} \\
 \BasicType {\K} {\b\t \OK} \\
}{
 \BasicType {\K} {\Class \C {\b{\Ftype\f\t}}{\b\md} \OK}
}
\end{mathpar}

\hrulefill

\opdef{\BasicType \Sk {$~\md \OK$}}{Well-formed method}
\begin{mathpar}
\IRule{WFMeth}{
 \EnvType {\EsE{\Ftype\x\tp}}{\e}{\t}\\
 \BasicType {\K} {\tp \OK} \\
 \BasicType {\K} {\t \OK}
}{
 \BasicType {\E~ \s~ \K} {\Mdef\m\x\tp\t\e \OK}
}

\IRule{WFFieldS}{
 \EnvType {\EsE{\Ftype\x\tp}}{\e}{\t} \\
 \BasicType {\K} {\tp \OK} \\
 \BasicType {\K} {\t \OK}
}{
 \BasicType {\E ~\s ~\K} {\Mdef\f\x\tp\t\e \OK}
}

\IRule{WFFieldR}{
 \EnvType \Es\e\t \\
 \BasicType {\K} {\t \OK} 
}{
 \BasicType {\E~ \s~ \K} {\Mdefz\f\t\e \OK}
}

\end{mathpar}

\hrulefill

\opdef{\BasicType \K {$~\t \OK$}}{Well-formed types}
\begin{mathpar}
\IRule{WFStar}{
}{
 \BasicType {\K} {\any \OK}
}

\IRule{WFClass}{
 \C \in $~dom$(\K)
}{
 \BasicType {\K} {\C \OK}
}

\end{mathpar}


Type checking is standard.

Field accessor rules W3 and W4 require a typed receiver, since \any does
not have any methods a receiver typed at \any will never typecheck.

Shallow casts, W9, do not change the type of the expression. We are casting
to the name of \t not to \t.  In practice that means that all expression
types in Transient will drift towards \any.

\hrulefill

\opdef{\EnvType\Es\e\t}{\e has type \t in environment \E against heap \s}
\begin{mathpar}
\IRule{W1}{
   \HasType \E\x\t
 }{
   \EnvType \Es\x\t
}

\IRule{W2}{
  \EnvType \Es\e\tp \\
 \BasicType{\M{\none}}{\tp \Sub \t}
 }{
   \EnvType \Es\e\t 
}   

\IRule{W3}{
  \EnvType \Es\e\t \\
  \Mtype \f{}\tp \inc \t  
}{
  \EnvType \Es{\Get\e\f}\tp
}    

\IRule{W4}{
  \EnvType \Es\e\t \\
  \Mtype \f\tp\tp \inc \t  \\
  \EnvType \Es\ep\tp
}{
  \EnvType \Es{\Set\e\f\ep}\tp
}    

\IRule{W5}{
  \EnvType \Es\e\t \\
  \Mtype \m\tp\tpp\inc \t  \\
  \EnvType \Es\ep\tp
}{
  \EnvType \Es{\Call\e\m\ep}\tpp
}    

\IRule{W6}{
  \EnvType \Es\e\any \\
  \EnvType \Es\ep\any
}{
  \EnvType \Es{\DynCall\e\m\ep}\any
}    

\IRule{W7}{
  \b{\EnvType \Es\e\t} \\ 
  \Class \C {\b{\Ftype\f\t}} {\b\md}
}{
  \EnvType \Es{\New\C{\b\e}}\C
}

\IRule{W8}{
  \EnvType \Es\e\tp
}{
  \EnvType \Es{\SubCast\t\e}\t
}

\IRule{W9}{
  \EnvType \Es\e\tp
}{
  \EnvType \Es{\ShaCast\t\e}\any  %%!!!  not \t !!!
}

\IRule{W10}{
  \EnvType \Es\e\tp
}{
  \EnvType \Es{\BehCast\t\e}\t
}

\IRule{W11}{
  \EnvType \Es\e\tp
}{
   \EnvType \Es{\MonCast\t\e}\t
}

\IRule{W12}{
  \s(\a) = \obj\C{\b{\a_1}}
}{
  \EnvType \Es\a\C
}
\end{mathpar}


%%% defining the auxiliary functions below, they include:
%%% methdef
%%% typeof
%%% read (field)
%%% write (field)

\hrulefill

\opdef{\methz\m\C\K}{Auxiliary function: Method definition}

\begin{equation*}
\methz\m\C\K = \Mdef\m\x\t\t\e ~~s.t.~~ \begin{cases}

 \K(\C) = \Class \C {\b{\Ftype\f\t}}{\dots \Mdef\m\x\t\t\e \dots} \\
%  \m == \names\md  \\
\end{cases}
\end{equation*}

\hrulefill

\opdef{\mtype\m\C\K}{Auxiliary function: Method definition}

\mtype\m\C\K = \convert{\methz\m\C\K} 

\hrulefill

\opdef{\classoff\C\K}{Auxiliary function: Type definition}

\begin{equation*}
\classoff\C\K = \Mxt{MT} ~~s.t.~~ \begin{cases}

 \K(\C) = \Class \C {\b{\Ftype\f\t}}{\b\md} \\
 \Mxt{MT} = \{\convert{\b\md} \oplus \forall ~\Ftype\fp\tp \in \b{\Ftype\f\t} ~|~ \fp \notin \b{\names{\md}} ~.~ \convertFD{\Ftype\fp\tp}\}

\end{cases}
\end{equation*}

\hrulefill


\opdef{\classoffs\a\s\K}{Auxiliary function: Type definition}

\begin{equation*}
\classoffs\a\s\K = \Mxt{MT} ~~s.t.~~ \begin{cases}

 \s(\a) = \obj\C{\b\a} \\
 \K(\C) = \Class \C {\b{\Ftype\f\t}}{\b\md} \\
 \Mxt{MT} = \{\convert{\b\md} \oplus \forall ~\Ftype\fp\tp \in \b{\Ftype\f\t} ~|~ \fp \notin \b{\names{\md}} ~.~ \convertFD{\Ftype\fp\tp}\}

\end{cases}
\end{equation*}

\hrulefill

\opdef{\field\C\K}{Auxiliary function: Field definition}

\begin{equation*}
\field\C\K = \b{\Ftype\f\t} ~~s.t.~~ \begin{cases}

 \K(\C) = \Class \C {\b{\Ftype\f\t}}{\b\md}
\end{cases}
\end{equation*}

\hrulefill

\opdef{\method\C\K}{Auxiliary function: Method definition}

\begin{equation*}
\method\C\K = \b\md ~~s.t.~~ \begin{cases}

 \K(\C) = \Class \C {\b{\Ftype\f\t}}{\b\md}\\
\end{cases}
\end{equation*}

\hrulefill

\opdef{\methods\a\s\K}{Auxiliary function: Method definition}

\begin{equation*}
\methods\a\s\K = \b\md ~~s.t.~~ \begin{cases}

 \s(\a) = \obj\C{\b\a} \\
 \K(\C) = \Class \C {\b{\Ftype\f\t}}{\b\md}\\
\end{cases}
\end{equation*}

% \hrulefill

% \opdef{\classof\a\s\K}{Auxiliary function: Type definition}
% 
% \begin{equation*}
% \classof\a\s\K = \classoff\C\K ~~s.t.~~ \begin{cases}
% 
%  \s(\a) = \obj\C{\b\a}\\
% 
% \end{cases}
% \end{equation*}

\hrulefill

\opdef{\readf \s\a\f}{Auxiliary function: Field read}

\begin{equation*}
\readf \s\a{\f_{i}} = \a_{i} ~~s.t.~~ \begin{cases}

 \s(\a) = \obj\C{\dots \a_{i} \dots}\\
 \K(\C) = \Class \C {\dots \Ftype{\f_{i}}{\t_{i}} \dots}{\b\md}\\
\end{cases}
\end{equation*}

\hrulefill

\opdef{\setf \s\a\f\ap}{Auxiliary function: Field write}

\begin{equation*}
\setf \s\a{\f_{i}}\ap = \sp ~~s.t.~~ \begin{cases}

 \s(\a) = \obj\C{\dots \a_{i} \dots}\\
 \K(\C) = \Class \C {\dots \Ftype{\f_{i}}{\t_{i}} \dots}{\b\md}\\
 \sp = \s[\Bind{\a}{\obj\C{\dots \ap \dots}}]
\end{cases}
\end{equation*}

\hrulefill

\opdef{\convert{\md}}{Conversion function: Method definition to method type}

\convert{\Mdef\m\x\t\t\e} = \Mtype\m\t\t 

\convert{\Mdef\f\x\t\t\e} = \Mtype\f\t\t 

\convert{\Mdefz\f\t\e} = \Mtype\f{}\t 

\hrulefill

\opdef{\convertFD{\Ftype{\f}{\t}}}{Conversion function: Field definition to field types}

\convertFD{\Ftype{\f}{\t}} = \Mtype\f\t\t, \Mtype\f{}\t

\hrulefill

\opdef{\names{\mt}}{Naming function: Method types}

\names{\Mdef\m\x\t\t\e} = \m

\names{\Mdef\f\x\t\t\e} = \f

\names{\Mdefz\f\t\e} = \f

\hrulefill

\opdef{\names{\md}}{Naming function: Method definition\footnote{All of these simply functions should go into the appendix.}}

\names{\Mtype\m\t\t} = \m

\names{\Mtype\f\t\t} = \f

\names{\Mtype\f{}\t} = \f


% 
% \hrulefill
% 
% \opdef{\getmdef({\m,\C,\K})}{Gets the definition of \m in \C with respect to \K}
% 
% $\getmdef({\m,\C,\K_1 ~ \Class\C\fd{\md_1 ~ \Mdef\m\x{\t_1}{\t_2}{ \e } ~ \md_2} ~ \K_2}) = \Mdef\m\x{\t_1}{\t_2}{ \e }$
% 
% \hrulefill
% 
\newcommand{\getmtype}{\text{mtype}}
% 
% \opdef{\getmtype({\m,\C,\K})}{Gets the type of \m in \C with respect to \K}
% 
% $\getmtype({\m,\C,\K_1 ~ \Class\C\fd{\md_1 ~ \Mdef\m\x{\t_1}{\t_2}{ \e } ~ \md_2} ~ \K_2}) = \Mtype\m{\t_1}{\t_2}$

\section{Generative Casts}

\subsection{Behavioral}



\hrulefill

\begin{mathpar}
\IRule{Cast-Wrap}{
  \D\;\mathit{fresh} \\
  \ap\;\mathit{fresh} \\
  \obj\C{\b{\a_1}} = \s(\a) \\
  \sp = \Heap \s{\Bind \ap{\obj\D{\a}} } \\
  \k = \wrapper{\C,\t,\D}
}{
  \behcast\a\t\s\K {\K,~\Bind{\D}{\k}~}\ap\sp
}
\end{mathpar}

What does the wrap function do? In english.

At its core, the wrap function protects types. If a function on an object has a declared argument type, then the wrap function will produce a wrapper for that function that ensures that any argument passed in matches the declared type. Likewise, if we assert that an untyped function has a return type, then the generated wrapper guarantees that the returned value of the untyped function is of the asserted type by inserting a cast to the right type.

In our context, wrappers are just generated classes, produced when the runtime system encounters a behavioral cast. These casts need to ensure two key properties, which we will refer to as \emph{soundness} and \emph{completeness}. In the context of casts, soundness refers to correct enforcement of types, whereby protected methods will not observably violate their type guarantees, while completeness ensures that a cast will not lose methods.

This last requirement is somewhat unconventional, and its need is illustrated in a simple example. 

\begin{verbatim}
class C {
  m(x:int):int { ... }
}
class D {
  m(x:*):* { ... }
  f(x:*):* { ... }
}
(<D>(<C>new D()))@f(2)
\end{verbatim}

In this example, if we were to implement wrappers by wrapping over only the declared methods, despite D having a method f, it is "lost" when D is cast to C. As a result, when we cast it back from C to D, the wrapper added to ensure that C's invariants held does not have a method f, and our call will produce a method not understood exception. To avoid this issue, our wrappers will have to retain all untyped methods when casting untyped to typed, and typed methods if a typed cast does not mention them, so that they may be recovered in a later cast.

\hrulefill

\newcommand{\tfa}{\text{\emph{for every }}}
\newcommand{\wh}{\text{\emph{where}}}
\newcommand{\wc}{\text{wrapClass}}
\newcommand{\n}{\xt{n}}

\begin{align*}
\xt{n} &::= \m \B \f
\end{align*}

\begin{tabbing}
$\wrapper{\C,$\=$ \Cp, \D} = $\\
\>$\xt{class}~\D$\=$~\{$\\
\>\>$\xt{th}$\=$\xt{at} : \C $ \\
\>\>$\Mdefa\n{\b{\HT\x{\t_2}}}{\tp_2} { \BehCast{\tp_2}{\this.\that.\n(\b{\BehCast{\t_1}{\x}})}}$\\
\>\>\>$\tfa \Mtype\n{\b{\t_1}}{\tp_1} \in \classoff\C\K \wedge \Mtype\n{\b{\t_2}}{\tp_2} \in \classoff\Cp\K$\\
\\
\>\>$\Mdefa\n{\b{\HT\x{\t_1}}}{\tp_1} { \this.\that.\n(\b{\x})}$\\
\>\>\>$\tfa \Mtype\n{\b{\t_1}}{\tp_1} \in \classoff\C\K \wedge \Mtype\n{\b{\t_2}}{\tp_2} \not\in \classoff\Cp\K$\\
\\
\>\>$\Mdefa\n{\b{\HT\x{\t_2}}}{\tp_2} { \New{\xt{Error}}{}@\xt{error}()}$\\
\>\>\>$\tfa \Mtype\n{\b{\t_1}}{\tp_1} \not\in \classoff\C\K \wedge \Mtype\n{\b{\t_2}}{\tp_2} \in \classoff\Cp\K$\\
\>$\}$
\\
\\
$\wrapper{\C,$\=$ \any, \D} = $\\
\>$\xt{class}~\D$\=$~\{$\=\\
\>\>$\xt{that} : \C$ \\ \\
\>\>$\Mdefa\n{\b{\HT\x\any}}{\any}{\BehCast{\any}{\this.\xt{that}.\n(\b{\BehCast{\t}{\x}})}}$ \\
\>\>\>$\tfa \Mtype\n{\b{\t}}{\tp} \in \classoff\C$\\
\>$\}$\\
\end{tabbing}


\wc is more complex than TODO, because different wrappers are needed, depending on if the target type is a concrete type \C, or is the dynamic type \any. If the target type is \C, then it will generate the wrapper methods for the type \C, then insert ``passthrough'' methods. methods of the same type as the original and that just call the original internally, that provide the completness property mentioned above. Likewise, if the target type is \any, then \wc will simply generate a method that casts its argument to the right type and its return type to \any for every one of the typed methods in the source, ensuring soundness and completness.


\subsection{Monotone}

%The motivation behind the monotonic semantics is that we must ensure that every reference to an object is still valid up to \any types, and as a result, can only proceed by replacing \any types with concrete types, possibly recursively. We accomplish this by \emph{rewriting} the class that objects are associated with, enforcing the new type guarantee, while simultaneously checking that all of the extant field values match the type that they need to be.

%In order to do any of this, however, the new type of the object is required. In Siek et al.'s original formalism, the new type was computed simultaneously as the object was being cast, but this creates problems when recursive objects are encountered. For example, consider the following scenario:

%\begin{verbatim}
%class C { x:any y:any m(y:any):any }
%class D {}
%class E { x:{x:{y:int}} y:any}

%a = new C(new D(),2)
%b = new C(a, 2)
%a.x(b)
%c = <E>a
%\end{verbatim}

%Here, when we go to update the value at \x, we then need to go into the type recursively, which points to a new instance of \C. Then, we assert a new type for \x in that new \C, which sends us back to the starting object, where we update the type of \xt{y} to be int, from \any. As a result, we cannot cast simply by examining the statically knowable types, but have to consider the heap structure when determining the final typing for any object.

%We break this operation up into two steps:

%\begin{itemize}
%\item \emph{Type Determination}. We first determine a valid heap typing that both retains the monotonic property, while satisfying the new cast assertion. To do this, we recursively examine the heap structure, iterating until the heap is unchanged by the last type alteration.
%\item \emph{Heap Update}. In the previous phase, we only determined a new valid typing for the heap. In the second step, we take this typing and apply it to the objects that exist in the heap.
%\end{itemize}

%By splitting out the two concerns, we avoid issues where the future state of the heap depends on values produced in the casting process and that do not exist yet. 

Monotonic aims to have \emph{every typed reference still be valid}, including the ones that previously existed, as well as the reference created by the cast. If a reference has a type with a non-star value, then the monotonic semantics will ensure that that type is not violated. To accomplish this, the monotonic cast needs to make sure of two properties:
\begin{itemize}
\item The values that \emph{currently} exist cannot violate any of the types that point to them. Casting needs to recursively ensure that all values referred to by the current object are of the claimed type.
\item Functions can not be called with or return values that violate any of the types that they are referred to with. The behaviour of the class needs to check that its types are not violated by lesser-typed call sites.
\end{itemize}
The second property is strongly reminisicent of the behavioural semantics, though with the interesting caveat that we now need to make sure that \emph{all} method invocations follow the typed calling conventions, rather than just the ones that inherit this particular type assertion.


\hrulefill

\newcommand{\rectype}[1]{\xt{rectype}(#1)}
\newcommand{\htype}[1]{\xt{htype}(#1)}
\newcommand{\fieldtypes}[1]{\xt{fieldtypes}(#1)}
\newcommand{\typez}[1]{\xt{types}(#1)}
\newcommand{\typeof}[1]{\xt{typeof}(#1)}
\newcommand{\classgen}[1]{\xt{classgen}(#1)}

\begin{align*}
\Sigma &::= \b{\a:\t}
\end{align*}

\begin{mathpar}
\IRule{Cast-Mono (monotonic)}{
  \rectype{\cdot, \cdot, \s, \a, \t} = \Sigma,\Kp\\
  \spec{\Sigma, \s, \Kp} = \sp
}{
  \moncast \a\t\s\K   \Kp {\sp}\\
}
\end{mathpar}

\hrulefill

\begin{mathpar}
\IRule{HT1}{
  \a \not\in \text{addr}(\Sigma)
}{
  \htype{\a,\Sigma,\sigma[\a \mapsto \C\{\b\ap\}]} = \C
}

\IRule{HT2}{
}{
  \htype{\a,\Sigma~\a:\t~\Sigma,\sigma} = \t
}
\end{mathpar}

\hrulefill

\renewcommand{\P}{\Mxt P}
\newcommand{\dmeet}[2]{#1\overset{\rightharpoonup}{\sqcap}#2}
\begin{mathpar}
\IRule{RC1}{
  \htype{\a,\Sigma,\sigma} = \tp\\
  \Sigma,\K \vdash \dmeet{\tp}{\t} \equiv \tpp \dashv \K' \\
  \typeof\tp \neq \typeof\tpp \\
  \fieldtypes{\tpp,\K,\s,\a} = \b\ap,\b{\t_f}\\
  \b{\rectype{\meet\Sigma{\ap:\tpp},\K',\sigma,\ap,\t_f} = \Sigma'~\K''}
}
{\rectype{\Sigma, \K, \sigma, \a, \t} = \left(\bigsqcap\b{\Sigma'}\right)~\K''}

\IRule{RC2}{
  \htype{\a,\Sigma,\sigma} = \tp\\
  \emptyset,\K \vdash \dmeet{\tp}{\t} \equiv \tpp \dashv \K'\\ % is this Omega right?
  \typeof\tp = \typeof\tpp \\
}{
  \rectype{\Sigma,\K,\sigma, \a, \t} = \Sigma~\K
}
\end{mathpar}

\hrulefill

\begin{mathpar}
\IRule{SP1}{
            \spec{\Sigma,\sigma,\K} = \sigma'}{
            \spec{\a : \D ~ \Sigma, \sigma[\a \mapsto \C\{\b{\ap}\}], \K} = \sigma'[\a\mapsto \D\{\b\ap\}]
            }


\IRule{SP2}{}{
            \spec{\cdot, \sigma, \K} = \cdot
            }
\end{mathpar}

\hrulefill

%fixme
\opdef{
  $\P~\K\vdash \dmeet{\t_1}{\t_2} \equiv \t \dashv \K$
}{
  Specializing $\t_1$ with $\t_2$ to produces $\t$ and $\K'$, under \P and \K 
}

\begin{mathpar}
\IRule{M1}{ }{\P~\K \vdash\dmeet{\t}{\any} \equiv \t\dashv\K}

\IRule{M2}{ }{\P~\K \vdash\dmeet{\any}{\t} \equiv \t\dashv\K}

\IRule{M3}{ }{\P~\K \vdash\dmeet{\t}{\t} \equiv \t\dashv\K}

\IRule{M4}{
      \b{\P~\K\vdash\dmeet{\t_3}{\t_1}  \equiv \t_5\dashv \K'} \\
    \P~\K' \vdash \dmeet{\t_2}{\t_4} \equiv \t_6\dashv \K''\\
    \P~\K'' \vdash \dmeet{\Type{\b{\mt_1}}}{\Type{\b{\mt_2}}} \equiv \Type{\b{\mt_3}} \dashv \K'''
}{
  \P~\K \vdash \dmeet{\Type{\Mtype\n{\b{\t_1}}{\t_2}~ \b{\mt_1}}}{\Type{\Mtype\n{\b{\t_3}}{\t_4} ~ \b{\mt_2}}} \equiv \Type{\Mtype\n{\b{\t_5}}{\t_6} ~ \b{\mt_3}}\dashv\K'''
} % Did we get co/contravariance right?

\IRule{M5}{ 
}{ \P~\K \vdash \dmeet{\Type{\b\mt}}{\Type{}} \equiv \Type{\b\mt} \dashv \K}

\IRule{M7}{
  \Mxt E \text{ fresh}\\
  \dmeet{\C}{\D} \not\in \text{dom}(\P)\\
  \P~\dmeet{\C}{\D}={\Mxt E}~\K \vdash \dmeet{{\text{typeof}}(\C,\K)}{{\text{typeof}}(\D,\K)} \equiv \t \dashv \K'\\
  \K'' = \K~\Mxt E\mapsto\classgen{\C,\t,\Mxt E}
}{
  \P~\K \vdash \dmeet{\C}{\D} \equiv {\Mxt E} \dashv \K''
}

\IRule{M8}{
  \P(\dmeet{\C}{\D}) = \Mxt E
}{
  \P~\K \vdash \dmeet{\C}{\D} \equiv {\Mxt E} \dashv \K
}
\end{mathpar}

The figure above is semantics for the meet operator for the semantics of the monotonic cast.

\begin{verbatim}
class C:
	def m1(self,x:C)->Dyn:
		return 2

class E:
	def m1(self,x:E)->Int:
		return 2

c = C()
e = castE(c)
c.m1(c)
\end{verbatim}


\hrulefill

\begin{mathpar}
\IRule{FT1}{
 \s(\a) = \obj\C{\b\a} \\
 \K(\C) = \Class \C {\b{\Ftype\f\t}}{\b\md} \\
 $|~\b\a|~$ = $~|~\b{\Ftype\f\t}|$ \\
}{
  \fieldtypes{\C,\K,\s,\a} = \b\a~\typeof{\b{\Ftype\f\t}}
}

\IRule{FT2}{
 \s(\a) = \obj\C{\b\a} \\
 \K(\C) = \Class \C {\b{\Ftype\f\t}}{\b\md} \\
 $|~\b\a|~$ = $~|~\b{\Ftype\f\t}|$ \\
}{
  \fieldtypes{\any,\K,\s,\a} = \b\a~\typeof{\b{\Ftype\f\t}}
}

% The types are not the types shown in the class, but those in the mt(s)
\IRule{FT3}{
 \s(\a) = \obj\C{\b\a} \\
 \K(\C) = \Class \C {\b{\Ftype\f\t}}{\b\md} \\
 \forall~ \f \in \names{\b{\Ftype\f\t}} ~.~ \f\in\names{\{\b\mt\}} \\ 
 \b{\t_f} = \forall~ \f \in \names{\b{\Ftype\f\t}} ~.~ \typez{\f, \{\b\mt\}}
}{
  \fieldtypes{\{\b\mt\},\K,\s,\a} = \b\a~\b{\t_f}
}
\end{mathpar}

\hrulefill

\begin{mathpar}
\IRule{TY}{
}{
  \typez{\f,~\dots ~\Mtype\f\t\t~ \dots ~\Mtype\f{}\t~ \dots} = \t
}
\end{mathpar}

\hrulefill

\begin{mathpar}
\IRule{TO}{
}{
  \typeof{\Ftype\f\t} = \t
}
\end{mathpar}

\hrulefill

\begin{tabbing}
$\classgen{\C,$\=$ ~\t,$\=$ ~\D, \K} = $\\
\>$\xt{class}~\D$\=$~\{$\=\\
\>\>$\f : \tp_1$\\
\>\>\>$\tfa \f : \t_1 \in \classoff\C\K \wedge \f:\tp_1 \in \typez{\t}$\\
\\
\>\>$\Mdefa\n{\b{\HT\x{\any}}}{\any} { \MonCast{\any}{\this.\n(\b{\MonCast{\t_1}{\x}})}}$\\
\>\>\>$\tfa \Mtype\n{\b{\t_1}}{\tp_1} \in \classoff\t\K$\\
\\
\>\>$\Mdefa\n{\b{\HT\x{\t_1}}}{\tp_1} { \MonCast{\tp_1}{\e}}$\\
\>\>\>$\tfa \Mtype\n{\b{\t_1}}{\tp_1} \in \classoff\t\K \wedge \Mdefa\n{\b{\HT\x{\any}}}{\any}\e \in \xt{methods}(\C,\K)$\\
\>$\}$
\end{tabbing}

\section{Translations}

Our core language supports the core functionality of gradual typing, but none of the surface syntax or quality of life features that full gradual tyyping systems do. To provide a representative surface syntax, we define several translations that convert surface syntax programs in an approximation of the source language into core language ones. 

\subsection{Strongscript}

\newcommand{\bC}{\xt{!C}}

\begin{tabular}{l@{~~}l@{}l@{}l}
\t  &::= ~ \any \B \C \B \bC \\ 
\end{tabular}


\begin{mathpar}
\IRule{A1}{\HasType{\E}\x\t}{\GenCast\E\x\x\t}

\IRule{A2}{
    \GenCast{\E}{\e_1}{\e_3}{\bC} \\ \m(\t_1):\t_2 \in \bC \\ \AnaCast{\E}{\e_2}{\e_4}{\t_1}
}{
    \GenCast{\E}{\Call{\e_1}\m{\e_2}}{\Call{\e_3}\m{\e_4}}{\t_2}
}

\IRule{A3}{
    \GenCast{\E}{\e_1}{\e_3}{\bC} \\ \f(\t_1):\t_1 \in \bC \\ \AnaCast{\E}{\e_2}{\e_4}{\t_1}
}{
    \GenCast{\E}{\Call{\e_1}\f{\e_2}}{\Call{\e_3}\f{\e_4}}{\t_1}
}

\IRule{A4}{
    \GenCast{\E}{\e_1}{\e_2}{\bC} \\ \f():\t_1 \in \bC
}{
    \GenCast{\E}{\Call{\e_1}\f{}}{\Call{\e_2}\f{}}{\t_1}
}

\IRule{A5}{
    \GenCast{\E}{\e_1}{\e_3}{\C} \\ \m(\t_1):\bC \in \C \\ \AnaCast{\E}{\e_2}{\e_4}{\t_1} 
}{
    \GenCast{\E}{\Call{\e_1}\m{\e_2}}{\SubCast{\C}{\DynCall{\e_3}\m{\e_4}}}{\bC} %Q: do I need to cast the return value of non-bang
}

\IRule{A6}{
    \GenCast{\E}{\e_1}{\e_3}{\C} \\ \m(\t_1):\C \in \C \\ \AnaCast{\E}{\e_2}{\e_4}{\t_1} 
}{
    \GenCast{\E}{\Call{\e_1}\m{\e_2}}{\DynCall{\e_3}\m{\e_4}}{\C} %Q: do I need to cast the return value of non-bang
}

\IRule{A6}{
    \GenCast{\E}{\e_1}{\e_3}{\C} \\ \f(\t_1):\t_1 \in \C \\ \AnaCast{\E}{\e_2}{\e_4}{\t_1}
}{
    \GenCast{\E}{\Call{\e_1}\f{\e_2}}{\DynCall{\e_3}\f{\e_4}}{\t_1} 
}

\IRule{A7}{
    \GenCast{\E}{\e_1}{\e_2}{\C} \\ \f():\t_1 \in \C
}{
    \GenCast{\E}{\Call{\e_1}\f{}}{\DynCall{\e_2}\f{}}{\t_1} 
}

\IRule{A8}{
    \GenCast{\E}{\e_1}{\e_3}{\any} \\ \AnaCast{\E}{\e_2}{\e_4}{\any}
}{
    \GenCast{\E}{\Call{\e_1}\m{\e_2}}{\DynCall{\e_3}\m{\e_4}}{\any}
}

\IRule{A9}{
    \GenCast{\E}{\e_1}{\e_3}{\any} \\ \AnaCast{\E}{\e_2}{\e_4}{\any}
}{
    \GenCast{\E}{\Call{\e_1}\f{\e_2}}{\DynCall{\e_3}\f{\e_4}}{\any}
}

\IRule{A10}{
    \GenCast{\E}{\e_1}{\e_2}{\any}
}{
    \GenCast{\E}{\Call{\e_1}\f{}}{\DynCall{\e_2}\f{}}{\any}
}

\IRule{A11}{
  \b{\AnaCast\E{\e_1}{\e_2}\t} \\ 
  \Class \C {\b{\Ftype\f\t}} {\b{\md}}
  }{\GenCast{\E}{\New\C{\b{\e_1}}}{\New\C{\b{\e_2}}}{\C}}
\end{mathpar}

\hrulefill

\begin{mathpar}
\IRule{AASC1}{
  \GenCast{\E}{\e_1}{\e_2}{\bC_2} \\
  {\C_2}\Sub{\C_1}\\
}{
  \AnaCast{\E}{\e_1}{\e_2}{\bC_1}
}

\IRule{AASC2}{
  \GenCast{\E}{\e_1}{\e_2}{\bC_2} \\
  {\C_2}\Sub{\C_1}\\
}{
  \AnaCast{\E}{\e_1}{\e_2}{\C_1}
}

\IRule{AASC3}{
  \GenCast{\E}{\e_1}{\e_2}{\C_2} \\
  {\C_2}\Sub{\C_1}\\
}{
  \AnaCast{\E}{\e_1}{\e_2}{\C_1}
}

\IRule{AASC4}{
  \GenCast{\E}{\e_1}{\e_2}{\C} \\
}{
  \AnaCast{\E}{\e_1}{\SubCast{\C}{\e_2}}{\bC}
}

\IRule{AASC5}{
  \GenCast{\E}{\e_1}{\e_2}{\any} \\
}{
  \AnaCast{\E}{\e_1}{\SubCast{\C}{\e_2}}{\bC}
}

\IRule{AASC6}{
  \GenCast{\E}{\e_1}{\e_2}{\any} \\
}{
  \AnaCast{\E}{\e_1}{\e_2}{\C}
}

\IRule{AASC7}{
  \GenCast{\E}{\e_1}{\e_2}{\t} \\ \t \neq \any
}{
  \AnaCast{\E}{\e_1}{\e_2}{\any}
}
\end{mathpar}

\end{document}


%\titlerunning{\it Evaluating the Design of R}
%\authorrunning{\it Morandat et al.}

\begin{document}
\title{The Design Space of Gradual Type Systems} 
\authorinfo{Ben and Jan}{Northeastern University}{EMAILS}
\maketitle


\conferenceinfo{CONF 'yy}{Month d--d, 20yy, City, ST, Country}
\copyrightyear{20yy}
\copyrightdata{978-1-nnnn-nnnn-n/yy/mm}
\copyrightdoi{nnnnnnn.nnnnnnn}

\begin{abstract}
The popularity of dynamically typed languages has given rise to a cottage
industry of incremental type systems. These type systems allow developpers
to annotate their code with types and to feel better about the correctness
of their code.  The exact properties that these incremental type systems are
able to guarantee vary greatly from one language to another. This paper
compares some of the main design choices by embedding them into one common
core calculus representative of dynamic object-oriented languages such as
Python, TypeScript or Ruby.

\end{abstract} 

\section{Introduction}

The fundamental property a sound type system provides is the guarantee that an 
expression such as 
\begin{lstlisting}
    x = y.m(z);
\end{lstlisting}
will not ``go wrong''. If the expression passes the typechecker and the
variable \code{x} is declared to be of some type \code{T}, setting aside
issues related to null pointers and non-termination, the result of
\code{m()} will be of type \code{T}.  This is not the case in a dynamic
language where the receiver \code{y} may lack the requested method, or may
expect an argument of a different type, or may return a value that does not
belong to type \code{T}.

For any dynamic language, from the early days of Lisp to more recent times
with JavaScript, there have been attempts to add annotations to document
programmer expectations about the nature of values flowing through their
code.  The motivation for burdening the language with these extra
annotations have been either to provide hints for a just-in-time compiler or
to help programmers catch errors early.

\newcommand{\Opt}[1]{#1$^{?}$\xspace}
\newcommand{\Con}[1]{#1$^{!}$\xspace}
\newcommand{\Pro}[1]{#1$^{()}$\xspace}
\newcommand{\dyn}{$\star$}

In the last decade, the design space for these incremental annotations seems
to have stabilized around three alternatives.  A number of systems offer
\emph{optional types}, \Opt{T},
\cite{PluggableTypes,Bracha93,typescript13,oopsla09} which provide local
guarantees but do not prevent type errors at call boundaries. Other systems
rely on \emph{promised types}, \Pro{T}, these type annotations are
associated to the values that flow through the program and represent a
promise that the value will either behave as if it was of that type, or that
type error will be emitted~\cite{siek14,tf-popl08}.  Lastly, some systems
offer \emph{concrete types} which provide the traditional soundness
guarantees one would expect in a statically typed
language~\cite{thorn,stongscript,csharp}. In addition all dynamic languages
have the type dyn (\dyn) as the default type for all dynamically typed
variables.

The design space of incremental type system is an interesting one as there
are seemingly important trade-offs to be made in three different dimensions:
expressiveness, assurance and performance.  By expressiveness, we mean how
much of the legacy code base can be typed (and at what cost in terms of
changes to the original programs).  By assurance, we mean what kinds of
guarantees does the type system give to programmers. At its weakest, the
type system can be little more than a machine-checked documentation, and at
its strongest it can be equivalent to a sound statically typed language.
Performance here refers to the additional runtime overheads that are caused
by the extra checks that the language must perform to provide the
aforementioned guarantees.

\newcommand{\name}{{\sf Gool}\xspace}

This paper elucidates the difference between different designs by setting
them all in a common context. We have picked a simple class based
object-oriented language, representative of the likes of Python, Ruby,
TypeScript. Our core calculus, named \name for Gradual Object-oriented
Language, is stateful, as mutation introduces additional challenges. \name
is agnostic as to the nature of the subtyping relation (structural or
nominal) as this seems to be an orthogonal design choice.

\section{Related Work}

Typescript~\cite{typescript13} is an extension of Javascript, enabling existing 
Javascript code to be typed without substantial changes. In order to do so, 
Typescript uses an optional type system, only providing type guarantees within 
local code, not across call boundaries. This approach eases the transition from
Javascript to typed code, but the guarantees provided are very weak, only 
ensuring type saftey in local contexts, restricting optimization as well as
limiting the information provided by a type signature.

The example of a concrete type sysetm we are considering in this work is
Strongscript~\cite{stongscript}. Strongscript builds on lambda JS and 
Typescript, strengthening the guarantees provided in the latter by providing
\emph{concrete} types, represented by an excalmation mark. Within Strongscript,
concrete types impose guarantess about the runtime values that they represent, 
a stronger guarantee than that provided by Typescript, while still allowing
programmers the flexibility of the Typescript approach, though with fewer
guarantees.

Another approach is that taken by~\cite{seik-monotonic} (Seik Monotonic)
This approach presents a system that takes a \emph{promised type} approach.
Its novelty comes from a new approach to avoiding expensive and complex
wrappers. Instead of ensuring that the type of references remain consistent
by adding a wrapper, this approach works by ensuring that the type of values in
the heap only get \emph{less} generic.

Within this approach, references that were well-typed previously will always
remain well-typed, as the new type is strictly less generic than the one that
it had at the time that the reference was checked. By doing this, the monotonic
approach can avoid many of the expensive checks that wrapper approaches incur, 
while still providing similar functionality. However, this approach is less
generic than wrapper-based systems, as the monotonic constraint is much stronger
than the more usual well-typed one.

\section{Core Calculus}

\begin{figure}
\syntax

\caption{Abstract Syntax for the Core Calculus}
\end{figure}

The core calculus is a simple object based system, including classes and 
restricted mutability. We have declarations $d$, which act as the high level 
building blocks, types $\tau$, and expressions $e$. This construction is 
similar to the core calculus in StrongScript, though with some major 
modifications.

In our calculus, a class is represented by a \emph{class declaration}, which 
includes the nominal relationships between that class and its parents. Then, we 
represent fields as labels with types, and notate methods as a label, a list 
of typed parameters, a return type, and a method body. This construction models 
many other object systems while remaining fairly trivial.

Our types are modeled after those in StrongScript. In addition to the basic 
types $\intt$ and $\strt$, we have the $\anyt$ type, representing a dynamic 
value, as well as two kinds of class type. $C$ represents a weaker, optional, 
class type with few guarantees. Only finding a $C$ implies very little about 
the eventual value that is recieved at runtime. In contrast, however, $!C$ is 
a concrete type, one with strong static guarantees that will hold at runtime. 

Our expression grammar is fairly simple, with a few interesting components. 
Mutability is expressed our system by altering fields, which are only 
accessible from inside an object. This is expressed in our expression
grammar by having get and set forms only involving labels. 

\section{Examples}


\begin{figure}
\opdef{$\t <: \t'$}{$\t$ is a subtype of $\t'$}
\begin{mathpar}
\inferrule*[lab=SClass]{
  \CLASS \C {\bar{\xt{D}}} {\ldots}
}{\C <: \xt{D}}

\inferrule*[lab=STrans]{
  \C <: \xt{D} \\ \xt{D} <: \xt{E}
}{\C <: \xt{E}}
\end{mathpar}
\opdef{$\Gamma\vdash e : \t$}{$e$ has type $\t$ in context $\G$}
\begin{mathpar}
\TVar

\TInt

\TSub

\TUApp

\TApp

\TIf

\TPlus

\TNew

\TCast
\end{mathpar}
\opdef{$\Gdash e \tlate e' : \t$}{$e$ translates to $e'$ producing type $\t$ in context $\G$}
\opdef{$\CICAST{e}{\t}{e'}$}{$e$ translates to $e'$ ensuring type $\t$ in context $\G$}
\begin{mathpar}
\CICast

\CICallAny

\CICall

\CIIf

\CIPlus

\CIVar

\CINew
\end{mathpar}
\caption{Common Statics}
\end{figure}


\begin{figure}
\opdef{$e,\s \ereduce e',\s'$}{$e$ with heap $\s$ evaluates to $e'$ and $\s$ in one step}
\begin{mathpar}
\EFrame

\ENew

\EInv

\EField

\EAssign

\EIfZ

\EIfNZ

\EPlus

\ECastInt

\ECastAny
\end{mathpar}
\caption{Common Dynamics}
\end{figure}

\section{Monotonic Semantics}

\begin{figure}
\opdef{$\tau \sim \tau'$}{$\tau$ is compatible with $\tau'$}
\begin{mathpar}
\inferrule*[lab=CNSym]{ }{\tau \sim \tau}

\inferrule*[lab=CNCons1]{ }{\any \sim \tau}

\inferrule*[lab=CNCons2]{ }{\tau \sim \any}

\inferrule*[lab=CNSub1]{
  \CLASS \C {\ldots} {\bar{f:\t_C},\bar{d(\bar{\xt{x}:\t_C''}):\t_C'=e}} \\
  \CLASS {\xt{D}} {\ldots} {\bar{f:\t_D},\bar{d(\bar{\xt{x}:\t_D''}):\t_D'=e}} \\
  \bar{\t_C \sim \t_D}\\
  \bar{\t_C' \sim \t_D'}\\
  \bar{\t_C'' \sim \t_D''}}{\C \sim \xt{D}}
\end{mathpar}
\opdef{$\tau \sim \tau'$}{$\tau$ is subtype compatible with $\tau'$}
\begin{mathpar}
\inferrule*[lab=CSSub]{\C <: \xt{D}}{ \C \lesssim \xt{D}}

\inferrule*[lab=CSCons]{\tau \sim \tau'}{\tau \lesssim \tau'}

\inferrule*[lab=CSTrans]{\tau_1 \lesssim \tau_2 \\ \tau_2 \lesssim \tau_3}{\tau_1 \lesssim \tau_3}
\end{mathpar}
\caption{Monotonic operators}
\end{figure}


\begin{figure}
\opdef{$\Gamma\vdash e : \t$}{$e$ has type $\t$ in context $\G$}
\begin{mathpar}
\inferrule*[lab=TClass]{ 
  \bar{\G \vdash \f : \t}\\
  \bar{\G \vdash \MTYPE{\m}{\bar{x:\t''}}{\t'} = e} \\ 
  \f:\t \in \xt{D} \implies \f:\t \in \C\\
  \MTYPE{d}{\bar{x:\t''}}{\t'}\in \xt{D} \implies \MTYPE{d}{\bar{x:\t''}}{\t'} \in \C\\
 }{
  \Gdash \CLASS \C{\xt{D}}{\bar{\f:\t},\bar{\MDEF d{\bar{x:\t''}}{\t'}e}}
  }
\end{mathpar}
\opdef{$\CICAST{e}{\t}{e'}$}{$e$ translates to $e'$ ensuring type $\t$ in context $\G$}
\begin{mathpar}
\inferrule*[lab=CIIDown]{\Gdash e \tlate e' : \xt{D} \\ \C <: \xt{D}}{\CICAST{e}{\C}{\cast{\C}e'}}

\inferrule*[lab=CIIUp]{\Gdash e \tlate e' : \C \\ \C <: \xt{D}}{\CICAST{e}{\xt{D}}{e'}}

\inferrule*[lab=CIICons]{\Gdash e \tlate e' : \t' \\ \t' \lesssim \t \\ \t \neq \t' }{\CICAST{e}{\t}{\cast{\t}e'}}

\inferrule*[lab=CIIId]{\Gdash e \tlate e' : \t}{\CICAST{e}{\t}{e'}}
\end{mathpar}
\opdef{$e,\s \ereduce e',\s'$}{$e$ with heap $\s$ evaluates to $e'$ and $\s$ in one step}
\begin{mathpar}
\inferrule*[lab=ECast]{ \s' = \text{cast}(\s, v, \t) }{ \cast{\t}a,\s \ereduce a,\s'}
\end{mathpar}
\caption{Evaluation rules for the monotonic semantics}
\end{figure}


\begin{figure}
\begin{mathpar}
\inferrule*[]{}{\t \spec \t = \t}

\inferrule*[]{}{\t \spec \any = \t}

\inferrule*[]{}{\any \spec \t = \t}

\inferrule*[]{B \not\in types(C)}{C \spec B = gen(C,B)}

\inferrule*[]{B \in types(C)}{C \spec B = C}
\end{mathpar}
\caption{Meet function for the monotonic semantics}
\end{figure}

\begin{figure}
\begin{mathpar}
\inferrule*[lab=CClass]{\s[a \is \{\bar{\f = v} \B \C\}] \\ 
\C \neq \C \spec \xt{D} \\ 
\s' = \s[a \is \{\bar{\f = v} \B \C \spec \xt{D}\}]\\
\f:\t \in \C\spec \xt{D} \implies \sigma'' = \text{cast}(\s', v, \t)}{\text{cast}(\s, a, \xt{D}) = \s''}

\inferrule*[lab=CCycle]{\s[a \is \{\bar{\f = v} \B \C\}] \\ \C = \C \spec \xt{D}}{\text{cast}(\s, a, \xt{D}) = \s}

\inferrule*[lab=CInt]{ }{\text{cast}(\s,\xt{int}[n],\xt{\int}) = \s}

\inferrule*[lab=CSub]{\s[a \is \{\bar{\f = v} \B \C\}]}{\text{cast}(\s, a, \C) = \s}
\end{mathpar}
\caption{Casting in the monotonic system}
\end{figure}


\begin{figure}
\begin{align*}
&fieldType(A,B,f) = \\
&\left\{\begin{array}{lcl}
\t_A \spec \t_B &:& f:\t_A \in fields(A) \wedge f:\t_B\in fields(B)\\
\t_A &:& f:\t_A \in fields(A) 
\end{array}\right.\\
&argType(A,B,m,x) =\\
&\left\{\begin{array}{lcl}
\t''_A \spec \t''_B &:& m(\ldots,{x:\t''_A},\ldots):\t'_A\in methods(A) \wedge \\
&&m(\ldots,x:\t''_B,\ldots):\t'_B \in methods(B)\\
\t''_A &:& m(\ldots,{x:\t''_A},\ldots):\t'_A\in methods(A) 
\end{array}\right.\\
&retType(A,B,m) = \\
&\left\{\begin{array}{lcl}
\t'_A \spec \t'_B &:& m(\ldots,{x:\t''_A},\ldots):\t'_A\in methods(A) \wedge \\
&&m(\ldots,x:\t''_B,\ldots):\t'_B \in methods(B)\\
\t'_A &:& m(\ldots,{x:\t''_A},\ldots):\t'_A\in methods(A) 
\end{array}\right.
\end{align*}
\caption{Monotonic class generation helper functions}
\end{figure}

\begin{figure}
\begin{lstlisting}
classMeet(A,B)
  A = class A {(*@\bar{\f_A:\t_A},\bar{m(\bar{x:\t''_A}):\t'_A = e_A}@*)}
  B = class B {(*@\bar{\f_B:\t_B},\bar{m(\bar{x:\t''_B}):\t'_B = e_B}@*)}
  return class AB {
    (*@\bar{\f:fieldType(A,B,\f)}@*),
    (*@\bar{f():fieldType(A,B,\f) = \xt{this}.f}@*),
    (*@\bar{m(\bar{argType(A,B,m,x)}):retType(A,B,m) =}@*)
    	(*@\bar{\cast{retType(A,B,m)}\this.m'(\bar{\cast{argType(A,B,m,x)}x})}@*)
  }
\end{lstlisting}
\caption{Monotonic class meet function}
\end{figure}

Define $types(C) = \{A_1,\ldots,A_n\}$ such that 

$classMeet(classMeet(A_1,A_2)\ldots,A_n) = C$

Define $gen(C, B) = classMeet(C,B)$

classMeet is applied to every string of classes $C_1,\ldots,C_n$ where $C_1 \lesssim C_i$ for $i=2,\ldots,n$.

\section{Wrapper Semantics}


\begin{figure}
\begin{mathpar}

\inferrule*[lab=TNew]{
  \bar{\Gdash e:\t} \\ 
  \{\ldots \bar{\f:\t} \ldots\} = fields(\C)\\
  typed(\C)
}{
  \Gdash \NEW \C{\bar e}:\C
}


\inferrule*[lab=TTypedClass]{ 
  \bar{\Gdash D} \\  
  {\MTYPE{\m'}{\bar{x:\t''}}{\t'} \in\D \implies \MTYPE{\m'}{\bar{x:\t''}}{\t'}\in\C }\\
  {\f:\t \in\D \implies \f:\t \in\C}\\
  {\bar{\Gdash \f:\t}}\\
  {\bar{\Gdash \MDEF\m{\bar{x:\t''}}{\t'}e}}
 }{
  \Gdash \CLASS \C{\bar\D}{\bar{\f:\t},\bar{\MDEF\m{\bar{x:\t''}}{\t'}e}}
  }  

\inferrule*[lab=TUntypedClass]{ 
  \bar{\Gdash D} \\  
  {\MTYPE{\m'}{\bar{x:\any}}{\any} \in\D \implies \MTYPE{\m'}{\bar{x:\any}}{\any}\in\C }\\
  {\f:\any \in\D \implies \f:\any \in\C}\\
 }{
  \Gdash \CLASS \C{\bar\D}{\bar{\f:\any},\bar{\MDEF\m{\bar{x:\any}}{\any}e}}
  } 
\end{mathpar}
\caption{Wrapper typing rules}
\end{figure}


\begin{figure}
\begin{mathpar}
\inferrule*[lab=CIIUni]{ }{\CICAST{e}{\t}{e}}
\end{mathpar}
\caption{Analytic cast insertion for the wrapper semantics}
\end{figure}

\begin{figure}
\begin{mathpar}
\inferrule*[lab=CTypedClass]{typed(\C)}{\Gdash \NEW \C {\bar v} \tlate \cast{\C}\NEW\C{\bar v} : \any}

\inferrule*[lab=CUntypedClass]{ }{\Gdash \NEW \C {\bar v} \tlate \NEW\C{\bar v} : \any}
\end{mathpar}
\caption{Analytic cast insertion for the wrapper semantics}
\end{figure}


\begin{figure}
\begin{mathpar}
\inferrule*[lab=ECastRef]{
\s' = \s[a' \is \{\xt{orig} = a \B \xt{\#}\C\}] }{ \cast{\C}a,\s \ereduce a',\s'}
\end{mathpar}
\caption{Evaluation for the wrapper semantics}
\end{figure}

\begin{figure}

\begin{lstlisting}
wrapper(A)
  A = class A {(*@\bar{\f_A:\t_A},\bar{m(\bar{x:\t''_A}):\t'_A = e_A}@*)}
  return class #A {
    (*@\xt{orig}:\xt{A}@*),
    (*@\xt{orig():\xt{A} = this.orig}@*),
    (*@$\bar{m(\bar{x:\any}):\any = \cast{\t'_A}\xt{this.orig()}.m(\bar{\cast{\t''_A}v})}$@*)
  }
\end{lstlisting}

For every typed class $\xt{A}$, we run $wrapper(\xt{A})$ to get a $\xt{\#A}$ that ensures the type invariants on A.
\caption{Wrapper generation for wrapper semantics}
\end{figure}

\section{Optional Semantics}

\begin{figure}

\begin{mathpar}
\inferrule*[lab=SubRefl]{ }{ \t <: \t }

\inferrule*[lab=SubClass]{\class~\C~\implements~\D}{ \xt{!C} <: \xt{!D} }

\inferrule*[lab=SubWeak]{ \xt{!C} <: \xt{!D} }{ \C <: \xt{D}}

\inferrule*[lab=SubInj]{ }{ \xt{!C} <: \C}
\end{mathpar}

\caption{Optional typing subtyping semantics}
\end{figure}

\begin{figure}
\begin{mathpar}
\inferrule*[lab=TCInj]{\Gamma \vdash e : \tau' \\ \tau' = \any \vee \tau = \any \vee \tau' <: \tau }{ \Gamma \vdash e : \tau}

\inferrule*[right=TClass]{ 
  \bar{\MTYPE{\m'}{\bar{x:\t''}}{\t'} \in\D \implies \MTYPE{\m'}{\bar{x:\t''}}{\t'}\in\C }\\
  \bar{\G,\this:\C,\bar{x:\t''} \vdash e:\t'}
 }{
  \Gdash \CLASS \C{\bar\D}{\bar{\f:\t},\bar{\MDEF\m{\bar{x:\t''}}{\t'}e}}
  }
\end{mathpar}

\caption{Optional typing typing}
\end{figure}

\begin{figure}
\begin{mathpar}
\inferrule*[lab=CIIInsert]{\Gdash e \tlate e' : \any }{\CICAST{e}{\t}{\cast{\t}e'}}

\inferrule*[lab=CIISub]{\Gdash e \tlate e' : \xt{D} \\ \xt{D} <: \C}{\CICAST{e}{\C}{e'}}

\inferrule*[lab=CIISuper]{\Gdash e \tlate e' : \xt{D} \\ \C <: \xt{D}}{\CICAST{\cast{\C}e}{\C}{e'}}

\inferrule*[lab=CIIAny]{\Gdash e \tlate e' : \tau}{\CICAST{e}{\any}{e'}}
\end{mathpar}

\caption{Optional typing analytic cast insertion}
\end{figure}

\begin{figure}
\begin{mathpar}
\inferrule*[lab=ECastClass]{ \s[a\is\{\ldots\B\xt{D}\}] \\ \xt{D} <: \C }{ \cast{\C}a,\s \ereduce a,\s}
\end{mathpar}

\caption{Optional typing evaluation}
\end{figure}

\section{Conclusion}

\bibliographystyle{plain}

\bibliography{bib/compactdoi}
\end{document}


