\documentclass[preprint]{sigplanconf}
\usepackage{xspace,listings,url,subfigure,framed,amssymb,
            amsmath,mathpartir,hyperref,
            stmaryrd % double brackets llbracket
}
\newcommand{\rn}[1]{#1}
\newcommand{\doi}[1]{doi:~\href{http://dx.doi.org/#1}{\Hurl{#1}}}

\conferenceinfo{CONF 'yy}{Month d--d, 20yy, City, ST, Country}
\copyrightyear{20yy}
\copyrightdata{978-1-nnnn-nnnn-n/yy/mm}
\copyrightdoi{nnnnnnn.nnnnnnn}

%%% Metavariables %%%%%%%%%%%%%%%%%%%
\newcommand{\fd}{\M{\xt{fd}}}
\newcommand{\md}{\M{\xt{md}}}
\newcommand{\mt}{\M{\xt{mt}}}
\newcommand{\m}{\M{\xt{m}}}
\newcommand{\e}{\M{\xt{e}}}
\newcommand{\n}{\M{\xt{n}}}
\renewcommand{\d}{\M{\xt{d}}}
\renewcommand{\r}{\M{\xt{r}}}
\newcommand{\f}{\M{\xt{f}}}
\newcommand{\fb}{\M{\xt{f!}}}
\newcommand{\x}{\M{\xt{x}}}
\renewcommand{\t}{\M{\xt{t}}}
\renewcommand{\c}{\M{\xt{c}}}
\newcommand{\C}{\M{\xt{C}}}
\newcommand{\D}{\M{\xt{D}}}
\newcommand{\this}{\M{\xt{this}}}
\newcommand{\err}{\M{\bt{err}}}
\renewcommand{\d}{\M{\xt{d}}}
\newcommand{\s}{\M{\sigma}}
\renewcommand{\a}{\M{\xt a}}
\newcommand{\T}{\M{\xt T}}
\newcommand{\tp}[1]{\M{ \t_{#1} }}
\newcommand{\ep}[1]{\M{ \e_{#1} }}
\newcommand{\ap}[1]{\M{ \a_{#1} }}
\renewcommand{\mp}[1]{\M{ \m_{#1} }}
\renewcommand{\sp}[1]{\M{ \s_{#1} }}
\newcommand{\none}{\M{\cdot}}
%% Keywords %%%%%%%%%%%%%%%%%%
\newcommand{\new}{\M{\bt{new}}}
\newcommand{\class}{\M{\bt{class}}}
%% Expressions %%%%%%%%%%%%%%%%%%%%
\newcommand{\Get}[2]{\M{#1.#2}}
\newcommand{\Set}[3]{\M{#1.#2:=#3}}
\newcommand{\Call}[3]{\M{#1.#2(#3)}}
\newcommand{\New}[2]{\M{\new\;#1({#2})}}
\newcommand{\Cast}[2]{\M{\langle{#1}\rangle{#2}}}
%% Types %%%%%%%%%%%%%%%%%%%%%%%%%%%
\newcommand{\any}{\M{\star}}
\newcommand{\Type}[1]{\M{\{ #1 \}}}
\newcommand{\HT}[2]{\M{{#1}\!:{#2}}}
%% Classes %%%%%%%%%%%%%%%%%%%%%%%%%
\newcommand{\Mdef}[5]{\M{ \HT { #1( \b{\HT{#2}{#3}})}{#4}~ \{\, {#5} \,\} }}
\newcommand{\SMdef}[5]{\M{ \HT { #1!( \HT{#2}{#3})}{#4}= {#5}}}
\newcommand{\GMdef}[3]{\M{ \HT { #1()}{#2}={#3}}}
\newcommand{\Ftype}[2]{\M{ \HT{#1}{#2} }}
\newcommand{\Fdef}[3]{\M{ \HT{#1}{#2}={#3} }}
\newcommand{\Mtype}[3]{\M{ \HT { #1( #2 )}{#3}}}
\newcommand{\Class}[3]{\M{\bt{class}\;#1\,\{\, #2 ~ #3\, \}}}
%%% Dynamics %%%%%%%%%%%%%%%%%%%%%%%%%
\newcommand{\is}{\M{\mapsto}}
\newcommand{\Obj}[3]{ \M{\{ #1 \}^{#2}_{#3}}}
\newcommand{\Heap}[2]{\M{ #1[ #2 ] }}
\newcommand{\alloc}[4]{\M{#1,#2  = \xt{alloc}(#3, #4)}}
\newcommand{\dispatch}[5]{\M{#1,#2 = \xt{dispatch}(#3,#4,#5)}}
\newcommand{\notdispatch}[5]{\M{#1,#2 \not = \xt{dispatch}(#3,#4,#5)}}
\newcommand{\readfield}[4]{\M{#1 = \xt{readfield}(#2,#3,#4)}}
\newcommand{\setfield}[5]{\M{#1 = \xt{writefield}(#2,#3,#4,#5)}}


%% Formatting %%%%%%%%%%%%%%%%%%%%%%%%%%%
\newcommand{\Alt}[1]{ &\B #1 \\}
\newcommand{\B}{\M{~|~}}
\newcommand{\M}[1]{\ensuremath{#1}\xspace}
\newcommand{\xt}[1]{{\sf{#1}}\xspace}
\newcommand{\bt}[1]{\xt{\bf #1}}
\renewcommand{\b}[1]{\M{\overline{#1}}}
\newcommand{\opdef}[2]{\framebox[1.1\width]{#1} ~ #2\\}

\newcommand{\IRule}[3]{\inferrule*[lab={\tiny #1}]{#2}{#3}}
%\newcommand{\IRule}[3]{\inferrule{#2}{#3}}    %% No label
\newcommand{\CondRule}[3]{ #3 &~{\emph{if}} #2 \\}
\newcommand{\NoCondRule}[2]{ #2 &       \\}
\newcommand{\Reduce}[4]{\M{ #1~#2 \rightarrow #3~#4}}
\newcommand{\ReduceA}[4]{\M{ #1 ~ #2 } &  \M { \rightarrow #3 ~ #4}}
\newcommand{\inc}{\M{\in}}
\newcommand{\Update}[3]{\M{#1[ #2 := #3]}}
\newcommand{\Bind}[2]{\M{#1 \is #2}}
\newcommand{\NotSub}{\M{\not<:}}
\newcommand{\Sub}{\M{<:}}
\newcommand{\classofis}[2]{\M{\xt{classof}(#1)=#2}}
\newcommand{\typeofis}[3]{\M{\xt{typeof}(#1,#2)=#3}}
\newcommand{\classof}[1]{\M{\xt{classof}(#1)}}
\newcommand{\typeof}[2]{\M{\xt{typeof}(#1,#2)}}
\newcommand{\Sel}[2]{\M{#1(#2)}}

\newcommand{\EnvType}[3]{ \M{#1 \vdash #2 : #3}}
\newcommand{\HasType}[3]{ \M{#1 (#2) = #3}}
\newcommand{\E}{\M{\Gamma}}
\newcommand{\Es}{\E ~\s}

\newcommand{\TransClass}[2]{\M{ #1 \hookrightarrow #2 }}
\newcommand{\TransExp}[4]{\M{ #1 \vdash #2 \hookrightarrow #3 : #4 }}
\newcommand{\mdp}[1]{\M{\md_{#1}}}
\newcommand{\setter}[1]{\M{\llbracket #1  \rrbracket}_{\xt{set}}}
\newcommand{\getter}[1]{\M{\llbracket #1  \rrbracket_{\xt{get}}}}
\newcommand{\dynamic}[1]{\M{\llbracket #1  \rrbracket_{\xt{dyn}}}}
\newcommand{\invoke}[1]{\M{\llbracket #1 \rrbracket_{\xt{inv}}}}
\newcommand{\Dyn}[1]{\M{#1^{\any} }}

\newcommand{\Ctx}[1]{\M{\xt{E}[#1]}}

\newcommand{\casts}[1]{\llbracket #1 \rrbracket_{\xt{casts}}}
\newcommand{\fcast}[1]{\M{\llbracket #1 \rrbracket_{\xt{translate}}}}
\newcommand{\wrapper}[1]{\M{\llbracket #1  \rrbracket}_{\xt{wrap}}}
\newcommand{\proxy}[1]{\M{\llbracket #1  \rrbracket}_{\xt{proxy}}}
\newcommand{\utw}[1]{\M{\llbracket #1  \rrbracket}_{\xt{untyped}}}
\newcommand{\tw}[1]{\M{\llbracket #1  \rrbracket}_{\xt{typed}}}

\newtheorem{definition}{Definition}
\newcommand{\WF}[1]{\ensuremath{\xt{WF}(#1)}\xspace}


\newcommand{\Weak}{\text{\small?}\hspace{0.1em}}
\newcommand{\WType}[1]{\Weak\Type{#1}}
\newcommand{\GenCast}[4]{#1 \vdash #2 \hookrightarrow #3 \Rightarrow #4}
\newcommand{\AnaCast}[4]{#1 \vdash #2 \rightsquigarrow #3 \Leftarrow #4}


\newcommand{\stcons}[2]{#1\lesssim #2}
\newcommand{\meet}[2]{#1\sqcap #2}
\newcommand{\refine}[2]{#1\vartriangleright #2}
\newcommand{\inscast}[1]{\M{\llbracket #1 \rrbracket_{\xt{ins}}}}
\newcommand{\adapt}[1]{\M{\llbracket #1 \rrbracket_{\xt{adapt}}}}
\newcommand{\castfn}[3]{\text{cast}(#1,#2,#3)}

\newcommand{\p}{\xt{p}}

\newcommand{\typed}[1]{\text{typed}(#1)}
\newcommand{\untyped}[1]{\text{untyped}(#1)}


%%%%%%%%%%%%%%%%%%%%%%%%%%%%%%%%%%%%%%%%%%%%%%%%%%%%%%%%%%%%%%%%%%%%%%%%%%%
\begin{document}
\title{Designing Gradual Types for Objects
\vspace{-1cm}} 
\authorinfo{}{}{} % Annon : Benjamin Chung, Jan Vitek}{Northeastern University}{}
\maketitle

\begin{abstract}
The popularity of dynamically typed languages has given rise to a cottage
industry of type systems that provide various degrees of assurance
while allowing some code to remain free of type annotations. 
The motivation underlying these new type systems is that they give
programmers a way to incrementally migrate a code base from untyped
to typed.  It is becoming increasingly clear that the design of these
type systems must carefully consider issues such expressiveness of the 
annotation language, granularity of type regions, and performance overheads
of any associated run-time checks.  This paper focuses on gradual types for 
object-oriented languages and attempts to explain some of the key design 
dimensions and trade-offs of three implemented gradual types systems.
This is achieved by embedding the different design into a core calculus
that is broadly representative of dynamic languages such as Python, JavaScript
and Ruby.
\end{abstract} 

\section{Introduction}
Well-typed programs go wrong every day. Short of programming with a proof
assistant, our programs will remain replete with errors that lead to
undesirable outcomes.  The question that face language designers is what
classes of errors are frequent enough that it makes sense to try to catch in
the programming language itself, and how much linguistic machinery to expose
to end-users for that purpose.  At one end of the spectrum, some of the most
popular languages of the day rely solely on dynamic checks to catch errors
as the program runs. Towards the other end, statically typed languages
expose a set of type annotations that programmers have to use in their code
and, in return, these languages ensure that many errors are ruled entirely.

Gradual typing attempts to merge both of these world-views by allowing
developers to add annotations in an incremental manner, with the promise of
rewards in error prevention commensurate to the effort invested in selecting
where to put annotations. Types have other benefits, by making programmer
intent explicit, they play the role of simple machine-checked documentation,
and they also have the potential of providing useful information for
the compiler to generate efficient code.

This paper explores the design space of gradual type systems for
object-oriented languages. We aim to expose the main forces influencing the
design of practical systems, and to provide advice for the next generation
of gradually typed languages. To this end, we have designed a minimialistic
object-based language that supports statically and dynamically typed code
fragments. On top of this common core, we then build models of several
practical gradual type systems and illustrate their differences in terms of
catching errors. While our work has a formal flavor, we are careful to
discuss the practical implications of different choices in the light of
our experience with the implementation of several gradually typed systems.

The fundamental property that a type system object-based language should
guarantee is that an expression of the form 
\vspace{-1mm}\[ \xt{x.m(y)} \]

\vspace{-1mm}\noindent does not result in, to borrow Smalltalk terminology,
a \emph{message not understood} error.  That is to say, method \xt{m} exists
in the object bound to variable \x. Many static type systems only
provide a partial guarantee as they make an exception for \xt{null} or
\xt{undefined} references; the call may fail as the test whether \x is
bound to an object is deferred to execution.

At the heart of the design of a gradual type system lies the choice of
definition of \emph{soundness} and the static and dynamic mechanisms
employed to enforce it. The literature is ripe with proposals with subtly
different properties and mechanisms.  One key distinction is what it means
for a variable \x to be of type \T. In a traditional statically typed
language, \x will refer to either an instance of the class denoted by \T or
one of its sub-classes; we say that \T is a \emph{concrete type}. In some
proposals, a variable \x of type \T can refer to an instance of any class,
in this case we say that \T is \emph{unconstrained type}. Lastly, in some
systems \x is allowed to refer to any value but the language implementation
will ensure that the value behaves as if it was of type \T, we call this a
\emph{promised type}.

While statically typed languages such as Scala only offer concrete types and
dynamically typed languages like Python have a single, implicit,
unconstrained type, gradually typed language have experimented with various
combinations.  C\# has concrete typed but also support a single
unconstrained type \xt{dyn}~\cite{Bierman10}. Thorn~\cite{popl10} and
StrongScript~\cite{ecoop15} both support concrete types as well as
unconstrained types. Languages such as Facebook's Hack and
TypeScript~\cite{typescript13} support only unconstrained types.  Finally
Reticulated Python~\cite{siek14} and Typed Racket~\cite{tf-dls06} support
promised types.



%%%%%%%%%%%%%%%%%%%%%%%%%%%%%%%%%%%%%%%%%%%%%%%%%%%%%%%%%%%%%%%%%%%%%%%%%%%
\section{A Common Core for Gradually Typed Objects}

This section introduces the small object calculus we use to illustrate the
differences between the three gradual type systems.  The calculus was
carefully designed to act as a common core on top of which gradual type
systems can grafted with relative ease. As usual in such exercises, we aimed
for minimality, language features not directly required for our development
were omitted. The result is an imperative class-based system that does not
support inheritance. The calculus adopts a very simple structural type
system with concrete types and a single unconstrained type.

The common core allows writing traditional object-oriented examples such as
a typed \xt{Point} class.\footnote{For our examples we use integers and
  common arithmetic operations even if they are not in core calculus.}

%%% FIXME: formatting
\begin{verbatim}
class Point {
  x : Int
  y : Int
  addx( v : Int ) : Int {
       this.x!( this.x() + v )
  }
  addy( v : Int ) : Int {
       this.y!( this.y() + v )
  }
}
\end{verbatim}

The type system ensures that, if \xt{pt} is declared to be of type
\xt{Point}, an operation such as \xt{pt.addx(42)} will not get stuck.  As
this small language requires all variables to be initialized, there checking
for \xt{null} is not needed as it likely would in a full fledged language.

The language also supports a single unconstrained type, denote \any (pronounced
dyn), thus one could write the following method

%%% FIXME: formatting
\begin{verbatim}
  mkPt( u : * ) : Point {
      new Point( u.x() 
                 (<Point> u).y() )
  }
\end{verbatim}

This method will accept an instance of any class and will return a new
\xt{Point}. At run-time, the code will get stuck if the dynamic cast
\xt{\Cast{\xt{Point}}{\xt{u}}} fails. If \x is a declared of type \any, an
expression such as \xt{u.x()} will fail if the object bound to \xt{u} does
not have a getter \xt{x()}.


For brevity we use annotate variables with nominal types, but in the
calculus this is merely shorthand for writing the set of methods suported 
by the class. Assignment to fields is performed by automatically generated
getter and setter methods.

The syntax of the common core appears in figure~\ref{syn}.  \x ranges
over variables, \f ranges over field names, \m can be either a plain method
name \d, a field getter \f, or a field setter \fb, \C and \D range over
class names. We use the overbar notation to denote a possibly empty
sequence. In the common core, a class is defined as by a set of fields,
\b\f, with distinct names and a set of method definitions, \b\m. An instance
of class is constructed by the expression \New\C{\b\e} where each argument
is used to initialize the corresponding field. Fields are encapsulated and
can only be accessed and updated in the corresponding getter and setter from
the \this variable.  Expressions include variable access, field access and
update, method invocation, object creation, and type casts.



%%%%%%%%%%%%%%%%%%%%%%%% SYNTAX %%%%%%%%%%%%%%%%%%%%%%%%%%%%%%%%%%%%%%%%%%%%%%%%
\begin{figure}[!h]\center\begin{minipage}{4cm}\begin{tabular}{l@{~~~}l}
\e &::=  \x \\
   \Alt{ \Get\e\f }
   \Alt{ \Set\e\f\e }
   \Alt{ \Call\e\m{\b\e} }
   \Alt{ \New\C{\b\e} }
   \Alt{ \Cast\t\e }
\end{tabular}\end{minipage}\begin{minipage}{4cm}\begin{tabular}{l@{~~~}l}
\fd &::= 
    \Ftype\f\t   \\
\md &::=
    \Mdef\m\x\t\t\e \\
\c &::= \Class \C {\b{\fd}}{\b{\md} } \\
\mt &::= \Mtype\m{\b\t}\t\\
\t &::= ~ \any \\
   \Alt{ \Type{  \b{ \mt } } }
\end{tabular}\end{minipage}
\caption{Abstract Syntax for the Core Calculus}\label{syn}
\end{figure}
%%%%%%%%%%%%%%%%%%% END OF SYNTAX %%%%%%%%%%%%%%%%%%%%%%%%%%%%%%%%%%%%%%%%%%%%%%

Our types are structural and only account for the methods in an object's
interface. For type annotations, we use the name of classes, ranged over by
\C and \D, as shorthand for the set of methods defined in the respective
classes.


The dynamic semantics evaluates expression extended with object references,
denoted \a, and errors, denoted \err, together with a heap \s mapping
references to object values. Object values contain all fields, methods, a
type, and a class; they are denoted \Obj{\b\a}\t\C. The
class is used for locating methods and the type is used for type casts. The
need for keeping them separate will become clear later.

The semantics uses evaluation context \Ctx\e.

Selecting an object from the heap is written \Sel\s\a, while a heap is
extended with a new object by \Heap{\s}{\Bind{\a}{\Obj{\dots}\t\C}}.


For an object reference \a, such that \M{\Sel\s\a=\Obj{\dots}\t\C}, we have 
\classofis{\Sel\s\a}\C and \typeofis\s\a\t.

We use the notation \Mdef\m\x\t\t \inc\C to select a method in a class
definition and \Fdef\f\t\a\inc\Obj{\dots}\t\C to express the selection of a
field. Lastly a field of an object can be update with the notation
\Update{\Obj{\dots}\t\C}\f\a.


\subsection{Dynamic Semantics}

We use a typical small-step operational semantics with a heap. Each step steps an expression and a heap to a result and a new heap, where a result can either be a new expression or an error. Errors halt execution, and can be lifted through execution frames.

\subsection{Typing}

As an object-oriented calculus, our type system has subtyping. We use a structural subtyping mechanism which supports with, depth, and reordering. 

Each class in our system must be well-formed, as described in Figure 4. Each method must typecheck against the named arguments, producing the result type. Our type system has two major issues: it does not provide any guarantees about the type of the arguments to an untyped method, and it does not mention getters and setters at all.

To fill these holes, we use class translation, which performs 2 major tasks:
\begin{itemize}
\item Generating accessor methods, to enable user code to retreive and modify field values.
\item Generating wrapper methods, that take all $\*$ arguments and use casts to ensure that the values that they were passed are correct. 
\end{itemize}

% TODOS:
% $ should not be a valid character in a method name <- explain in text
% c translates to c with $ <- explain in text
% Figure 2 sigma <- not sure?


A program consists of an expression and an implicit class table which holds
definitions for all the classes that can be used in the program.

We will require that the class table be, first, \emph{well formed}, and,
then, \emph{completed}.  A well formed class table is made up of classes
that are well-typed and where no special methods have been used.  A well
formed table is completed by adding setter and getter methods, synthesizing
dynamic methods and specializing calls with dynamic receivers.

\begin{definition} A class table is well-formed if for every class \C  in
the class table, every method \Mdef\d\x{\tp1}{\tp2}\e in \C is such that
\EnvType{\this:\C,\b{\x:\tp1}}\e{\tp2} holds.
\end{definition}

Note that a well-formed class table does not have setter, getter, or dynamic
methods. Furthermore there are no occurrence of dynamic calls.

The completion process extends each class definition with getter, setters,
dynamic methods, and specializes dynamic calls.

\begin{definition} A class table is completed if for every 
 \Class \C {\b{\fd}} {\b{\md}} in the class table: for every field
 \Ftype\f\t\inc\C, we add the setter \SMdef\f\x\t\t{\Set\this\f\x} and the
 getter \GMdef\f\t{\Get\this\f} to \C; for ever method definition
 \Mdef\d\x\t{\tp1}\e\inc\C, a dynamic method \Mdef{\Dyn\d}\x\any\any{
   \Cast\any{\Call\this\d{\b{\Cast\t\x }} }} is added to \C and the body \e of
 \m is replaced by  \ep1 where \TransClass\e{\ep1}.
\end{definition}

I am aware that the last bit of the definitin is wrong. Will fix.


%%%%%%%%%%%%%%%%%%%%%% DYNAMIC SEMANTICS %%%%%%%%%%%%%%%%%%%%%%%%%%%%%%%%%%%%%%
\begin{figure}

\begin{minipage}{8cm}
\opdef{\Reduce{\ep 1}{\sp 1}{\ep 2}{\sp 2}}{\ep1 \sp1 evaluates to \ep2 \sp2 or \err in a step}\\[-1mm]
\begin{tabular}{@{}l@{}l@{~}l@{~}l}
\CondRule{E1}{ %% e -> e'
  \Reduce {\ep 1}{\sp 1}{\ep 2}{\sp 2}
}{
  \ReduceA {\Ctx{\ep1}}{\sp 1}{\Ctx{\ep2}}{\sp 2}
}
\CondRule{E2}{ %% new C -> a
   \alloc{\sp2}{\ap1}{\sp1}{\Obj{\b\a}\C\C}
}{ 
    \ReduceA{ \New\C{\b\a} }{\sp1}{\ap1}{\sp 2}
}
\CondRule{E3}{ %% a.m -> e
   \dispatch{\b\x}\e\s\a\m
}{
   \ReduceA{\Call\a\m{\b{\ap 1}}}\s{[\a/\this~\b{{\ap 1}/\x}]\e}\s
}

\CondRule{E3}{
   \notdispatch{\b\x}\e\s\a\m
}{
   \ReduceA{\Call\a\m{\b{\ap 1}}}\s{\err}{}
}
\CondRule{E3}{ 
     \readfield{\ap1}\s\a\f
}{
  \ReduceA{\Get\a\f}{\s}{\ap 1}{\s}
}
\CondRule{E4}{
     \setfield{\sp2}{\sp1}\a\f{\ap1}
}{
     \ReduceA{\Set\a\f{\ap 1}}{\sp 1}{\ap 1}{\sp 2}
}
\NoCondRule{E5}
{ 
   \ReduceA{ \Cast\any\a}\s \a\s
}
\CondRule{E6}{
  \typeof\s\a \Sub \tp 1
}{ 
    \ReduceA{\Cast{\tp 1}\a}\s\a\s
}
\CondRule{E7}{
  \typeof\s\a \NotSub \tp 1
}{ 
    \ReduceA{\Cast{\tp 1}\a}\s\err{}
}
\CondRule{E8}{
    \Reduce\e{\sp 1}\err{}
}{
    \ReduceA{\Ctx\e}{\sp1}\err{}
}
\end{tabular}\end{minipage}
%%%%%%%%%%%%%%%%%%% CONTEXTS %%%%%%%%%%%%%%%%%%%%%%%%%%%%%%%%%%%%%%%%%%%%
\\[3mm]
\begin{minipage}{4cm}\begin{tabular}{l@{~~}l@{~}l}
\s &::= ~~\none & \B ~~
  \Heap\s{\Bind\a{\Obj{\b\a}{\t}{\C}}} \\[2mm]
\xt{E} &::=    \Get\square\f &\B~
       \Set\square\f\e   ~\B~
       \Set\a\f\square   ~\B~  
       \Call\square\m\e  ~\B~
      \Call\a\m{\b\a\,\square\,\b\e} \\
 &\B~     \Cast\t\square  &\B~
      \New\C{\b \a\,\square\,\b\e}
\end{tabular}
\end{minipage}
\caption{Common Core Calculus Dynamic Semantics.}
\end{figure}

%%%%%%%%%%%%%%%%%%%%%%%% SUBTYPING %%%%%%%%%%%%%%%%%%%%%%%%%%%%%%%%%%%%%%%%
\newcommand{\ESub}[3]{#1 \vdash #2 \leq #3}

\begin{figure}
\opdef{$\tp1 \Sub \tp2$}{\tp 1 is a subtype of \tp 2}
\opdef{$\ESub{\Xi}{\tp1}{\tp2}$}{\tp 1 is a subtype of \tp 2 against environment $\Xi$}
\begin{mathpar}
\IRule{L1}{\ESub\emptyset{\tp1}{\tp2}}{\tp1 \Sub \tp2}

\IRule{S1}{}{\ESub\Xi\t\t}

\IRule{S2}{}{\ESub\Xi\t{\Type{}}}

%\IRule{S3}{\text{Get type for C and D, recurse}}{\ESub\Xi\C{\xt{D}}}

\IRule{S3}{
  \tp1 \leq \tp2 \in \Xi
}{
 \ESub{\Xi}{\tp1}{\tp2}  
}

\IRule{S4}{
    \Xi' = \Xi,~{\t \leq \Type{\Mtype\m{\b{\tp1}}{\tp2} ~ \b{\mt} \,}} \\
    \ESub{\Xi'}\t{\Type{\b{ \mt } }}\\
     \Mtype\m{\b{\tp3}}{\tp4} \inc \t \\
    \b{\ESub{\Xi'}{\tp3}{\tp1}} \\
    \ESub{\Xi'}{\tp2}{\tp4} \\    
}{
   \ESub\Xi\t{\Type{\Mtype\m{\b{\tp1}}{\tp2} ~ \b{\mt} \,} } 
}
\end{mathpar}
\caption{Subtyping}
\end{figure}


%%%%%%%%%%%%%%%%%%%%%%%%% WELLFORMDNESS %%%%%%%%%%%%%%%%%%%%%%%%%%%%%%%%%%


\begin{figure}
\begin{mathpar}
\IRule{A1}{\HasType{\E}\x\t}{\GenCast\E\x\x\t}

\IRule{A4}{
	\GenCast{\E}{\e_1}{\e_2}{\t_1} \\
	\Mtype\m{\b{\tp 2}}{\tp 3} \inc \t_1\\
	\b{\AnaCast{\E}{\e_3}{\e_4}{\tp 2}} \\
}{
	\GenCast{\E}{\Call{\e_1}\m{\b{\e_2}}}{\Call{\e_3}{\m}{\b{\e_4}}}{\tp 3}
}

\IRule{A5}{
	\GenCast{\E}{\e_1}{\e_3}{\any} \\
	\b{\AnaCast{\E}{\e_2}{\e_4}{\any}} \\
}{
	\GenCast{\E}{\Call{\e_1}\m{\b{\e_2}}}{\Call{\Cast{\Type{\Mtype{\m^*}{\b{\any}}\any}}{\e_3}}{\Dyn\m}{\b{\e_4}}}{\any}
}

\IRule{A6}{
  \b{\AnaCast\E{\e_1}{\e_2}\t} \\ 
  \Class \C {\b{\Ftype\f\t}} {\b{\md}}
  }{\GenCast{\E}{\New\C{\b{\e_1}}}{\New\C{\b{\e_2}}}{\C}}
\end{mathpar}
\caption{Synthetic cast insertion}
\end{figure}

%%%%%%%%%%%%%%%%%%%%%%%%% CLASS TRANSLATION %%%%%%%%%%%%%%%%%%%%%%%%%%%%%
\begin{figure}
\opdef{\TransExp\E{\ep1}{\ep 2}\t}{\ep1 translates to \ep2 in environment \E with type $\t$}
\begin{mathpar}
\IRule{AASC2}{
  \GenCast{\E}{\ep1}{\ep2}{\tp2} \\
  \tp2 <: \tp1\\
}{
  \AnaCast{\E}{\ep1}{\ep2}{\tp1}
}

\IRule{AASC3}{
  \GenCast{\E}{\ep1}{\ep2}{\any} \\
}{
  \AnaCast{\E}{\ep1}{\ep2}{\any}
}
\end{mathpar}
\caption{Analytic Cast Insertion}
\end{figure}

\begin{figure}
\opdef{\TransExp\E{\ep1}{\ep 2}\t}{\ep1 translates to \ep2 in environment \E with type $\t$}
\begin{mathpar}
\IRule{CT1}{
 \b{ \mdp1} \equiv \b{\fcast{\md}} ~~ \b{\setter\fd} ~~ \b{\getter\fd}\\  \b{\mdp2} \equiv \b{\mdp1} ~~ \b{\proxy{\mdp1}}
}{ 
  \TransClass { \Class \C {\b{\fd}} {\b{\md}} }{  \Class \C {\b{\fd}} {\b{\mdp2}} }
}

\IRule{}{
  \AnaCast{\b{\x:\tp1}}{\ep1}{\ep2}{\tp2}
}{
  \fcast{\Mdef\m\x{\tp1}{\tp2}{\ep1}} \equiv \Mdef\m\x{\tp1}{\tp2}{\ep2}
}

\IRule{}{
}{
  \getter{\f:\t} \equiv \m() : \t ~ \{ \this.\f \}
}

\IRule{}{
}{
  \setter{\f:\t} \equiv \m(\x:\t) : \t ~ \{ \this.\f=\x \}
}

\IRule{}{
}{
  \proxy{\Mdef\m\x{\tp1}{\tp2}{\ep1}} \equiv \Mdef{\m^*}\x{\any}{\any}{\Cast{\any}{\Call\this\m{\b{\Cast{\tp1}\x}}}}
}
\end{mathpar}
\caption{Class Translation}
\end{figure}
%%%%%%%%%%%%%%%%%%%%%%%%%%%%%%%%%%%%%%%%%%%%%%%%%%%%%%%%%%%%%%%%%%%%%%%%%


\subsection{Type soundness:}
Theorem 1: Type translation. If $\EnvType{\Es}\e\t$ and $\TransExp\E\e{\e'}\t$ then $\EnvType{\E ~ \cdot}{\e'}\t$.

Theorem 2: Progress. If $\EnvType{\Es}{\e}{\t}$ then $\s,\e \rightarrow \s',\e'$ for $\e'$ expression or $\e' ~ \err$.

Theorem 3: Preservation. If $\s,\e \rightarrow \s',\e'$ and $\EnvType{\Es}{\e}{\t}$ then $\EnvType{\E~\s'}{\e'}{\t}$.

For a list of classes $\bar{\c}$ such that $\b{\c ~~ WF}$ and an expression $e$ such that $\EnvType\cdot\e\t$ and $\b{\TransClass\c{\c'}}$, we have $\cdot~e \rightarrow^* \sigma~r$ (against $\b{\c'}$) where $r$ is either a value or $\err$.


\begin{figure}
\begin{mathpar}
\IRule{TRNew1}{
  \b{\AnaCast\E{\e_1}{\e_2}\t} \\ 
  \Class \C {\b{\Ftype\f\t}} {\b{\md}} \\
  \typed{\C}
  }{\GenCast{\E}{\New\C{\b{\e_1}}}{\New\C{\b{\e_2}}}{\C}}

\IRule{TRNew2}{
  \b{\AnaCast\E{\e_1}{\e_2}\any} \\ 
  \Class \C {\b{\Ftype\f\any}} {\b{\md}} \\
  \untyped{\C}
  }{\GenCast{\E}{\New\C{\b{\ep1}}}{\Cast{\any}{\New\C{\b{\ep2}}}}{\any}}

\IRule{AASC1}{
  \GenCast{\E}{\ep1}{\ep2}{\t} \\
  \t \neq \any \\
  \C = \tw{\t}
}{
  \AnaCast{\E}{\ep1}{\Cast{\any}{\New\C{\ep2}}}{\any}
}

\IRule{AASC2}{
  \GenCast{\E}{\ep1}{\ep2}{\any} \\
  \t \neq \any \\
  \C = \utw{\t}
}{
  \AnaCast{\E}{\ep1}{{\New\C{\Cast{\t}{\ep2}}}}{\t}
}

\IRule{MTWrap}{
}{
  \tw{\Mtype\m{\b{\tp1}}\t} = \Mdef\m\x{\tp1}{\t}{\Cast{\t}{(\Cast{\Type{\Mtype{\m^*}{\b\any}{\any}}}{\this.\xt{s}}).\m^*(\b{\Cast{\any}{\x}})}}
}

\IRule{MUTWrap}{
}{
  \utw{\Mtype{\m}{\b{\tp1}}\t} = \Mdef{\m^\xt{*}}\x{\any}{\any}{\Cast{\any}{\this.\xt{s}.\m(\b{\Cast{\tp1}{\x}})}}
}

\IRule{TWrap}{
  \Class\C{\xt{s} : \any}{\b{\tw{\mt}}}
}{
  \tw{\Type{\b{\mt}}} = \C
}

\IRule{TUWrap}{
  \Class\C{\xt{s} : \Type{\b{\mt}}}{\b{\utw{\mt}}}
}{
  \utw{\Type{\b{\mt}}} = \C
}
\end{mathpar}
\caption{Translation for Typed Racket}
\end{figure}

Typed Racket uses a much stricter definition of where $\any$ types can go, where every class is either fully typed or fully untyped. To describe this, we alter definition 1 and 2 to


\begin{definition} A Typed Racket class table is well-formed if for every class \C  in
the class table, every method is of the form \Mdef\d\x{\tp1}{\tp2}\e where
\EnvType{\this:\C,\b{\x:\tp1}}\e{\tp2} holds in \C, and all types in \C are either $\any$ or all not $\any$
\end{definition}

Typed Racket uses wrappers to ensure that typed code type guarantees are not violated, and that untyped code follows the types that it is casted to. We generate these wrappers using the following mechanism.

\begin{definition}
Every place a type \Type{\mt} where $\mt=\Mtype\m{\b{\tp1}}{\tp2}$ flows from typed into untyped code, generate a wrapper $\Class{\C'}{~\xt{orig}:\t,\b{\md}}$ where $\md = \Mdef\m\x{\any}{\any}{\Cast{\any}{\xt{this}.\xt{orig}.\m(\b{\Cast{\tp1}x})}}$.
\end{definition}

To model Typed Racket, we then need to enforce the property that objects are wrapped at typed/untyped boundaries, and ensuring that untyped code cannot use typed code internally through casts. We do this by altering the completion process, giving us a new definition. We use the wrappers we generated using definition 4 to enforce the type guarantees
\newpage
\begin{definition} A Typed Racket class table is completed if for every 
 \Class \C {\b{\fd}} {\b{\md}} in the class table: 
 \begin{itemize}
 \item for every type \Type{\b{\md}}, add a class \xt{D} such that 
 \begin{itemize}
 \item for every method $\Mtype\m{\tp1}{\tp2} \in \b{\md}$, generate a function in \xt{D} $\Mdef\m{x}{\any}{\any}{\Cast{\any}{\New{\xt{D}}{\xt{this}.\xt{orig}.\m(\New\C{\b{\Cast{\tp1}x}})}}}$
 \end{itemize}
 \item for every field \Ftype\f\t\inc\C, we add to \C:
 \begin{itemize}
 \item A setter \SMdef\f\x\t\t{\Set\this\f\x} 
 \item A getter \GMdef\f\t{\Get\this\f}; 
 \end{itemize}
 \item For every method definition \Mdef\d\x\t{\tp1}\e\inc\C:
 \begin{itemize}
 \item A dynamic method \Mdef{\Dyn\d}\x\any\any{\Cast\any{\Call\this\d{\b{\Cast\t\x }} }} is added to \C 
 \item We replace the body \e of \m is replaced by \ep1 where \TransClass\e{\ep1}.
 \end{itemize}
 \end{itemize}
\end{definition}

One of the issues inherent to the Strongscript approach (and is apparent in
the common core as well) is that a strict interpretation of the static type
system causes the programmer to have to write a very large number of
``obvious'' casts, breaking untyped code and seemingly-sensible typed
code. We can solve this by having the compilation process insert the types
for the programmer, which we describe using a cast insertion system.

Our cast insertion approach is based on a bidirectional type
system~\cite{}, where each expression either \emph{synthesizes}, or
inherently makes, a type, or is \emph{analyzed} against a type where the
type system checks to make sure that the expression still ``works'' under
the given type. This approach has been used for a number of other tasks,
including inferring types to select sub-languages~\cite{}, infer
types for higher rank languages~\cite{}. In our case, they allow us
to simply specify in an extensible manner where to insert casts.

Synthetic cast insertion is closer to a traditional type system, as it produces types from terms in a similar manner. However, instead of having non algorithmic cases where types are known (for example, in the arguments to a typed function), the synthetic cast insertion judgment defers to the analytic cast insertion mechanism with the known type. Importantly for us, the basic semantics of the static type system does not change between any of the type systems under consideration, and as a result the synthetic cases are not changed by any of the systems.

Analytic cast insertion ensures that an expression is of a given type. We use analytic cast insertion when a type is known for an expression and we want to force that expression to be of the correct type, which we do by inserting the appropriate cast. The actual details of analytic cast insertion vary depending on the system under consideration (most notably, the monotonic semantics has a notion of consistency that differs from the other two systems).

\begin{figure}

$\t = \ldots \B {\WType{\Mtype\m\t\t}}$

\begin{mathpar}
\IRule{STS1}{
}{
	\tp1 \Sub \Weak\tp1
}

\IRule{STS2}{
	\tp1 \Sub \tp2
}{
	\Weak\tp1 \Sub \Weak\tp2
}

\IRule{A5}{
	\GenCast{\E}{\e_1}{\e_3}{\Weak\tp1} \\
	\Mtype{\m}{\tp2}{\tp3} \in \tp1 \\
	\b{\AnaCast{\E}{\e_2}{\e_4}{\Weak\tp2}} \\
}{
	\GenCast{\E}{\Call{\e_1}\m{\b{\e_2}}}{\Call{\Cast{\Type{\Mtype{\m^*}{\b{\any}}\any}}{\e_3}}{\Dyn\m}{\b{\Cast\any{\e_4}}}}{\Weak\tp3}
}

\IRule{AASC1}{
  \GenCast{\E}{\ep1}{\ep2}{\Weak\tp2}
}{
  \AnaCast{\E}{\ep1}{\Cast{\tp1}{\ep2}}{\tp1}
}

\IRule{AASC2}{
  \GenCast{\E}{\ep1}{\ep2}{\tp2}
}{
  \AnaCast{\E}{\ep1}{\ep2}{\Weak\tp1}
}
\end{mathpar}
\caption{Strongscript}
\end{figure}

\begin{figure}
\begin{mathpar}
\IRule{A5}{
	\GenCast{\E}{\e_1}{\e_3}{\tp1} \\
	\Mtype{\m}{\tp2}{\tp3} \in \tp1 \\
	\b{\AnaCast{\E}{\e_2}{\e_4}{\tp2}} \\
}{
	\GenCast{\E}{\Call{\e_1}\m{\b{\e_2}}}{\Cast{\tp3}{(\Call{\e_3}{\m^*}{\b{\Cast{\any}{\e_2}}})}}{\tp3}
}
\end{mathpar}
\caption{Transient}
\end{figure}

\clearpage
\newpage

\section{Monotonic}

%%%%%%%%%%%%%%%% Monotonic Statics

\begin{figure}[h]
\begin{mathpar}
\IRule{CS1}{ }{\stcons{\t}{\t}}

\IRule{CS2}{ }{\stcons{\any}{\t}}

\IRule{CS3}{ }{\stcons{\t}{\any}}

\IRule{CS4}{ \tp 1 \Sub \tp 2 }{\stcons{\tp 1}{\tp 2}}

\IRule{CS5}{
	\stcons{\tp3}{\tp1} \\
    \stcons{\tp2}{\tp4} \\
    \stcons{\Type{\b{\md_1} ~ \b{\md_2}}}{\Type{\b{\md_3} ~ \b{\md_4}}}
}{\stcons{\Type{\b{\md_1} ~ \Mtype\m{\tp1}{\tp2}~ \b{\md_2}}}{\Type{\b{\md_3} ~ \Mtype\m{\tp3}{\tp4} ~ \b{\md_4}}}}
\end{mathpar}
\caption{Consistent Subtyping}
\end{figure}

The key static difference between the monotonic semantics and the other systems we have considered is the consistent subtyping relationship. Traditional subtyping can be thought of as if a value satisfies one type, then it will satisfy any supertype of that type. Consistent subtyping encapsulates both this notion and the new idea of consistency. Consistency for objects means that two types have static type information that does not conflict - e.g. $\Type{\Ftype\f\any}$ could very well be able to replace $\Type{\Ftype\f\C}$, though we are unable to tell statically. Consistency allows us to interchange these types, inserting a cast where required.

Subtype consistency combines the two properties. Subtyping lets us add and remove fields from a type, and consistency allows us to make the type we are going to more or less dynamic. Going by the above intuition, a value of type $\any$ could work with any other type and vice versa, and clearly two identical types are compatible. The most complex case is when we have two class types, both with the same method $\m$, which we resolve by ensuring that $\m$ has parameters and return types that are consistent and continuing on through the rest of the class.

Adding this to our static type system is simple. We just introduce the rule that allows types to be converted via $\stcons{}{}$. The details of how to make the dynamics work with this static addition will be covered later.

\begin{figure}{h}
\opdef{$\meet{\tp1}{\tp2} \equiv \tp3$}{The most specific type common to $\tp1$ and $\tp2$ is $\tp3$}
\begin{mathpar}
\IRule{M1}{ }{\meet{\t}{\any} \equiv \t}

\IRule{M2}{ }{\meet{\any}{\t} \equiv \t}

\IRule{M3}{ }{\meet{\t}{\t} \equiv \t}

\IRule{M4}{
	\b{\meet{\tp3}{\tp1}  \equiv \tp5} \\
    \meet{\tp2}{\tp4} \equiv \tp6\\
    \meet{\Type{\b{\md_1}}}{\Type{\b{\md_2}}} \equiv \Type{\b{\md_3}} 
}{\meet{\Type{\Mtype\m{\b{\tp1}}{\tp2}~ \b{\md_1}}}{\Type{\Mtype\m{\b{\tp3}}{\tp4} ~ \b{\md_2}}} \equiv \Type{\Mtype\m{\b{\tp5}}{\tp6} ~ \b{\md_3}}}
\end{mathpar}
\caption{Meet for the monotonic system}
\end{figure}

This idea leads us into the core of the monotonic idea. Under this system, if we have a field of some given type $\t$, we do not know if the value of that field is truly of the expected type (since the class could have been cast to one where that field is $\any$ and the field updated), and therefore have to check the value of $\f$ at every point it is used. The solution used by the monotonic approach is to ensure that if a type is given to a field, then that type is never violated.

Consistency allows us to convert between, for example, $\Type{\Ftype\f\any, \Ftype{\xt{g}}\C}$ and $\Type{\Ftype\f\C, \Ftype{\xt{g}}\any}$, as neither has static typing information that would rule out a value that satisfied the other type. However, if we have a value that satisfied both, we would end up with a type that had neither type exactly - we would have $\Type{\Ftype\f\C, \Ftype{\xt{g}}\C}$. This is the key to the monotonic system - we never throw away any type information away. If, at any point in the program, we have an object with some specialized (e.g. not all $\any$) type, then those type assumptions will never be violated. As a result, we can ensure that no get operation will need to be checked.

\begin{figure}
\begin{mathpar}
\IRule{AA2}{\GenCast{\E}{\ep1}{\ep2}{\tp 2} \\ \tp2 \not\Sub \tp1 \\ \stcons{\tp2}{\tp1}}{\AnaCast\E{\ep1}{{\Cast{\tp1}{\ep2}}}{\tp1}}
\end{mathpar}
\caption{Analytic cast insertion}
\end{figure}


\begin{figure}

Index every type $\t$ in the source of the program with an index $i$ (denoted $\t^i$). Then, for every class $\C$ and every $i$ where $\stcons\c{\t^i}$, we can generate a most specific type as $\t' \equiv \meet\C{\t^i}$. We then produce a wrapper class $\C^i$ such that
\begin{mathpar}
\IRule{}{
\Mtype\m{\b{\tp3}}{\tp4} \in \t
}{
\adapt{\Mdef\m\x{\tp1}{\tp2}\e, \t} \equiv \\ \Mdef \m\x\any{\meet{\tp2}{\tp4}}{\xt{this}.\m'(\b{\Cast{\meet{\tp1}{\tp3}}{x}})},\\
\Mdef {\m'}\x{\meet{\tp1}{\tp3}}{\meet{\tp2}{\tp4}}{\Cast{\meet{\tp2}{\tp4}}\e},	
}

\IRule{}{
	\b{\adapt{\md_1, \t^i}} \equiv \md_2
}{\TransClass{\Class\C{\b{\fd}}{\b{\md_1}}}{\Class{\C^i}{\b{\fd}}{\b{\md_2}}}}
\end{mathpar}
\caption{Monotonic translation}
\end{figure}


\begin{figure}
\opdef{\Reduce{\ep 1}{\sp 1}{\ep 2}{\sp 2}}{\ep 1 with heap \sp 1 evaluates to \ep 2 and \sp 2 in a step}

\begin{mathpar}
\IRule{}{
	\Heap{\sp1}{\Bind{\ap 1}{\Obj{\b\a}{\tp2}\C\}}} \\
	\tp2 \not\equiv \meet{\tp1}{\tp2} \\
	\sp2 = \Heap{\s}{\Bind{\ap 1}{\Obj{\b\a}{\meet{\tp1}{\tp2}}{\refine{\C}{\tp1}}}} \\
	\f:\t'_i \in \meet{\tp1}{\tp2}\\
	\s'_1 = \castfn\a{\t'_1}{\sp2} \\ \ldots \\ \s'_n = \castfn\a{\t'_n}{\s'_{n-1}}\\
}{
	\castfn{\ap1}{\tp1}{\sp1} \equiv \s'_n\\
}
% \Reduce{\Cast{\tp 1}{\ap1}}{\sp1}{\ap1}{\sp2}

\IRule{}{
	\Heap{\sp1}{\Bind{\ap 1}{\Obj{\b\a}{\tp2}\C\}}} \\
	\tp2 \equiv \meet{\tp1}{\tp2} \\
}{
	\castfn{\ap1}{\tp1}{\sp1} \equiv \s'_n\\
}

\IRule{}{
	\sp2 = \castfn{\ap1}{\tp1}{\sp1}
}{
	\Reduce{\Cast{\tp 1}{\ap1}}{\sp1}{\ap1}{\sp2}
}
\end{mathpar}
\caption{Monotonic Dynamic Semantics.}
\end{figure}


\bibliographystyle{plain}
\bibliography{../bib/compactdoi}


\begin{figure}
\opdef{\EnvType{\Es}\e\t}{\e has type \t in environment \E against heap \s}
\begin{mathpar}
\IRule{W1}{
    \HasType{\E}\x\t
 }{
   \EnvType{\Es}\x\t
}

\IRule{W2}{
   \EnvType{\Es}\e{\tp 1} \\ \tp 1 \Sub \t
 }{
   \EnvType{\Es}\e\t 
}   

\IRule{W4}{
   \EnvType{\Es}\e{\C} \\ \Mtype\m{\b{\tp2}}\t\inc \C  \\  \b{\EnvType{\Es}{\ep1}{\tp 2}}
}{
    \EnvType{\Es}{\Call\e\m{\b{\ep1}}}\t
}    

\IRule{W5}{
  \b{\EnvType{\Es}\e\t} \\ 
  \Class \C {\b{\Ftype\f\t}} {\b{\md}}
}{
  \EnvType{\Es}{\New\C{\b\e}}\C
}

\IRule{W6}{
  \EnvType{\Es}\e{\tp1}
}{
   \EnvType{\Es}{\Cast\t\e}\t
}

\IRule{W7}{
  \EnvType{\Es}{\ep1}{\tp1} \\
  \f : \t \in \tp1 \\
}{
  \EnvType{\Es}{\Get\e\f}{\t}
}

\IRule{W8}{
  \EnvType{\Es}{\ep1}{\tp1} \\
  \f : \t \in \tp1 \\
  \EnvType{\Es}{\ep2}{\t}
}{
  \EnvType{\Es}{\Set{\ep1}\f{\ep2}}\t
}

\IRule{W9}{
  \s(\ap1) = \Obj{\b{\f=\ap2}}{\tp1}{\C} \\ \b{\f:{\tp2} \in \C} \\ \b{\EnvType{\Es}{\ap2}{\tp2}} \\ 
  \tp1 <: \C \\
}{
  \EnvType{\Es}{\ap1}\C
}
\end{mathpar}
\caption{Typing rules for the common core}
\end{figure}

\end{document}

