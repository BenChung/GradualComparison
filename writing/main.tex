\documentclass[preprint]{sigplanconf}
\usepackage{xspace,listings,url,subfigure,framed,amssymb,mathpartir,
            stmaryrd % double brackets llbracket
}
\newcommand{\rn}[1]{#1}

\conferenceinfo{CONF 'yy}{Month d--d, 20yy, City, ST, Country}
\copyrightyear{20yy}
\copyrightdata{978-1-nnnn-nnnn-n/yy/mm}
\copyrightdoi{nnnnnnn.nnnnnnn}

%%% Metavariables %%%%%%%%%%%%%%%%%%%
\newcommand{\fd}{\M{\xt{fd}}}
\newcommand{\md}{\M{\xt{md}}}
\newcommand{\mt}{\M{\xt{mt}}}
\newcommand{\m}{\M{\xt{m}}}
\newcommand{\e}{\M{\xt{e}}}
\newcommand{\n}{\M{\xt{n}}}
\renewcommand{\d}{\M{\xt{d}}}
\renewcommand{\r}{\M{\xt{r}}}
\newcommand{\f}{\M{\xt{f}}}
\newcommand{\fb}{\M{\xt{f!}}}
\newcommand{\x}{\M{\xt{x}}}
\renewcommand{\t}{\M{\xt{t}}}
\renewcommand{\c}{\M{\xt{c}}}
\newcommand{\C}{\M{\xt{C}}}
\newcommand{\D}{\M{\xt{D}}}
\newcommand{\this}{\M{\xt{this}}}
\newcommand{\err}{\M{\bt{err}}}
\renewcommand{\d}{\M{\xt{d}}}
\newcommand{\s}{\M{\sigma}}
\newcommand{\fv}{\M{\xt{fv}}}
\renewcommand{\a}{\M{\xt a}}
\newcommand{\F}{\M{\xt F}}
\newcommand{\tp}[1]{\M{ \t_{#1} }}
\newcommand{\ep}[1]{\M{ \e_{#1} }}
\newcommand{\ap}[1]{\M{ \a_{#1} }}
\renewcommand{\mp}[1]{\M{ \m_{#1} }}
\renewcommand{\sp}[1]{\M{ \s_{#1} }}
\newcommand{\none}{\M{\cdot}}
%% Keywords %%%%%%%%%%%%%%%%%%
\newcommand{\new}{\M{\bt{new}}}
\newcommand{\class}{\M{\bt{class}}}
%% Expressions %%%%%%%%%%%%%%%%%%%%
\newcommand{\Get}[2]{\M{#1.#2}}
\newcommand{\Set}[3]{\M{#1.#2:=#3}}
\newcommand{\Call}[3]{\M{#1.#2(#3)}}
\newcommand{\New}[2]{\M{\new\;#1({#2})}}
\newcommand{\Cast}[2]{\M{\langle{#1}\rangle{#2}}}
%% Types %%%%%%%%%%%%%%%%%%%%%%%%%%%
\newcommand{\any}{\M{\star}}
\newcommand{\Type}[1]{\M{\{ #1 \}}}
\newcommand{\HT}[2]{\M{{#1}\!:{#2}}}
%% Classes %%%%%%%%%%%%%%%%%%%%%%%%%
\newcommand{\Mdef}[5]{\M{ \HT { #1( \b{\HT{#2}{#3}})}{#4}={#5}}}
\newcommand{\SMdef}[5]{\M{ \HT { #1!( \HT{#2}{#3})}{#4}= {#5}}}
\newcommand{\GMdef}[3]{\M{ \HT { #1()}{#2}={#3}}}
\newcommand{\Ftype}[2]{\M{ \HT{#1}{#2} }}
\newcommand{\Fdef}[3]{\M{ \HT{#1}{#2}={#3} }}
\newcommand{\Mtype}[3]{\M{ \HT { #1( #2 )}{#3}}}
\newcommand{\Class}[3]{\M{\bt{class}\;#1\{ #2 ~ #3 \}}}
%%% Dynamics %%%%%%%%%%%%%%%%%%%%%%%%%
\newcommand{\is}{\M{\mapsto}}
\newcommand{\Obj}[3]{ \M{\{ #1 \}^{#2}_{#3}}}
\newcommand{\Heap}[2]{\M{ #1[ #2 ] }}
%% Formatting %%%%%%%%%%%%%%%%%%%%%%%%%%%
\newcommand{\Alt}[1]{ &\B #1 \\}
\newcommand{\B}{\M{~|~}}
\newcommand{\M}[1]{\ensuremath{#1}\xspace}
\newcommand{\xt}[1]{\sf{#1}}
\newcommand{\bt}[1]{\xt{\bf #1}}
\renewcommand{\b}[1]{\M{\overline{#1}}}
\newcommand{\opdef}[2]{\framebox[1.1\width]{#1} ~ #2\\}

\newcommand{\IRule}[3]{\inferrule*[lab={\tiny #1}]{#2}{#3}}
%\newcommand{\IRule}[3]{\inferrule{#2}{#3}}    %% No label
\newcommand{\CondRule}[3]{ #3 & {if} #2 \\}
\newcommand{\NoCondRule}[2]{ #2 &       \\}
\newcommand{\Reduce}[4]{\M{ #1~#2 \rightarrow #3~#4}}
\newcommand{\ReduceA}[4]{\M{ #1 ~ #2 } &  \M { \rightarrow #3 ~ #4}}
\newcommand{\inc}{\M{\in}}
\newcommand{\Update}[3]{\M{#1[ #2 := #3]}}
\newcommand{\Bind}[2]{\M{#1 \is #2}}
\newcommand{\NotSub}{\M{\not<:}}
\newcommand{\Sub}{\M{<:}}
\newcommand{\classofis}[2]{\M{\xt{classof}(#1)=#2}}
\newcommand{\typeofis}[2]{\M{\xt{typeof}(#1)=#2}}
\newcommand{\classof}[1]{\M{\xt{classof}(#1)}}
\newcommand{\typeof}[1]{\M{\xt{typeof}(#1)}}
\newcommand{\Sel}[2]{\M{#1(#2)}}

\newcommand{\EnvType}[3]{ \M{#1 \vdash #2 : #3}}
\newcommand{\HasType}[3]{ \M{#1 (#2) = #3}}
\newcommand{\E}{\M{\Gamma}}
\newcommand{\Es}{\E ~\s}

%%%%%%%%%%%%%%%%%%%%%%%%%%%%%%%%%%%%%%%%%%%%%%%%%%%%%%%%%%%%%%%%%%%%%%%%%%%
\begin{document}
\title{Designing Gradual Types for Objects} 
\authorinfo{Benjamin Chung, Jan Vitek}{Northeastern University}{}
\maketitle

\begin{abstract}
The popularity of dynamically typed languages has given rise to a cottage
industry of incremental type systems. These type systems allow developers
to annotate their code with types and to feel better about the correctness
of their code.  The exact properties that these incremental type systems are
able to guarantee vary greatly from one language to another. This paper
compares some of the main design choices by embedding them into one common
core calculus representative of dynamic object-oriented languages such as
Python, TypeScript or Ruby.
\end{abstract} 

\section{Introduction}

\newpage

%%%%%%%%%%%%%%%%%%%%%%%%%%%%%%%%%%%%%%%%%%%%%%%%%%%%%%%%%%%%%%%%%%%%%%%%%%%
\section{A Common Core}

A common core calculus is used to compare the various gradual type
systems. This common core is an imperative object-based system with
structural subtyping. Figure~\ref{syn} summarize the syntax of this small
language where \x ranges over variables, \f ranges over field names, \m can
be either a plain method name \d, a field getter \f, or a field setter \fb,
\C and \D range over class names. We use the overbar notation to denote a
possibly empty sequence. In the common core, a class is defined as by a set
of fields, \b\f, with distinct names and a set of method definitions,
\b\m. An instance of class is constructed by the expression \New\C{\b\e} where
each argument is used to initialize the corresponding field. Fields are
encapsulated and can only be accessed and updated in the corresponding
getter and setter from the \this variable.  Expressions include variable
access, field access and update, method invocation, object creation, and
type casts.



%%%%%%%%%%%%%%%%%%%%%%%%%%%%% SYNTAX %%%%%%%%%%%%%%%%%%%%%%%%%%%%%%%%%%%%%%%%%%%%%%%%
\begin{figure}[!h]\begin{minipage}{3cm}\begin{tabular}{l@{~~~}l}
\e &::=  \x \\
   \Alt{ \Get\e\f }
   \Alt{ \Set\e\f\e }
   \Alt{ \Call\e\m{\b\e} }
   \Alt{ \New\C{\b\e} }
   \Alt{ \Cast\t\e }
\end{tabular}\end{minipage}\begin{minipage}{3cm}\begin{tabular}{l@{~~~}l}
\fd &::= 
    \Ftype\f\t   \\
\md &::=
    \Mdef\m\x\t\t\e \\
\c &::= \Class \C {\b{\fd}}{\b{\md} } \\
\mt &::= \Mtype\m\t\t\\
\t &::= ~ \any \\
   \Alt{ \Type{  \b{ \mt } } }
\end{tabular}\end{minipage}
\caption{Abstract Syntax for the Core Calculus}\label{syn}
\end{figure}
%%%%%%%%%%%%%%%%%%%%%%%% END OF SYNTAX %%%%%%%%%%%%%%%%%%%%%%%%%%%%%%%%%%%%%%%%%%%%%%%%%%%%

Our types are structural and only account for the methods in an object's
interface. For type annotations, we use the name of classes, ranged over by
\C and \D, as shorthand for the set of methods defined in the respective
classes.


The dynamic semantics evaluates expression extended with object references,
denoted \a, and errors, denoted \err, together with a heap \s mapping
references to object values. Object values contain all fields, methods, a
type, and a class; they are denoted \Obj{\b{\Fdef\f\t\a}}{\tp 1}{\C}. The
class is used for locating methods and the type is used for type casts. The
need for keeping them separate will become clear later.

The semantics uses evaluation context \M{\F[\e]}.

Selecting an object from the heap is written \Sel\s\a, while a heap is
extended with a new object by \Heap{\s}{\Bind{\a}{\Obj{\dots}\t\C}}.


For an object reference \a, such that \M{\Sel\s\a=\Obj{\dots}\t\C}, we have 
\classofis{\Sel\s\a}\C and \typeofis{\Sel\s\a}\t.

We use the notation \Mdef\m\x\t\t \inc\C to select a method in a class
definition and \Fdef\f\t\a\inc\Obj{\dots}\t\C to express the selection of a
field. Lastly a field of an object can be update with the notation
\Update{\Obj{\dots}\t\C}\f\a.

% TODOS:
% $ should not be a valid character in a method name <- explain in text
% c translates to c with $ <- explain in text
% Figure 2 sigma <- not sure?

%%%%%%%%%%%%%%%%%%%%%%%%%%% DYNAMIC SEMANTICS %%%%%%%%%%%%%%%%%%%%%%%%%%%%%%%%%%%%%%%%%%%%%%%%%%%%%%
\begin{figure*}
\begin{tabular}{l|@{~~~~~~~~}l}
\begin{minipage}{11cm}
\opdef{\Reduce{\ep 1}{\sp 1}{\ep 2}{\sp 2}}{\ep 1 with heap \sp 1 evaluates to \ep 2 and \sp 2 in a step}

\begin{tabular}{@{}l@{}l@{~}l@{~}l}
\CondRule{E1}{ 
  \Reduce {\ep 1}{\sp 1}{\ep 2}{\sp 2}
}{
  \ReduceA {\F[\ep 1]}{\sp 1}{\F[\ep 2]}{\sp 2}
}
\CondRule{E2}{
     \sp 2 = \Heap{\sp 1}{ \Bind{\ap 1}{\Obj{\b{\Fdef\f\t\a}}\C\C\}}}
}{ 
    \ReduceA{ \New\C{\b\a} }{\sp 1}{\ap 1}{\sp 2}
}
\CondRule{E3}{
     \Mdef\m\x{\tp 1}{\tp 2}\e\inc \classof{\Sel\s\a}
}{
   \ReduceA{\Call\a\m{\b{\ap 1}}}\s{[\a/\this~\b{{\ap 1}/\x}]\e}\s
}

\CondRule{E3}{
     \Mdef\m\x{\tp 1}{\tp 2}\e\M{\not\in} \classof{\Sel\s\a}
}{
   \ReduceA{\Call\a\m{\b{\ap 1}}}\s{\err}\s
}

\CondRule{E3}{ 
     \Fdef\f\t{\ap 1}\inc \Sel\s\a
}{
  \ReduceA{\Get\a\f}{\s}{\ap 1}{\s}
}
\CondRule{E4}{
     \sp 2 = \Heap{\sp 1}{\Bind\a{\Update{\Sel{\sp 1}\a}\f {\ap 1}}}
}{
     \ReduceA{\Set\a\f{\ap 1}}{\sp 1}{\ap 1}{\sp 2}
}
\NoCondRule{E5}
{ 
   \ReduceA{ \Cast\any\a}\s \a\s
}
\CondRule{E6}{
  \typeof{\Sel\s\a} \Sub \tp 1
}{ 
    \ReduceA{\Cast{\tp 1}\a}\s\a\s
}
\CondRule{E7}{
  \typeof{\Sel\s\a} \NotSub \tp 1
}{ 
    \ReduceA{\Cast{\tp 1}\a}\s\err\s
}
\CondRule{E8}{
    \Reduce\e{\sp 1}\err{\sp 2}
}{
    \ReduceA{\F[\e]}{\sp1}\err{\sp 2}
}
\end{tabular}\end{minipage}&
%%%%%%%%%%%%%%%%%%% CONTEXTS %%%%%%%%%%%%%%%%%%%%%%%%%%%%%%%%%%%%%%%%%%%%
\begin{minipage}{4cm}\begin{tabular}{l@{~~}l}
\fv &::= \Fdef\f\t\a \\
\s &::= \none \\
\Alt{  \Heap\s{\Bind\a{\Obj{\b{\fv}}} } } 
\F &::=    \Get\square\f \\
   \Alt{    \Set\square\f\e}
   \Alt{    \Set\a\f\square    }
   \Alt{    \Call\square\m\e}
   \Alt{    \Call\a\m{\b\a\,\square\,\b\e}}
    \Alt{   \Cast\t\square }
   \Alt{   \New\C{\b \a\,\square\,\b\e}}
\end{tabular}
\end{minipage}\end{tabular}
\caption{Common Core Calculus Dynamic Semantics.}
\end{figure*}





%%%%%%%%%%%%%%%%%%%%%%%% SUBTYPING %%%%%%%%%%%%%%%%%%%%%%%%%%%%%%%%%%%%%%%%
\begin{figure}
\opdef{\tp 1 \Sub \tp 2 }{\tp 1 is a subtype of \tp 2}
\begin{mathpar}
\IRule{S1}{}{\t\Sub\t}

\IRule{S2}{}{\t \Sub \Type{} }

\IRule{S3}{
    \t \Sub \Type{\b{ \mt } } \\
     \Mtype\m{\b{\tp3}}{\tp4} \inc \t \\
    \tp2 \Sub \tp4 \\    
    \b{\tp3 \Sub \tp1}
}{
   \t \Sub \Type{\Mtype\m{\b{\tp1}}{\tp2} ~ \b{\mt} \, } 
}
\end{mathpar}
\caption{Subtyping}
\end{figure}


%%%%%%%%%%%%%%%%%%%%%%%%%%%%%%% WELLFORMEDNESS %%%%%%%%%%%%%%%%%%%%%%%
\begin{figure}
\opdef{\EnvType\E\e\t}{\e has type \t in environment \E}
\begin{mathpar}
\IRule{W1}{
    \HasType\E\x\t
 }{
   \EnvType\E\x\t
}

\IRule{W2}{
   \EnvType\E\e{\tp 1} \\ \tp 1 \Sub \t
 }{
   \EnvType\E\e\t 
}   

\IRule{W3}{
   \EnvType\E\e\any \\ \b{\EnvType\E{\ep1}\t }
}{
   \EnvType\E{\Call\e\m{\b{\ep1}}}\any
}

\IRule{W4}{
   \EnvType\E\e{\tp1} \\ \Mtype\m{\b{\tp2}}\t\inc \tp1  \\  \b{\EnvType\E{\ep1}{\tp 2}}
}{
    \EnvType\E{\Call\e\m{\b{\ep1}}}\t
}    

\IRule{W5}{
  \b{\EnvType\E\e\t} \\ 
  \Class \C {\b{\Ftype\f\t}} {\b{\md}}
}{
  \EnvType\E{\New\C{\b\e}}\C
}

\IRule{W6}{
  \EnvType\E\e{\tp1}
}{
   \EnvType\E{\Cast\t\e}\t
}

\IRule{W7}{
	\M{\forall} \Mdef\m\x{\tp1}{\tp2}\e \inc \b\md ~\B~
       \EnvType{\this:\C,\x:\tp1}\e{\tp2}
}{ 
	\Class \C {\b{\fd}} {\b{\md}}  ~~ WF
}
\end{mathpar}
\caption{Class well-formedness}
\end{figure}
%%%%%%%%%%%%%%%%%%%%%%%%% WELLFORMDNESS %%%%%%%%%%%%%%%%%%%%%%%%%%%%%%%%%%

\newcommand{\TransClass}[2]{\M{ #1 \hookrightarrow #2 }}
\newcommand{\TransExp}[4]{\M{ #1 \vdash #2 \hookrightarrow #2 ~ #3 }}
\newcommand{\mdp}[1]{\M{\md_{#1}}}
\newcommand{\setter}[1]{\M{\llbracket #1  \rrbracket}_{\xt{set}}}
\newcommand{\getter}[1]{\M{\llbracket #1  \rrbracket_{\xt{get}}}}
\newcommand{\dynamic}[1]{\M{\llbracket #1  \rrbracket_{\xt{dyn}}}}
\newcommand{\invoke}[1]{\M{\llbracket #1 \rrbracket_{\xt{inv}}}}
\newcommand{\Dyn}[1]{\M{#1^{\any} }}
%%%%%%%%%%%%%%%%%%%%%%%%% CLASS TRANSLATION %%%%%%%%%%%%%%%%%%%%%%%%%%%%%
\begin{figure}
\opdef{\EnvType\E\e\t}{\e has type \t in environment \E}
\begin{mathpar}
\IRule{}{
      \TransExp\E\e{\ep2}\any\\
     \b{\TransExp\E{\ep1}{\ep3}{\t3}}
 }{
    \TransExp\E{\Call\e\m{\b{\ep1}}} {\Call{\ep2}{\Dyn{\m}}{\b{\ep3}}} \any
}   

\IRule{}{
   \EnvType\E{\ep 1}\any \\ \b{\EnvType\E{\ep 2}\t }
}{
   \EnvType\E{\Call{\ep 1}\m{\b{\ep 2}}}\any
}

\IRule{}{
  \TransExp{\this:\C,\x:\tp1}\e{\ep1}{\tp2}
}{
   \invoke{\Mdef\d\x\t{\tp1}\e  } \equiv  \Mdef\d\x\t{\tp1}{\ep1}
}    

\IRule{}{
}{
   \dynamic{\Mdef\d\x\t{\tp1}\e} \equiv  \Mdef{\Dyn\d}\x\any\any{ \Cast\any{\Call\this\d{\b{\Cast\t\x }} }}
}

\IRule{}{
}{
  \getter{\Ftype\f\t} \equiv   \GMdef\m\t{\Get\this\f}
}

\IRule{}{
}{
  \setter{\Ftype\f\t} \equiv   \SMdef\m\x\t\t{\Set\this\f\x}
}

\newcommand{\union}{\M{\cup}}
\IRule{}{
 \b{ \mdp1} \equiv \b{\setter\fd} ~~ \b{\getter\fd} ~~ \b{\dynamic\md} ~~ \b{\invoke\md}
}{ 
	\TransClass { \Class \C {\b{\fd}} {\b{\md}} }{  \Class \C {\b{\fd}} {\b{\mdp1}} }
}
\end{mathpar}
\caption{Translations}
\end{figure}
%%%%%%%%%%%%%%%%%%%%%%%%%%%%%%%%%%%%%%%%%%%%%%%%%%%%%%%%%%%%%%%%%%%%%%%%%




\newcommand{\GenCast}[6]{#1 \vdash_{#2}^{#3} #4 \hookrightarrow #5 \Rightarrow #6}
\newcommand{\AnaCast}[6]{#1 \vdash_{#2}^{#3} #4 \rightsquigarrow #5 \Leftarrow #6}
\begin{figure}
\begin{mathpar}
\IRule{A1}{\HasType\E\x\t}{\GenCast\E\C\d\x\x\t}

\IRule{A2}{ }{\GenCast{\E}{\C}\f{\Get\this\f}{\Get\this\f}{\t}}

\IRule{A3}{\Ftype\f\t \inc \C \\ \AnaCast\E\C{\f!}{\e_1}{\e_2}{\t}}{\GenCast{\E}{\C}{\f!}{\Set\this\f{\e_1}}{\Set\this\f{\e_1}}{\t}}

\IRule{A4}{
	\GenCast{\E}{\C}\d{\e_1}{\e_2}{\t_1} \\
	\Mtype\m{\b{\tp 2}}{\tp 3} \inc \t_1\\
	\b{\AnaCast{\E}{\C}\d{\e_3}{\e_4}{\tp 2}} \\
}{
	\GenCast{\E}{\C}\d{\Call{\e_1}\m{\b{\e_2}}}{\Call{\e_3}{\m}{\b{\e_4}}}{\tp 3}
}

\IRule{A5}{
	\GenCast{\E}{\C}\d{\e_1}{\e_1'}{\any} \\
	\b{\AnaCast{\E}{\C}\d{\e_2}{\e_2'}{\any}} \\
}{
	\GenCast{\E}{\C}\d{\Call{\e_1}\m{\b{\e_2}}}{\Call{\e_1'}{\m\$}{\b{\e_2'}}}{\any}
}

\IRule{A6}{
  \b{\AnaCast\E\C\d{\e_1}{\e_2}\t} \\ 
  \Class \C {\b{\Ftype\f\t}} {\b{\md}}
  }{\GenCast{\E}{\C}\d{\New\C{\b{\e_1}}}{\New\C{\b{\e_2}}}{\C}}

\IRule{A7}{
	\GenCast{\E}{\C}\d{\e_1}{\e_2}{\t_2}
}{\GenCast{\E}{\C}\d{\Cast{\tp 1}{\e_1}}{\Cast{\t_1}{\e_2}}{\tp 1}}
\end{mathpar}
\caption{Cast insertion}
\end{figure}

\section{Appendix}
\begin{figure}
\begin{mathpar}
\IRule{}{
  \Ftype\f{\t} \inc \classof{\Sel\s\a}
}{ 
 \EnvType\Es\C\f{\Get\a\f}{\t}
}

\IRule{}{
  \EnvType\Es\C\f\e{\tp1} \\
  \Ftype{\f!}{\tp2}\inc\classof{\Sel\s\a}\\
  \tp1 \Sub \tp2
}{ 
    \EnvType\Es\C{\f!}{\Set\a\f\e}{\tp1} 
}

\IRule{}{
   \typeofis{\Sel\s\a}\t 
}{
   \EnvType\Es\C\a\t
}
\end{mathpar}
\caption{Runtime typing rules}
\end{figure}
\end{document}

