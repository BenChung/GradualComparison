\documentclass[sigconf]{acmart}

\usepackage{xspace,listings,url,framed,amssymb,colortbl,
            amsmath,tex/mathpartir,hyperref,doi, rotating,
            stmaryrd, graphicx, tikz, colortbl, xparse, etoolbox, pgffor,booktabs} % double brackets llbracket
%% Formatting
\newcommand{\EM}[1]{\ensuremath{#1}\xspace}
\newcommand{\xt}[1]{{\sf{#1}}}
\newcommand{\bt}[1]{\xt{\bf #1}}
\renewcommand{\b}[1]{\EM{\overline{#1}}}
\newcommand{\EMxt}[1]{\EM{\xt{#1}}}
\newcommand{\EMbt}[1]{\EM{\bt{#1}}}

%% Variables
\newcommand{\x}   {\EMxt x}
\newcommand{\n}   {\EMxt n}
\newcommand{\e}   {\EMxt e}
\newcommand{\m}   {\EMxt m}
\newcommand{\s}   {\EM{\sigma}}
\renewcommand{\t} {\EMxt t}
\newcommand{\ta}  {\EM{\tau}}
\renewcommand{\a} {\EMxt a}
\newcommand{\K}   {\EMxt K}
\renewcommand{\k} {\EMxt k}
\newcommand{\Kp}  {{\EMxt{K'}}}
\newcommand{\Kpp}  {{\EMxt{K''}}}
\newcommand{\Kppp}  {{\EMxt{K'''}}}
\newcommand{\ep}  {{{\EMxt{e'}}}}
\newcommand{\epp}  {{{\EMxt{e''}}}}
\renewcommand{\sp}{{{\EM{\s'}}}}
\newcommand{\spp}{{{\EM{\s''}}}}
\newcommand{\ap}  {\EM{\a'}}
\newcommand{\aE}[1]  {\EM{\a_{#1}}}
\newcommand{\app}  {\EM{\a''}}
\newcommand{\tp}  {\EM{ \t'}}
\newcommand{\tpp}  {\EM{ \t''}}
\newcommand{\C}   {\EMxt C}
\newcommand{\Cp}  {\EMxt{C'}}
\newcommand{\EC}   {\EMxt E}
\newcommand{\fd}  {\EMxt{fd}}
\newcommand{\md}  {\EMxt{md}}
\newcommand{\mdpp}  {\EM{\md'}}
\newcommand{\mt}  {\EMxt{mt}}
\newcommand{\mtp}  {\EMxt{mt'}}
\newcommand{\mtpp}  {\EMxt{mt''}}
\newcommand{\M}{\EMxt M}
\newcommand{\MN}  {\EMxt{M\,K}}
\newcommand{\MNargs}[1]  {\EMxt {M #1~K}}
\newcommand{\f}   {\EMxt f}
\newcommand{\fp}   {\EMxt{ f'}}
\newcommand{\E}   {\EM{\Gamma}}
\newcommand{\EE}   {\EM{\mathcal{E}}}
\newcommand{\any} {\EM{\star}}
\newcommand{\this}{\EMxt{this}}
\newcommand{\that}{\EMxt{that}}
\newcommand{\none}{\EM{\cdot}}
\newcommand{\D}   {\EMxt D}
\newcommand{\Dp}   {\EMxt{D'}}
\newcommand{\p}   {\EMxt p}
\newcommand{\np}{\n'}

\newcommand{\Get}[2]   {\EM{#1.#2()}}
\newcommand{\Set}[3]   {\EM{#1.#2(#3)}}
\newcommand{\Call}[3]  {\EM{#1.#2(#3)}}
\newcommand{\DynCall}[3]  {\EM{#1@#2(#3)}}

\newcommand{\New}[2]   {\EM{\new\;#1(#2)}}
\newcommand{\SubCast}[2]{\EM{<\hspace{-.6mm}{#1}\hspace{-.6mm}>\hspace{-1mm}\;{#2}}}
\newcommand{\ShaCast}[2]{\EM{\prec #1 \succ #2}}
\newcommand{\MonCast}[2]{\EM{\triangleleft\; #1 \triangleright #2}}
\newcommand{\BehCast}[2]{\EM{\blacktriangleleft #1 \blacktriangleright #2}}
\newcommand{\new}      {\EM{\bt{new}}}
\newcommand{\HT}[2]    {\EM{{#1}\!:{#2}}}
\newcommand{\Mdef}[5]  {\EM{ \HT{ #1( \HT{#2}{#3})}{#4}\;\{{#5}\}}}
\newcommand{\Mdefz}[3] {\EM{ \HT{ #1()}{#2}\;\{{#3}\}}}
\newcommand{\Mdefa}[4]  {\EM{ \HT{ #1( #2 )}{#3}~\{{#4}\}}}
\newcommand{\obj}[2]   { \EM{ #1\{#2\}}}
\newcommand{\alloc}[4] {\EM{#1\;#2  = \xt{alloc}(#3, #4)}}
\newcommand{\cast}[8]  {\EM{#6\;#7\;#8=\xt{#5 cast}(#1, #2, #3, #4)}}
\newcommand{\behcast}[7]  {\EM{\xt{behcast}(#1, #2, #3, #4)=#5\,#6\,#7}}
\newcommand{\moncast}[6]  {\EM{\xt{moncast}(#1, #2, #3, #4)=#5\,#6}}

\newcommand{\Alt}[1]   { &\B #1 \\}
\newcommand{\B}        {\EM{~|~}}
\newcommand{\bang}     {\EM{\xt{!}}}

\newcommand{\dispatch}[5] {\EM{#1\;#2 = \xt{disp}(#3,#4,#5)}}
\newcommand{\readf}[4]{\EM{\xt{read}(#1,#2,#3,#4)}}
\newcommand{\convert}[1]{\EM{\xt{cnvtMD}(#1)}}
\newcommand{\convertFD}[1]{\EM{\xt{cnvtFD}(#1)}}
\newcommand{\readfield}[4]{\EM{#1 = \xt{read}(#2,#3,#4)}}
\newcommand{\setf}[5] {\EM{\xt{write}(#1,#2,#3,#4,#5)}}
\newcommand{\Reduce}[6]   {\EM{{#1}~{#2}~{#3} \rightarrow {#4}~{#5}~{#6}}}
\newcommand{\ReduceA}[6]  {\EM{#1~#2~#3 } & \EM{\rightarrow #4~#5~#6}}
\newcommand{\Class}[3]    {\EM{\bt{class}\;#1\,\{\,#2~#3\,\}}}
\newcommand{\Ftype}[2]    {\EM{ \HT{#1}{#2} }}
\newcommand{\Fdef}[2]    {\EM{ \HT{#1}{#2} }}
\newcommand{\Mtype}[3]    {\EM{ \HT{#1(#2)}{#3}}}
\newcommand{\Type}[1]     {\EM{\{#1\}}}

\newcommand{\opdef}[2]    {\framebox[1.1\width]{#1} ~ #2\\}
\newcommand{\Map}[2]     {\EM{ #1[#2] }}
\newcommand{\Bind}[2]     {\EM{#1 \mapsto #2}}

\newcommand{\Sub}{\EM{<:}}
\newcommand{\OK}{\EM{~\checkmark}}
\newcommand{\SubE}[1]{\EM{<:_{#1}}}
\newcommand{\names}[1]{\EM{\xt{names}(#1)}}
\newcommand{\untyped}[1]{\EM{\xt{untyped}(#1)}}

\newcommand{\mnames}[1]{\EM{\xt{methName}(#1)}}
\newcommand{\fnames}[1]{\EM{\xt{fieldName}(#1)}}


\newcommand{\ConsSub}{\EM{\lesssim}}

\newcommand{\CondRule}[3]{ #3 &~ #2 \\}
\newcommand{\SuchRule}[3]{ #3 &~{\emph{s.t.}} #2 \\}
\newcommand{\EnvType}[5]{ \EM{#1\,#2\,#3\vdash #4 : #5}}

\newcommand{\IRule}[4][]{\inferrule*[lab={\tiny #2},#1]{#3}{#4}}
\newcommand{\HasType}[3]{ \EM{#1 (#2) = #3}}
\newcommand{\wrapper}[1]{\EM{\xt{wrap}(#1)}}
\newcommand{\spec}[4]{\EM{\xt{spec}(#1,#2,#3,#4)}}

\newcommand{\castfn}[4]{\text{cast}(#1,#2,#3,#4)}
\newcommand{\GenCast}[5]{#1~#2 \vdash #3 \hookrightarrow #4 \Uparrow #5 }
\newcommand{\AnaCast}[5]{#1~#2 \vdash #3 \Downarrow #5 \hookrightarrow #4}
\newcommand{\TransClass}[2]{\EM{ #1 \rightharpoonup #2 }}
\newcommand{\inv}[2]{\xt{invoke}(#1, #2)}
\newcommand{\classoff}[2]{\EM{\xt{mtypes}(#1,#2)}}
\newcommand{\classoffs}[3]{\EM{\xt{mtypes}(#1,#2,#3)}}
\newcommand{\mtype}[3]{\EM{\xt{mtype}(#1,#2,#3)}}
\newcommand{\wftype}[3]{\EM{\xt{wftype}(#1,#2,#3)}}

\newcommand{\field}[2]{\EM{\xt{field}(#1,#2)}}
\newcommand{\In}{\EM{\in}}

\newcommand{\T}{\EM{\xt T}}
\newcommand{\Cast}{Cast }
\newcommand{\fb}{\EM{\xt{f!}}}

\newcommand{\AND}{\EM{\wedge}}
\newcommand{\App}[2]{\EM{#1(#2)}}

\newcommand{\StrSub}[4]{\EM{#1~#2\vdash #3\Sub #4}}
\newcommand{\tmeet}[4]{\xt{tmeet}(#1,#2,#3,#4)}
\newcommand{\mmeet}[4]{\xt{mmeet}(#1,#2,#3,#4)}
\newcommand{\mtypes}[2]{\xt{mtypes}(#1,#2)}


\newcommand{\WFtype}[2]{\EM{#1\vdash#2 \OK}}
\newcommand{\WF}[4]{\EM{#1\,#2\,#3\vdash#4 \OK}}
\newcommand{\WFp}[3]{#1~#2~#3\OK}

\renewcommand{\P}{\EMxt P}
\newcommand{\Pp}{\EMxt{P'}}


\newcommand{\retype}[5]{\xt{retype}(#1,#2,#3,#4,#5)}
\newcommand{\htype}[3]{\EM{\xt{htype}(#1,#2,#3)}}
\newcommand{\ftypes}[4]{\xt{ftypes}(#1,#2,#3,#4)}
\newcommand{\typeof}[1]{\xt{typeOf}(#1)}
\newcommand{\classgen}[1]{\xt{classgen}(#1)}
\renewcommand{\S}{\EM{\tau}}
\newcommand{\Sp}{\EM{\tau'}}
\newcommand{\Spp}{\EM{\tau''}}
\newcommand{\EQ}{\EM{\equiv}}

\newcommand{\Dom}[1]{\EM{\xt{dom}(#1)}}
\newcommand{\fresh}[1]{\EM{#1~\xt{fresh}}}

\newcommand{\progtrans}[2]{#1 ~\hookrightarrow_p~ #2}
\newcommand{\classtrans}[3]{#1 \vdash #2 ~\hookrightarrow_c~ #3}
\newcommand{\methtrans}[4]{#1~#2 \vdash #3 ~\hookrightarrow_m~ #4}
\newcommand{\statictype}[2]{\xt{static}(#1,#2)}


% Copyright
%\setcopyright{none}
%\setcopyright{acmcopyright}
%\setcopyright{acmlicensed}
\setcopyright{rightsretained}
%\setcopyright{usgov}
%\setcopyright{usgovmixed}
%\setcopyright{cagov}
%\setcopyright{cagovmixed}


% DOI
\acmDOI{10.475/123_4}

% ISBN
\acmISBN{123-4567-24-567/08/06}

%Conference
\acmConference[WOODSTOCK'97]{ACM Woodstock conference}{July 1997}{El
  Paso, Texas USA}
\acmYear{1997}
\copyrightyear{2016}
\acmArticle{4}
\acmPrice{15.00}


\begin{document}
\title{Monotonic Gradual Typing in a Common Calculus}
\subtitle{Extended Abstract}


\author{Ben Trovato}
\authornote{Dr.~Trovato insisted his name be first.}
\orcid{1234-5678-9012}
\affiliation{%
  \institution{Institute for Clarity in Documentation}
  \streetaddress{P.O. Box 1212}
  \city{Dublin}
  \state{Ohio}
  \postcode{43017-6221}
}
\email{trovato@corporation.com}

\author{John Smith}
\affiliation{\institution{The Th{\o}rv{\"a}ld Group}}
\email{jsmith@affiliation.org}

% The default list of authors is too long for headers.
\renewcommand{\shortauthors}{B. Trovato et al.}


\begin{abstract}
This paper provides a sample of a \LaTeX\ document which conforms,
somewhat loosely, to the formatting guidelines for
ACM SIG Proceedings.\footnote{This is an abstract footnote}
\end{abstract}

%
% The code below should be generated by the tool at
% http://dl.acm.org/ccs.cfm
% Please copy and paste the code instead of the example below.
%
\begin{CCSXML}
<ccs2012>
 <concept>
  <concept_id>10010520.10010553.10010562</concept_id>
  <concept_desc>Computer systems organization~Embedded systems</concept_desc>
  <concept_significance>500</concept_significance>
 </concept>
 <concept>
  <concept_id>10010520.10010575.10010755</concept_id>
  <concept_desc>Computer systems organization~Redundancy</concept_desc>
  <concept_significance>300</concept_significance>
 </concept>
 <concept>
  <concept_id>10010520.10010553.10010554</concept_id>
  <concept_desc>Computer systems organization~Robotics</concept_desc>
  <concept_significance>100</concept_significance>
 </concept>
 <concept>
  <concept_id>10003033.10003083.10003095</concept_id>
  <concept_desc>Networks~Network reliability</concept_desc>
  <concept_significance>100</concept_significance>
 </concept>
</ccs2012>
\end{CCSXML}

\ccsdesc[500]{Computer systems organization~Embedded systems}
\ccsdesc[300]{Computer systems organization~Redundancy}
\ccsdesc{Computer systems organization~Robotics}
\ccsdesc[100]{Networks~Network reliability}


\keywords{ACM proceedings, \LaTeX, text tagging}


\maketitle


\section{Complete translation for Monotonic}

\subsection{Monotonic cast static and dynamic rules}

\begin{minipage}{0.35\textwidth}
\begin{mathpar}
\IRule{W10}{
  \EnvType \Env\s\K\e\tp
}{
  \EnvType \Env\s\K{\MonCast\t\e}\t
}
\end{mathpar}
\end{minipage}
\begin{minipage}{0.5\textwidth}
\begin{tabular}{l@{}l@{~}l@{~}l}
\CondRule{E11}{  %Monotonic cast  
  \moncast \a\t\s\K  \Kp\ap\sp    
}{    
  \ReduceA  \K{\MonCast \t\a}\s \Kp\ap\sp   
} \\
\multicolumn{4}{l}{\EE ::= \ldots \B \MonCast\t\EE }
\end{tabular}
\end{minipage}

\subsection{Monotonic translation for program, class, and method}

\opdef{$\progtrans{\e~\K}{\e'~{\K'}}$}{Monotonic translation for programs}
\opdef{$\classtrans {\K}{\Class \C{...}{...}~\K}{\Class \C{...}{...}~\K}$}{Monotonic translation for classes}
\opdef{$\methtrans {\K}{\C}{\md}{\md'}{\K}$}{Monotonic translation for methods}

\begin{mathpar}
\IRule{PT}{
  {\classtrans{\K}{\K}{\K'}} \\ \GenCast{\K}{\cdot}{\e}{\ep}{\t} 
}{\progtrans{\e~\K}{\e'~{\K'}}}

\IRule{MCT1}{
  \D \text{ fresh}\\
  \k = \classgen{\C,\getmds\C\K,{\classoff\C\K},{\classoff\C\K},\D,\K}
}{
  \monowrap{\C}{\K} = \D~\k
}

\IRule{MCT2}{
}{
  \monowrap{\any}{\K} = \any
}

\IRule{CR1}{ 
  \b{\methtrans \K\C\md{\md'}{\K_m}} \\
  \classtrans \K\Kp\Kpp \\
  \b{\monowrap\t\Kpp = \tp~\Kppp}
}{
   \classtrans \K{\Class \C{\b{\Ftype\f\t}}{\b\md}~\Kp}{\Class \C {\b{\Ftype\f\tp}}{\b{\md'}}~\Kpp~\K_m~\b{\Kppp}}}

\IRule{CR2}{ 
}{
  \classtrans \K\cdot\cdot
}

\IRule{MTT}{
  \AnaCastMono \K{\HT\this\C~\HT\x\t}\e\ep\tp{\K_1} \\
  \monowrap{\t}\K = \t[2]~\K_2 \\
  \monowrap{\tp}\K = \tp[2]~\K_3
}{
  \methtrans \K\C{\Mdef\m\x{\t}{\tp}\e}{\Mdef\m\x{\t_2}{\tp[2]}\ep}{\K_1~\K_2~\K_3}
}
\end{mathpar}

\subsection{Monotonic translation for expressions}

\opdef{$\GenCastMono{\K}\E\e\ep\t{\K}$}{Monotonic translation for expressions}


\begin{mathpar}
\IRule{MOS1}{\HasType{\E}\x\t}{\GenCastMono{\K}\E\x\x\t{}}

\IRule[width=30em]{MOS2}{
    \GenCastMono\K\Env{\e_1}{\e_3}{\C}{\K_1} \\ \src{\f(\b{\t_1}):\t_2 \in \classoff\C\K} \\ \b{\AnaCastMono\K\Env{\e_2}{\e_4}{\t_1}{\K_2}}
}{
    \GenCastMono\K\Env{\Call{\e_1}\f{\b{\e_2}}}{\Call{\e_3}\f{\b{\e_4}}}{\t_2}{\K_1~\b{\K_2}}
}

\IRule[width=30em]{MOS3}{
    \GenCastMono\K\Env{\e_1}{\e_3}{\C}{\K_1} \\ 
    \m({\t_1}):\t_2 \in \classoff\C\K \\ 
    \statictype{\t_1}{\K}{\cdot} \\ 
    {\AnaCastMono\K\Env{\e_2}{\e_4}{\t_1}{\K_2}}
}{
    \GenCastMono\K\Env{\Call{\e_1}\m{{\e_2}}}{\KCall{\e_3}\m{{\e_4}}{\t_1}{\t_2}}{\t_2}{\K_1~\b{\K_2}}
}

\IRule[width=30em]{MOS3}{
    \GenCastMono\K\Env{\e_1}{\e_3}{\C}{\K_1} \\ \m({\t_1}):\t_2 \in \classoff\C\K \\ \lnot\statictype{\t_1}{\K}{\cdot} \\ \b{\AnaCastMono\K\Env{\e_2}{\e_4}{\t_1}{\K_2}}
}{
    \GenCastMono\K\Env{\Call{\e_1}\m{{\e_2}}}{\KCall{(\MonCast\any\e_3)}\m{\MonCast\any{\e_4}}\any\any}{\t_2}{\K_1~\b{\K_2}}
}

\IRule{MOS4}{
    \GenCastMono\K\Env{\e_1}{\e_3}{\any}{\K_1} \\ {\AnaCastMono\K\Env{\e_2}{\e_4}{\any}{\K_2}}
}{
    \GenCastMono\K\Env{\Call{\e_1}\m{{\e_2}}}{\DynCall{\e_3}\m{{\e_4}}}{\any}{\K_1~\K_2}
}

\IRule{MOS5}{
  \b{\AnaCastMono{\K}\E{\e_1}{\e_2}\t{\K}} \\ 
  \Class \C {\b{\Ftype\f\t}} {\b{\md}} \\
  \D~\text{fresh} \\
  \k = \classgen{\C,\getmds\C\K,\classoff\C\K,\classoff\C\K,\D,\K} \\
  }{\GenCastMono\K\Env{\New\C{\b{\e_1}}}{\New\D{\New\C{\b{\e_2}}}}{\C}{\K~\k}}
\end{mathpar}

\subsection{Monotonic translation for bidirectional expressions}

\opdef{$\AnaCastMono\K\Env\e\ep\t\K$}{Monotonic translation for bidirectional expressions}


\begin{mathpar}
\IRule{MOA1}{
  \GenCastMono\K\Env\e\ep\tp\K \\
  \K \vdash \tp \Sub \t
}{
  \AnaCastMono\K\Env\e\ep\t\K
}

\IRule{MOA2}{
  \GenCastMono\K\Env\e\ep\tp\K \\
  \consistent\K\t\tp
}{
  \AnaCastMono\K\Env\e{\MonCast\t\ep}\t\K
}
\end{mathpar}

\section{Generative Monotone Casts}

\subsection{Retype function}\label{retype}

Formally, the \xt{retype} function takes a list of object
addresses \b\a and a list of types to ascribe to them \b\t, and updates the
heap \s and class table \K. 

\begin{align*}
\S &::= \cdot \B \Map\S{\Bind\a\C}
\end{align*}
\begin{mathpar}
\IRule{CRM1}{
  \htype \a\S\s\K = \C~\Kp\\
  \tmeet\C\t\cdot\Kp = \D\,\Kpp\\ 
  \C\not\EQ\D  \\
  \ftypes \a\D \s\Kpp = \b\ap~\b\tp \\
  \Sp = \Map\S{\Bind\a\D } \\
  \retype{\b\ap}{\b\tp}\Sp\s\K = \Spp\,\K'''
}{
  \retype \a\t\S\s\K = \Spp\,\K'''
}

\IRule{CRM2}{
  \htype \a\S\s\K = \tp\,\Kp\\
  \tmeet\tp\t\cdot\Kp = \tpp\,\Kp \\
  \tp\EQ\tpp \\
}{
  \retype \a\t\S\s\K = \S~\Kpp
}

\IRule{CRM3}{
  \retype\a\t\S\s\K = \Sp\,\Kp\\
  \retype{\b\a}{\b\t}\Sp\s\Kp = \Spp\,\Kpp
}{
  \retype {\a\,\b\a}{\t\,\b\t}\S\s\K = \Spp\,\Kpp
}
\end{mathpar}

\subsection{Spec function}\label{mono:spec}

Formally, the \xt{spec} (heap specialization) function takes a
sequence of object addresses \b\a, a heap typing \S, a heap \s
and a class table \K and returns a new heap where the objects
have been retyped. \Dom\S retrieves the list of addresses 
that have to be retyped.

\begin{mathpar}
\IRule{CMS1}{
  \E \text{ fresh}\\
  \D = \App\S\a \\
  \obj\C{\ap} = \App\s\a \\
  \obj\Cp{\b\app} = \App\s\ap \\
  \classoff\Cp\K = \b\mt \\
  \classoff\D\K = \b\mtp \\  
  \names{\b\mtp} \subseteq \names{\b\mt}\\
  \Kp = \K~\classgen{\Cp,\b\mt,\b\mtp,\E,\K} \\
  \sp = \Map\s{\Bind\a{\E\{\ap\}}}
}{
  \spec \a\S\s\K = \sp~\Kp
}

\IRule{CMS2}{
  \spec \a\S\s\K = \sp\\
  \spec {\b\a}\S\sp\K =\spp
}{
   \spec {\a\,\b\a}\S\s\K = \spp
}
\end{mathpar}

\subsection{Meet function}\label{monmeet}

The \texttt{mmeet} function is used by the \texttt{tmeet} functions to
perform the meet over the typing of each method within a class definition.
The \texttt{mmeet} function also takes four arguments, the method
signatures of the original class $\b\mt$, the method signatures of the cast
class $\b\mtp$, the environment $\P$, a class table $\K$, and outputs method
types $\b\mtpp$ and a class table $\Kp$. \\

\hrulefill

\opdef{
  $\mmeet{\b\mt}{\b\mtp}\P\K = \b\mtpp\,\Kp$
}{
}
\begin{mathpar}
\IRule{MM1}{
}{
  \mmeet{\b\mt}{\cdot}\P\K =\b{\mt} ~\K
}

\IRule{MM2}{
}{
  \mmeet{\cdot}{\b\mt}\P\K =\b{\mt} ~\K
}

\IRule{MM3}{ 
  \Mtype\f{}{\t} = \mt \\
  \Mtype\f{}{\tp} \in \b{\mtp} \\
  \tmeet{\t}{\tp}\P\K = \tpp~\Kp \\
  \Mtype\f{}{\tpp} = \mtpp
}{ 
   \mmeet{\mt}{\b{\mtp}}\P\K = \mtpp\,\Kp
}

\IRule{MM4}{ 
  \Mtype\f{\t}{\t} = \mt \\
  \Mtype\f{\tp}{\tp} \in \b{\mtp} \\
  \tmeet{\t}{\tp}\P\K = \tpp~\Kp \\
  \Mtype\f{\tpp}{\tpp} = \mtpp
}{ 
   \mmeet{\mt}{\b{\mtp}}\P\K = \mtpp\,\Kp
}


\IRule{MM5}{ 
  \Mtype\m{\t_1}{\t_2} = \mt \\
  \Mtype\m{\t_3}{\t_4} \in \b{\mtp} \\
  \tmeet{\t_3}{\t_1}\P\K = \t_5~\Kp \\
  \tmeet{\t_2}{\t_4}\P\Kp = {\t_6}~{\Kpp} \\
  \Mtype\n{\t_5}{\t_6} = \mtpp
}{ 
   \mmeet{\mt}{\b{\mtp}}\P\K = \mtpp\,\Kpp 
}

\IRule{MM6}{
  \mmeet{\mt}{\b{\mt_2}}\P\K = \mt_3~\Kp\\
  \mmeet{\b{\mt_1}}{\b{\mt_2}}\P\Kp = \b{\mt_4}~\Kpp
}{
  \mmeet{\mt~\b{\mt_1}}{\b{\mt_2}}\P\K =\mt_3\b{\mt_4} ~\Kpp
}
\end{mathpar}
\\

\subsection{Monotonic dynamic expression translation}\label{montrans}

\opdef{$\rtranstz{\b{\mt}}{\b{\mtp}}\e\ep$}{Monotonic dynamic expression translation}

\begin{mathpar}
\IRule{MREW1}{ }{ \rtranstz{\b{\mt}}{\b{\mtp}}\x\x }
\\
\IRule[width=25em]{MREW2}{ 
  \Mtype\m{{\t_1}}{\tp_1} \in \b{\mt} \\ 
  \Mtype\m{{\C_2}}{\Cp_2} \in \b{\mtp} \\ 
  {\rtranstz{\b{\mt}}{\b{\mtp}}{\e}{\ep}}
}{
  \rtranstz{\b{\mt}}{\b{\mtp}}{\KCall{\this}\m{\e}{\t[1]}{\tp[1]}}{\MonCast{\tp_1}{\KCall{\this}\m{{\MonCast{\C_2}{\ep}}}{\C_2}{\Cp_2}}}
}
\\
\IRule[width=25em]{MREW3}{ 
  \Mtype\f{\b{\t_1}}{\tp_1} \in \b{\mt} \\ 
  \Mtype\f{\b{\t_2}}{\tp_2} \in \b{\mtp} \\ 
  \b{\rtranstz{\b{\mt}}{\b{\mtp}}{\e}{\ep}}
}{
  \rtranstz{\b{\mt}}{\b{\mtp}}{\Call{\this}\f{\b\e}}{\MonCast{\tp_1}{\Call{\this}\f{\b{\MonCast{\t_2}{\ep}}}}}
}
\\

\IRule[width=20em]{MREW4}{ 
  \Mtype\m{{\t_1}}{\tp_1} \in \b{\mt} \\ 
  \Mtype\m{{\t_2}}{\tp_2} \not\in \b{\mtp} \\ 
  \b{\rtranstz{\b{\mt}}{\b{\mtp}}{\e}{\ep}}
}{
  \rtranstz{\b{\mt}}{\b{\mtp}}{\KCall{\this}\m{\e}{{\t_1}}{\tp_1}}{{\KCall{\this}\m{{{\ep}}}{{\t_1}}{\tp_1}}}
}

\IRule[width=20em]{MREW5}{ 
  \Mtype\f{\b{\t_1}}{\tp_1} \in \b{\mt} \\ 
  \Mtype\f{\b{\t_2}}{\tp_2} \not\in \b{\mtp} \\ 
  \b{\rtranstz{\b{\mt}}{\b{\mtp}}{\e}{\ep}}
}{
  \rtranstz{\b{\mt}}{\b{\mtp}}{\Call{\this}\f{\b\e}}{{\Call{\this}\f{\b{{\ep}}}}}
}

\IRule[width=20em]{MREW6}{ 
  \e_1 \neq \this \\ \rtranstz{\b{\mt}}{\b{\mtp}}{\e_1}{\e_2} \\ 
  \b{\rtranstz{\b{\mt}}{\b{\mtp}}{\ep_1}{\ep_2}}
}{
  \rtranstz{\b{\mt}}{\b{\mtp}}{\Call{\e_1}\f{\b{\ep_1}}}{\Call{\e_2}\f{\b{\ep_2}}}
}

\IRule[width=20em]{MREW7}{ 
  \e_1 \neq \this \\ \rtranstz{\b{\mt}}{\b{\mtp}}{\e_1}{\e_2} \\ 
  \b{\rtranstz{\b{\mt}}{\b{\mtp}}{\ep_1}{\ep_2}}
}{
  \rtranstz{\b{\mt}}{\b{\mtp}}{\KCall{\e_1}\m{{\ep_1}}{\t_1}{\tp[1]}}{\KCall{\e_2}\m{{\ep_2}}{\t_1}{\tp[1]}}
}

\IRule{MREW8}{ 
  \rtranstz{\b{\mt}}{\b{\mtp}}{\e_1}{\e_2} \\   
  \rtranstz{\b{\mt}}{\b{\mtp}}{\ep_1}{\ep_2}
}{
  \rtranstz{\b{\mt}}{\b{\mtp}}{\DynCall{\e_1}\m{\ep_1}}{\DynCall{\e_2}\m{\ep_2}}
}

\IRule{MREW9}{
  \b{\rtranstz{\b{\mt}}{\b{\mtp}}{\e}{\ep}} 
}{
  \rtranstz{\b\mt}{\b\mtp}{\New\C{\b\e}}{\New\C{\b\ep}}
}
\end{mathpar}


\subsection{Monotonic class generation}\label{classgen}

The \xt{monWrap} function generates the wrapper for the monotonic translations,
when a monotonic cast is required.

\footnotesize
\[\begin{array}{l@{~}r@{}r@{~}r@{~}r@{~}rr}
\arrayrulecolor{white}
\classgen{\C, \b\md, \bmt, \bmtp, \D, \K}= \\
\SP \class~\D~\{ \\
\SPP \Fdef\that\C
\\[1mm]
\SPP \Mdef\f\x\any\any {\SubCast\any{\MonCast\tp{
      \Set{\Get\this\that}\f{\MonCast\t{\MonCast\tp\x}}}}}
&
\All f \Mtype\f\t\t\in\bmt &\wedge& \Mtype\f\tp\tp\in\bmtp
\\[1mm]\hline
\SPP \Mdefz\f\any{\SubCast\any{\MonCast\tp{\Get{\Get\this\that}\f}}}
&
 \All f \Mtype\f{}\t \in \bmt &\wedge& \Mtype\f{}\tp \in \bmtp
\\[1mm]\hline
\SPP \Mdef\m\x\any\any {~\SubCast\any{\MonCast{\tp[2]}{{[{(\MonCast{\t[1]}\x)}/\x]\ep}}}~}
&     \All \m \Mdef\m\x{\t[1]}{\tp[1]}\e\in\b\md &\wedge& \Mtype\m{\t[2]}{\tp[2]}\in\bmtp \\
&&\wedge&\multicolumn{3}{l}{\rtranstz{\bmt}\bmtp{[{(\MonCast\t\x)}/\x]\e}{\ep}}
\\[1mm]\hline
\SPP \Mdef\m\x{\t[2]}{\tp[2]} {\MonCast{\tp[2]}{[{(\MonCast{\t[1]}\x)}/\x]\ep}~}
&     \All \m \Mdef\m\x{\t[1]}{\tp[1]}\e\in\b\md &\wedge& \Mtype\m{\t[2]}{\tp[2]}\in\bmtp \\
&&\wedge&\multicolumn{3}{l}{\rtranstz{\bmt}\bmtp{\e}{\ep}} \\
&&\wedge&\multicolumn{3}{l}{\statictype{\t[2]}{\K}{\cdot}}
\\[1mm]\hline
\SP\}
\end{array}
\]
\normalsize





\subsection{Monotonic equivalent type generation}\label{typegen}

The \xt{typegen} function is used by the \xt{tmeet} function to generate the new classes
of the type produced by the meet operation.

\footnotesize
\[\begin{array}{l@{~}r@{}r@{~}r@{~}r@{~}rr}
\arrayrulecolor{white}
\typegen{\bmt}{\D} = \\
\SP \class~\D~\{
\\[1mm]
\SPP \Mdef\m\x\t\tp {{\MonCast\tp{\x}}} 
&
\All m \Mtype\m\t\tp\in\bmt
\\[1mm]\hline
\SPP \Mdef\f\x\t\t {\x}
&
\All f \Mtype\f\t\t\in\bmt
\\[1mm]\hline
\SPP \Mdefz\f\t{{\MonCast\t{\New\D{}}}}
&
 \All f \Mtype\f{}\t \in \bmt
\\[1mm]
\SP\}
\end{array}
\]
\normalsize




\subsection{Htype function}

The function \htype\a\S\s\K
returns the class of the object at address \a.  \htype\a\S\s\K is \C if \C =
\App\S\a or if $\a\not\in\S$ and \obj\C{\b\a}=\App\s\a.

\begin{mathpar}
\IRule{HT1}{
  \a \not\in \text{addr}(\S) \\
  \enfortype\C\cdot\K = \D\,\W\,\Kp
}{
  \htype\a\S{\sigma[\a \mapsto \C\{\ap\}]}\K = \D~\Kp
}

\IRule{HT2}{
  \S(\a) = \t
}{
  \htype{\a}{\S}{\sigma}\K = \t~\K
}
\end{mathpar}

\subsection{Lifting function}

\hrulefill
\begin{align*}
\W &::= \cdot \B \Map\W{\Bind\C\D}
\end{align*}
\begin{mathpar}
\IRule{ENT1}{ 
  \Class\C{\hspace{-0.3em}}{\Fdef\that\Cp ~ \b\md} \in \K \\ 
  \C \notin \text{dom}(\W) \\
  \D\text{ fresh} \\
  \Wp = \W\,\Bind\C\D \\
  \Kp = \K\,\Class\D{\hspace{-0.3em}}{\Fdef\that\Cp ~ \enformt{\b{\md}}{\Wp}{\K}} \\ 
}{
  \enfortype{\C}{\W}{\K} = \D\,\Wp\,\Kp
}

\IRule{ENT2}{ 
  \W(\C) = \D \\
}{
  \enfortype{\C}{\W}{\K} = \D\,\W\,\K
}

\IRule{ENT3}{ 
}{
  \enfortype{\any}{\W}{\K} = \any\,\W\,\K
}
\end{mathpar}
\\

\hrulefill

\begin{mathpar}
\IRule{ENMT1}{
  \md =  \Mdef\m\x\any\any {~\SubCast\any{\MonCast\tp{{[{(\MonCast\t\x)}/\x]\ep}}}~} \\
  \enfortype\t\W\K = \tpp\,\Wp\,\Kp \\
  \enfortype\tp\Wp\Kp = \tppp\,\Wpp\Kpp \\
  \mdpp = \Mdef\m\x\tpp\tppp {\MonCast\tppp\x} \\
  \enformt{\b\md}{\Wpp}{\Kpp} = \b\mddp \\
}{
  \enformt{\md\,\b\md}{\W}{\K} = \mdpp\,\b\mddp
}

\IRule{ENMT2}{
  \md = \Mdef\m\x\t\tp {\ep~}\\ 
  \enfortype\t\W\K = \tpp\,\Wp\,\Kp \\
  \enfortype\tp\Wp\Kp = \tppp\,\Wpp\Kpp \\  
  \mdpp =\Mdef\m\x\tpp\tppp {\MonCast\tppp\x}\\
  \enformt{\b\md}{\Wpp}{\Kpp} = \b\mddp \\
}{
  \enformt{\md\,\b\md}{\W}{\K} = \mdpp\,\b\mddp
}

\IRule{ENFS}{
  \md = \Mdef\f\x\any\any {\SubCast\any{\MonCast\tp{\Set{\Get\this\that}\f{\MonCast\t{\MonCast\tp\x}}}}} \\ 
  \enfortype\t\W\K = \tpp\,\Wp\,\Kp \\
  \enfortype\tp\Wp\Kp = \tppp\,\Wpp\Kpp \\   
  \mdpp = \Mdef\f\x\tpp\tppp {\MonCast\tppp\x} \\
  \enformt{\b\md}{\Wpp}{\Kpp} = \b\mddp \\
}{
  \enformt{\md\,\b\md}{\W}{\K} = \mdpp\,\b\mddp
}

\IRule{ENFG}{
  \md =  \Mdefz\f\any{\SubCast\any{\MonCast\t{\Get{\Get\this\that}\f}}} \\ 
  \enfortype\t\W\K = \tp\,\Wp\,\Kp \\
  \mdpp = \Mdefz\f\tp {\MonCast\tp{\Get{\Get\this\that}\f}} \\
  \enformt{\b\md}{\Wp}{\Kp} = \b\mddp \\
}{
  \enformt{\md\,\b\md}{\W}{\K} = \mdpp\,\b\mddp
}

\IRule{ENE}{
}{
  \enformt{\cdot}{\W}{\K} = \cdot
}
\end{mathpar}

\bibliographystyle{ACM-Reference-Format}
\bibliography{sample-bibliography}

\end{document}
