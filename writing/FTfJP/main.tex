\documentclass[sigconf]{acmart}

\usepackage{xspace,listings,url,framed,amssymb,colortbl,
            amsmath,mathpartir,hyperref,doi, rotating,
            stmaryrd, graphicx, tikz, colortbl, xparse, etoolbox, pgffor,booktabs} % double brackets llbracket
%% Formatting
\newcommand{\M}[1]{\ensuremath{#1}\xspace}
\newcommand{\xt}[1]{{\sf{#1}}}
\newcommand{\bt}[1]{\xt{\bf #1}}
\renewcommand{\b}[1]{\M{\overline{#1}}}
\newcommand{\Mxt}[1]{\M{\xt{#1}}}
\newcommand{\Mbt}[1]{\M{\bt{#1}}}

%% Variables
\newcommand{\x}   {\Mxt x}
\newcommand{\e}   {\Mxt e}
\newcommand{\m}   {\Mxt m}
\newcommand{\s}   {\M{\sigma}}
\renewcommand{\t} {\Mxt t}
\renewcommand{\a} {\Mxt a}
\newcommand{\K}   {\Mxt K}
\renewcommand{\k} {\Mxt k}
\newcommand{\Kp}  {{\Mxt{K'}}}
\newcommand{\ep}  {{{\Mxt{e'}}}}
\renewcommand{\sp}{{{\M{\s'}}}}
\newcommand{\ap}  {\M{\a'}}
\newcommand{\tp}  {\M{ \t'}}
\newcommand{\C}   {\Mxt C}
\newcommand{\Cp}  {\Mxt{C'}}
\newcommand{\fd}  {\Mxt{fd}}
\newcommand{\md}  {\Mxt{md}}
\newcommand{\mt}  {\Mxt{mt}}
\newcommand{\f}   {\Mxt f}
\newcommand{\E}   {\M \Gamma}
\newcommand{\any} {\M{\star}}
\newcommand{\this}{\Mxt{this}}
\newcommand{\none}{\M{\cdot}}
\newcommand{\D}   {\Mxt D}

\newcommand{\Get}[2]   {\M{#1.#2()}}
\newcommand{\Set}[3]   {\M{#1.#2(#3)}}
\newcommand{\Call}[3]  {\M{#1.#2(#3)}}

\newcommand{\DynGet}[2]   {\M{#1@#2()}}
\newcommand{\DynSet}[3]   {\M{#1@#2(#3)}}
\newcommand{\DynCall}[3]  {\M{#1@#2(#3)}}

\newcommand{\New}[2]   {\M{\new\;#1(#2)}}
\newcommand{\wCast}[2] {\M{<{#1}>\;{#2}}}
\newcommand{\mCast}[2] {\M{\prec #1 \succ #2}}
\newcommand{\tCast}[2] {\M{\triangleleft\; #1 \triangleright #2}}
\newcommand{\cCast}[2] {\M{\blacktriangleleft #1 \blacktriangleright #2}}
\newcommand{\new}      {\M{\bt{new}}}
\newcommand{\HT}[2]    {\M{{#1}\!:{#2}}}
\newcommand{\Mdef}[5]  {\M{ \HT{ #1( \HT{#2}{#3})}{#4}~\{\,{#5}\,\}}}
\newcommand{\Mdefz}[3] {\M{ \HT{ #1()}{#2}~\{\,{#3}\,\}}}
\newcommand{\Ctx}[1]   {\M{\xt{E}[#1]}}
\newcommand{\Obj}[2]   { \M{ #2\{#1\}}}
\newcommand{\obj}[2]   { \M{ #1\{#2\}}}
\newcommand{\alloc}[4] {\M{#1\;#2  = \xt{alloc}(#3, #4)}}
\newcommand{\cast}[8]  {\M{#6\;#7\;#8=\xtns{#5 cast}(#1, #2, #3, #4)}}
\newcommand{\Alt}[1]   { &\B #1 \\}
\newcommand{\B}        {\M{~|~}}
\newcommand{\bang}     {\M{\xt{!}}}

\newcommand{\dispatch}[5] {\M{#1\;#2 = \xt{disp}(#3,#4,#5)}}
\newcommand{\readfield}[4]{\M{#1 = \xt{read}(#2,#3,#4)}}
\newcommand{\setfield}[5] {\M{#1 = \xt{write}(#2,#3,#4,#5)}}
\newcommand{\Reduce}[6]   {\M{{#1}~{#2}~{#3} \rightarrow {#4}~{#5}~{#6}}}
\newcommand{\ReduceA}[6]  {\M{#1~#2~#3 } & \M{\rightarrow #4~#5~#6}}
\newcommand{\Class}[3]    {\M{\bt{class}\;#1\,\{\,#2~#3\,\}}}
\newcommand{\Ftype}[2]    {\M{ \HT{#1}{#2} }}
\newcommand{\Mtype}[3]    {\M{ \HT{#1(#2)}{#3}}}
\newcommand{\Type}[1]     {\M{\{#1\}}}

\newcommand{\opdef}[2]    {\framebox[1.1\width]{#1} ~ #2\\}
\newcommand{\Heap}[2]     {\M{ #1[#2] }}
\newcommand{\Bind}[2]     {\M{#1 \mapsto #2}}

\newcommand{\Sub}{\M{<:}}
\newcommand{\ConsSub}{\M{\lesssim}}

\newcommand{\CondRule}[3]{ #3 &~{\emph{if}} #2 \\}
\newcommand{\EnvType}[3]{ \M{#1 \vdash #2 : #3}}
\newcommand{\Es}{\E ~\s}
\newcommand{\tw}[1]{\M{\xt{typed}(#1)}}
\newcommand{\IRule}[3]{\inferrule*[lab={\tiny #1}]{#2}{#3}}
\newcommand{\HasType}[3]{ \M{#1 (#2) = #3}}
\newcommand{\inc}{\M{\in}}
\newcommand{\wrapper}[1]{\M{\xt{wrap}(#1)}}
\newcommand{\xtns}[1]{{\sf{#1}}}
\newcommand{\utw}[1]{\M{\xt{untyped}(#1)}}
\newcommand{\meet}[2]{#1\sqcap #2}
\newcommand{\spec}[1]{\M{\xt{spec}(#1)}}

\newcommand{\castfn}[4]{\text{cast}(#1,#2,#3,#4)}
\newcommand{\GenCast}[4]{#1 \vdash #2 \hookrightarrow #3 \Uparrow #4 }
\newcommand{\AnaCast}[4]{#1 \vdash #2 \Downarrow #4 \hookrightarrow #3}
\newcommand{\TransClass}[2]{\M{ #1 \rightharpoonup #2 }}
\newcommand{\inv}[2]{\xt{invoke}(#1, #2)}

\newcommand{\III}{}
\III{

\newcommand{\n}{\M{\xt{n}}}
\renewcommand{\d}{\M{\xt{d}}}
\renewcommand{\r}{\M{\xt{r}}}
\newcommand{\fb}{\M{\xt{f!}}}
\renewcommand{\c}{\M{\xt{c}}}
\renewcommand{\d}{\M{\xt{d}}}
\newcommand{\T}{\M{\xt T}}
\newcommand{\cbp}[1]{\M{ \b{\C_{#1}}}}
\renewcommand{\mp}[1]{\M{ \m_{#1} }}
\newcommand{\class}{\M{\bt{class}}}
\newcommand{\SMdef}[5]{\M{ \HT { #1!( \HT{#2}{#3})}{#4}= {#5}}}
\newcommand{\GMdef}[3]{\M{ \HT { #1()}{#2}={#3}}}
\newcommand{\Fdef}[3]{\M{ \HT{#1}{#2}={#3} }}
\newcommand{\notdispatch}[5]{\M{#1,#2 \not = \xt{dispatch}(#3,#4,#5)}}
\newcommand{\NoCondRule}[2]{ #2 &       \\}
\newcommand{\Update}[3]{\M{#1[ #2 := #3]}}
\newcommand{\NotSub}{\M{\not<:}}
\newcommand{\classofis}[2]{\M{\xt{classof}(#1)=#2}}
\newcommand{\typeofis}[3]{\M{\xt{typeof}(#1,#2)=#3}}
\newcommand{\classof}[1]{\M{\xt{classof}(#1)}}
\newcommand{\typeof}[2]{\M{\xt{typeof}(#1,#2)}}
\newcommand{\Sel}[2]{\M{#1(#2)}}
\newcommand{\TransExp}[4]{\M{ #1 \vdash #2 \hookrightarrow #3 : #4 }}
\newcommand{\mdp}[1]{\M{\md_{#1}}}
\newcommand{\setter}[1]{\M{\xt{set}(#1)}}
\newcommand{\getter}[1]{\M{\xt{get}(#1)}}
\newcommand{\dynamic}[1]{\M{\xt{dyn}(#1)}}
\newcommand{\invoke}[1]{\M{\xt{inv}(#1)}}
\newcommand{\Dyn}[1]{\M{#1^{\any} }}
\newcommand{\casts}[1]{\M{\xt{casts}(#1)}}
\newcommand{\fcast}[1]{\M{\xt{translate}(#1)}}
\newcommand{\proxy}[1]{\M{\xt{proxy}(#1)}}
%\newtheorem{definition}{Definition}
%\newtheorem{thm}{Theorem}
\newcommand{\WF}[1]{\ensuremath{\xt{WF}(#1)}\xspace}
\newcommand{\Weak}{\text{\small?}\hspace{0.1em}}
\newcommand{\WType}[1]{\Weak\Type{#1}}


\newcommand{\stcons}[2]{#1\lesssim #2}
\newcommand{\refine}[2]{#1\vartriangleright #2}
\newcommand{\inscast}[1]{\M{\llbracket #1 \rrbracket_{\xt{ins}}}}
\newcommand{\adapt}[1]{\M{\llbracket #1 \rrbracket_{\xt{adapt}}}}

\newcommand{\p}{\xt{p}}
\newcommand{\cb}{\b{\C}}
\newcommand{\typed}[1]{\text{typed}(#1)}
\newcommand{\untyped}[1]{\text{untyped}(#1)}
\newcommand{\includecode}[2][c]{\lstinputlisting[caption=#2, escapechar=, style=custom#1]{#2}<!---->}

\newcommand{\ESub}[3]{#1 \vdash #2 \leq #3}
\newcommand{\Vect}[3]{\text{Vec}(#1,#2,#3)}
\newcommand{\tv}{\xt{v}}
\newcommand{\ts}{\xt{ts}}
\newcommand{\MetaSub}[2]{\xtns{Sub}(#1,#2)}

}


% Copyright
%\setcopyright{none}
%\setcopyright{acmcopyright}
%\setcopyright{acmlicensed}
\setcopyright{rightsretained}
%\setcopyright{usgov}
%\setcopyright{usgovmixed}
%\setcopyright{cagov}
%\setcopyright{cagovmixed}


% DOI
\acmDOI{10.475/123_4}

% ISBN
\acmISBN{123-4567-24-567/08/06}

%Conference
\acmConference[WOODSTOCK'97]{ACM Woodstock conference}{July 1997}{El
  Paso, Texas USA}
\acmYear{1997}
\copyrightyear{2016}
\acmArticle{4}
\acmPrice{15.00}


\begin{document}
\title{Monotonic Gradual Typing in a Common Calculus}
\subtitle{}

\author{Ben Trovato}
\authornote{Dr.~Trovato insisted his name be first.}
\orcid{1234-5678-9012}
\affiliation{%
  \institution{Institute for Clarity in Documentation}
  \streetaddress{P.O. Box 1212}
  \city{Dublin}
  \state{Ohio}
  \postcode{43017-6221}
}
\email{trovato@corporation.com}

\author{John Smith}
\affiliation{\institution{The Th{\o}rv{\"a}ld Group}}
\email{jsmith@affiliation.org}

% The default list of authors is too long for headers.
\renewcommand{\shortauthors}{B. Trovato et al.}


\begin{abstract}
This paper provides a sample of a \LaTeX\ document which conforms,
somewhat loosely, to the formatting guidelines for
ACM SIG Proceedings.\footnote{This is an abstract footnote}
\end{abstract}

%
% The code below should be generated by the tool at
% http://dl.acm.org/ccs.cfm
% Please copy and paste the code instead of the example below.
%
\begin{CCSXML}
<ccs2012>
 <concept>
  <concept_id>10010520.10010553.10010562</concept_id>
  <concept_desc>Computer systems organization~Embedded systems</concept_desc>
  <concept_significance>500</concept_significance>
 </concept>
 <concept>
  <concept_id>10010520.10010575.10010755</concept_id>
  <concept_desc>Computer systems organization~Redundancy</concept_desc>
  <concept_significance>300</concept_significance>
 </concept>
 <concept>
  <concept_id>10010520.10010553.10010554</concept_id>
  <concept_desc>Computer systems organization~Robotics</concept_desc>
  <concept_significance>100</concept_significance>
 </concept>
 <concept>
  <concept_id>10003033.10003083.10003095</concept_id>
  <concept_desc>Networks~Network reliability</concept_desc>
  <concept_significance>100</concept_significance>
 </concept>
</ccs2012>
\end{CCSXML}

\ccsdesc[500]{Computer systems organization~Embedded systems}
\ccsdesc[300]{Computer systems organization~Redundancy}
\ccsdesc{Computer systems organization~Robotics}
\ccsdesc[100]{Networks~Network reliability}


\keywords{ACM proceedings, \LaTeX, text tagging}


\maketitle


\section{Complete translation for Monotonic}

\subsection{Monotonic cast static and dynamic rules}

\begin{minipage}{0.35\textwidth}
\begin{mathpar}
\IRule{W10}{
  \EnvType \Env\s\K\e\tp
}{
  \EnvType \Env\s\K{\MonCast\t\e}\t
}
\end{mathpar}
\end{minipage}
\begin{minipage}{0.5\textwidth}
\begin{tabular}{l@{}l@{~}l@{~}l}
\CondRule{E11}{  %Monotonic cast  
  \moncast \a\t\s\K  \Kp\ap\sp    
}{    
  \ReduceA  \K{\MonCast \t\a}\s \Kp\ap\sp   
} \\
\multicolumn{4}{l}{\EE ::= \ldots \B \MonCast\t\EE }
\end{tabular}
\end{minipage}
\subsection{Monotonic translation for bidirectional expressions}

\begin{align*}
\TRG[M]{\x}\Env = \x & \\
\TRG[M]{\Call{\e_1}\m{{\e_2}}}\Env = \DynCall{\ep_1}\m{\ep_2} & \tag{\IF\EM{\TypeCk{\K,\Env}\e\any
\AND \TRG[M]{\e_1}\Env = \ep_1 \wedge \TAG[M]{\e_2}\Env{\any} = \ep_2}}\\
\TRG[M]{\Call{\e_1}\m{{\e_2}}}\Env = \KCall{\ep_1}\m{\ep_2}{\D[1]}{\D[2]} & \tag{\IF\EM{\TypeCk{\K,\Env}\e\C
\AND \Mtype\m{\D[1]}{\D[2]}\In\App\K\C \AND \TRG[M]{\e_1}\Env = \ep_1 \wedge \TAG[M]{\e_2}\Env{\D[1]} = \ep_2}}\\
\TRG[M]{\New\C{{\e_1}...}}\Env = \New\C{\ep_1...} & \tag{\IF\EM{\Ftype{\f[1]}{\t[1]}\In\App\K\C
    \AND \ep[1] = \TAG[M]{\e[1]}\Env{\t[1]} ~..}}\\
\TAG[M]{\e}{\Env}\t = \e' & \tag{\IF\EM{\TypeCk{\K,\Env}\e\tp \AND \K\vdash\tp\Sub\t}}\\
\TAG[M]{\e}{\Env}\t = \MonCast{\t}{\ep} & \tag{\IF\EM{\TypeCk{\K,\Env}\e\tp \AND \K\vdash\tp\not\Sub\t}}
\end{align*}

\section{Generative Monotone Casts}

\begin{mathpar}
\IRule{CM}{
  \s(\a) = \C\{\ap\ldots\} \\
  \tmeet \a\C\t\s\K = \Cp~\Kp\\
  \sp = \s[\a \mapsto \Cp\{\ap\ldots\}]
}{
  \moncast \a\t\s\K \Kp \sp\\
}
\end{mathpar}
{Monotonic cast semantics}


\opdef{
  $\tmeet{\t}{\tp}\P\K = \tpp\,\Kp$
}{
}
\begin{align*}
\P &::= \cdot \B \P ~{{(\C,\D)}\mapsto\Cp}
\end{align*}
\begin{mathpar}
\IRule{TM1}{ }{\tmeet\C\any\P\K = \C\,\K}

\IRule{TM2}{ }{\tmeet\any\C\P\K = \C\,\K}

\IRule{TM3}{ }{\tmeet\t\t\P\K = \t\,\K}

\IRule{TM4}{
  \fresh\Cp\\
  (\C,\D) \not\in\P \\
  \Pp = \P~{{(\C,\D)}\mapsto\Cp} \\
  \mtype\C\K = {\mt..}\\
  \mtype\D\K = {\mtp..}\\
  \mmeet{\mt..}{\mtp..}\Pp\K = \mtpp..\,\Kp\\
  \Kpp = \Kp~\typegen{\mtpp..}\Cp\\
}{
    \tmeet\C\D\P\K = \Cp\,\Kpp
}

\IRule{TM5}{
    \P(\C,\D) = \Cp
}{
    \tmeet\C\D\P\K = \Cp\,\K
}
\end{mathpar}
{The \texttt{tmeet} function}

\subsection{Meet function}\label{monmeet}

The \texttt{mmeet} function is used by the \texttt{tmeet} functions to
perform the meet over the typing of each method within a class definition.
The \texttt{mmeet} function also takes four arguments, the method
signatures of the original class $\mt..$, the method signatures of the cast
class $\mtp..$, the environment $\P$, a class table $\K$, and outputs method
types $\mtpp..$ and a class table $\Kp$. \\

\hrulefill
\begin{mathpar}
\IRule{MM1}{
}{
  \mmeet{\mt ..}{\cdot}\Env\K =\mt.. ~\K
}

\IRule{MM2}{
}{
  \mmeet{\cdot}{\mt ..}\Env\K ={\mt ..} ~\K
}

\IRule{MM6}{
  \Mtype\m{\t_3}{\t_4} \in {\mt_2} \\
  \tmeet{\t_3}{\t_1}\Env\K = \t_5~\Kp \\
  \tmeet{\t_2}{\t_4}\Env\Kp = {\t_6}~{\Kpp} \\
  \mmeet{{\mt_1 ..}}{{\mt_2 ..}}\Env\Kp = {\mt_3 ..}~\Kpp
}{
  \mmeet{\Mtype\m{\t_1}{\t_2}~{\mt_1 ..}}{{\mt_2 ..}}\Env\K =\Mtype\n{\t_5}{\t_6} ~{\mt_3 ..} ~\Kpp
}
\end{mathpar}

\subsection{Monotonic class generation}\label{classgen}

The \xt{monWrap} function generates the wrapper for the monotonic translations,
when a monotonic cast is required.

\footnotesize
\[\begin{array}{l@{~}r@{}r@{~}r@{~}r@{~}rr}
\arrayrulecolor{white}
\classgen{\C, \md.., \mt.., \mtp.., \D, \K}= \\
\SP \class~\D~\{ \\
\SPP \Fdef\f\t & \All {{\f}\!:{\t} \in \text{fields}(\K,\C)}
\\[1mm]\hline
\SPP \Mdef\m\x\any\any {~\MonCast\any{\KCall\this\m{\MonCast{\t[1]}\x}{\t[1]}{\t[2]}}~}
&     \All \m \Mdef\m\x{\any}{\any}\e\in\md.. &\wedge& \Mtype\m{\C_1}{\C_2}\in{\mtp..}
\\[1mm]\hline
\SPP \Mdef\m\x{\C_1}{\C_2} {\MonCast{\C_2}{[{(\MonCast{\any}\x)}/\x]\ep}~}
&     \All \m \Mdef\m\x{\any}{\any}\e\in\md.. &\wedge& \Mtype\m{\C_1}{\C_2}\in\mtp.. 
\\[1mm]\hline
\SP\}
\end{array}
\]
\normalsize





\subsection{Monotonic equivalent type generation}\label{typegen}

The \xt{typegen} function is used by the \xt{tmeet} function to generate the new classes
of the type produced by the meet operation.

\footnotesize
\[\begin{array}{l@{~}r@{}r@{~}r@{~}r@{~}rr}
\arrayrulecolor{white}
\typegen{\mt..}{\D} = \\
\SP \class~\D~\{
\\[1mm]
\SPP \Mdef\m\x\t\tp {{\MonCast\tp{\x}}} 
&
\All m \Mtype\m\t\tp\in\mt..
\\[1mm]
\SP\}
\end{array}
\]
\normalsize

\bibliographystyle{ACM-Reference-Format}
\bibliography{sample-bibliography}

\end{document}
