\documentclass[sigconf]{acmart}

\usepackage{xspace,listings,url,framed,amssymb,colortbl,
       amsmath,mathpartir,hyperref,doi, rotating, stmaryrd, 
       graphicx, tikz, colortbl, xparse, etoolbox, pgffor,booktabs} 
%% Formatting
\newcommand{\EM}[1]{\ensuremath{#1}\xspace}
\newcommand{\xt}[1]{{\sf{#1}}}
\newcommand{\bt}[1]{\xt{\bf #1}}
\renewcommand{\b}[1]{\EM{\overline{#1}}}
\newcommand{\EMxt}[1]{\EM{\xt{#1}}}
\newcommand{\EMbt}[1]{\EM{\bt{#1}}}

%% Variables
\newcommand{\x}   {\EMxt x}
\newcommand{\n}   {\EMxt n}
\newcommand{\e}   {\EMxt e}
\newcommand{\m}   {\EMxt m}
\newcommand{\s}   {\EM{\sigma}}
\renewcommand{\t} {\EMxt t}
\newcommand{\ta}  {\EM{\tau}}
\renewcommand{\a} {\EMxt a}
\newcommand{\K}   {\EMxt K}
\renewcommand{\k} {\EMxt k}
\newcommand{\Kp}  {{\EMxt{K'}}}
\newcommand{\Kpp}  {{\EMxt{K''}}}
\newcommand{\Kppp}  {{\EMxt{K'''}}}
\newcommand{\ep}  {{{\EMxt{e'}}}}
\newcommand{\epp}  {{{\EMxt{e''}}}}
\renewcommand{\sp}{{{\EM{\s'}}}}
\newcommand{\spp}{{{\EM{\s''}}}}
\newcommand{\ap}  {\EM{\a'}}
\newcommand{\aE}[1]  {\EM{\a_{#1}}}
\newcommand{\app}  {\EM{\a''}}
\newcommand{\tp}  {\EM{ \t'}}
\newcommand{\tpp}  {\EM{ \t''}}
\newcommand{\C}   {\EMxt C}
\newcommand{\Cp}  {\EMxt{C'}}
\newcommand{\EC}   {\EMxt E}
\newcommand{\fd}  {\EMxt{fd}}
\newcommand{\md}  {\EMxt{md}}
\newcommand{\mdpp}  {\EM{\md'}}
\newcommand{\mt}  {\EMxt{mt}}
\newcommand{\mtp}  {\EMxt{mt'}}
\newcommand{\mtpp}  {\EMxt{mt''}}
\newcommand{\M}{\EMxt M}
\newcommand{\MN}  {\EMxt{M\,K}}
\newcommand{\MNargs}[1]  {\EMxt {M #1~K}}
\newcommand{\f}   {\EMxt f}
\newcommand{\fp}   {\EMxt{ f'}}
\newcommand{\E}   {\EM{\Gamma}}
\newcommand{\EE}   {\EM{\mathcal{E}}}
\newcommand{\any} {\EM{\star}}
\newcommand{\this}{\EMxt{this}}
\newcommand{\that}{\EMxt{that}}
\newcommand{\none}{\EM{\cdot}}
\newcommand{\D}   {\EMxt D}
\newcommand{\Dp}   {\EMxt{D'}}
\newcommand{\p}   {\EMxt p}
\newcommand{\np}{\n'}

\newcommand{\Get}[2]   {\EM{#1.#2()}}
\newcommand{\Set}[3]   {\EM{#1.#2(#3)}}
\newcommand{\Call}[3]  {\EM{#1.#2(#3)}}
\newcommand{\DynCall}[3]  {\EM{#1@#2(#3)}}

\newcommand{\New}[2]   {\EM{\new\;#1(#2)}}
\newcommand{\SubCast}[2]{\EM{<\hspace{-.6mm}{#1}\hspace{-.6mm}>\hspace{-1mm}\;{#2}}}
\newcommand{\ShaCast}[2]{\EM{\prec #1 \succ #2}}
\newcommand{\MonCast}[2]{\EM{\triangleleft\; #1 \triangleright #2}}
\newcommand{\BehCast}[2]{\EM{\blacktriangleleft #1 \blacktriangleright #2}}
\newcommand{\new}      {\EM{\bt{new}}}
\newcommand{\HT}[2]    {\EM{{#1}\!:{#2}}}
\newcommand{\Mdef}[5]  {\EM{ \HT{ #1( \HT{#2}{#3})}{#4}\;\{{#5}\}}}
\newcommand{\Mdefz}[3] {\EM{ \HT{ #1()}{#2}\;\{{#3}\}}}
\newcommand{\Mdefa}[4]  {\EM{ \HT{ #1( #2 )}{#3}~\{{#4}\}}}
\newcommand{\obj}[2]   { \EM{ #1\{#2\}}}
\newcommand{\alloc}[4] {\EM{#1\;#2  = \xt{alloc}(#3, #4)}}
\newcommand{\cast}[8]  {\EM{#6\;#7\;#8=\xt{#5 cast}(#1, #2, #3, #4)}}
\newcommand{\behcast}[7]  {\EM{\xt{behcast}(#1, #2, #3, #4)=#5\,#6\,#7}}
\newcommand{\moncast}[6]  {\EM{\xt{moncast}(#1, #2, #3, #4)=#5\,#6}}

\newcommand{\Alt}[1]   { &\B #1 \\}
\newcommand{\B}        {\EM{~|~}}
\newcommand{\bang}     {\EM{\xt{!}}}

\newcommand{\dispatch}[5] {\EM{#1\;#2 = \xt{disp}(#3,#4,#5)}}
\newcommand{\readf}[4]{\EM{\xt{read}(#1,#2,#3,#4)}}
\newcommand{\convert}[1]{\EM{\xt{cnvtMD}(#1)}}
\newcommand{\convertFD}[1]{\EM{\xt{cnvtFD}(#1)}}
\newcommand{\readfield}[4]{\EM{#1 = \xt{read}(#2,#3,#4)}}
\newcommand{\setf}[5] {\EM{\xt{write}(#1,#2,#3,#4,#5)}}
\newcommand{\Reduce}[6]   {\EM{{#1}~{#2}~{#3} \rightarrow {#4}~{#5}~{#6}}}
\newcommand{\ReduceA}[6]  {\EM{#1~#2~#3 } & \EM{\rightarrow #4~#5~#6}}
\newcommand{\Class}[3]    {\EM{\bt{class}\;#1\,\{\,#2~#3\,\}}}
\newcommand{\Ftype}[2]    {\EM{ \HT{#1}{#2} }}
\newcommand{\Fdef}[2]    {\EM{ \HT{#1}{#2} }}
\newcommand{\Mtype}[3]    {\EM{ \HT{#1(#2)}{#3}}}
\newcommand{\Type}[1]     {\EM{\{#1\}}}

\newcommand{\opdef}[2]    {\framebox[1.1\width]{#1} ~ #2\\}
\newcommand{\Map}[2]     {\EM{ #1[#2] }}
\newcommand{\Bind}[2]     {\EM{#1 \mapsto #2}}

\newcommand{\Sub}{\EM{<:}}
\newcommand{\OK}{\EM{~\checkmark}}
\newcommand{\SubE}[1]{\EM{<:_{#1}}}
\newcommand{\names}[1]{\EM{\xt{names}(#1)}}
\newcommand{\untyped}[1]{\EM{\xt{untyped}(#1)}}

\newcommand{\mnames}[1]{\EM{\xt{methName}(#1)}}
\newcommand{\fnames}[1]{\EM{\xt{fieldName}(#1)}}


\newcommand{\ConsSub}{\EM{\lesssim}}

\newcommand{\CondRule}[3]{ #3 &~ #2 \\}
\newcommand{\SuchRule}[3]{ #3 &~{\emph{s.t.}} #2 \\}
\newcommand{\EnvType}[5]{ \EM{#1\,#2\,#3\vdash #4 : #5}}

\newcommand{\IRule}[4][]{\inferrule*[lab={\tiny #2},#1]{#3}{#4}}
\newcommand{\HasType}[3]{ \EM{#1 (#2) = #3}}
\newcommand{\wrapper}[1]{\EM{\xt{wrap}(#1)}}
\newcommand{\spec}[4]{\EM{\xt{spec}(#1,#2,#3,#4)}}

\newcommand{\castfn}[4]{\text{cast}(#1,#2,#3,#4)}
\newcommand{\GenCast}[5]{#1~#2 \vdash #3 \hookrightarrow #4 \Uparrow #5 }
\newcommand{\AnaCast}[5]{#1~#2 \vdash #3 \Downarrow #5 \hookrightarrow #4}
\newcommand{\TransClass}[2]{\EM{ #1 \rightharpoonup #2 }}
\newcommand{\inv}[2]{\xt{invoke}(#1, #2)}
\newcommand{\classoff}[2]{\EM{\xt{mtypes}(#1,#2)}}
\newcommand{\classoffs}[3]{\EM{\xt{mtypes}(#1,#2,#3)}}
\newcommand{\mtype}[3]{\EM{\xt{mtype}(#1,#2,#3)}}
\newcommand{\wftype}[3]{\EM{\xt{wftype}(#1,#2,#3)}}

\newcommand{\field}[2]{\EM{\xt{field}(#1,#2)}}
\newcommand{\In}{\EM{\in}}

\newcommand{\T}{\EM{\xt T}}
\newcommand{\Cast}{Cast }
\newcommand{\fb}{\EM{\xt{f!}}}

\newcommand{\AND}{\EM{\wedge}}
\newcommand{\App}[2]{\EM{#1(#2)}}

\newcommand{\StrSub}[4]{\EM{#1~#2\vdash #3\Sub #4}}
\newcommand{\tmeet}[4]{\xt{tmeet}(#1,#2,#3,#4)}
\newcommand{\mmeet}[4]{\xt{mmeet}(#1,#2,#3,#4)}
\newcommand{\mtypes}[2]{\xt{mtypes}(#1,#2)}


\newcommand{\WFtype}[2]{\EM{#1\vdash#2 \OK}}
\newcommand{\WF}[4]{\EM{#1\,#2\,#3\vdash#4 \OK}}
\newcommand{\WFp}[3]{#1~#2~#3\OK}

\renewcommand{\P}{\EMxt P}
\newcommand{\Pp}{\EMxt{P'}}


\newcommand{\retype}[5]{\xt{retype}(#1,#2,#3,#4,#5)}
\newcommand{\htype}[3]{\EM{\xt{htype}(#1,#2,#3)}}
\newcommand{\ftypes}[4]{\xt{ftypes}(#1,#2,#3,#4)}
\newcommand{\typeof}[1]{\xt{typeOf}(#1)}
\newcommand{\classgen}[1]{\xt{classgen}(#1)}
\renewcommand{\S}{\EM{\tau}}
\newcommand{\Sp}{\EM{\tau'}}
\newcommand{\Spp}{\EM{\tau''}}
\newcommand{\EQ}{\EM{\equiv}}

\newcommand{\Dom}[1]{\EM{\xt{dom}(#1)}}
\newcommand{\fresh}[1]{\EM{#1~\xt{fresh}}}

\newcommand{\progtrans}[2]{#1 ~\hookrightarrow_p~ #2}
\newcommand{\classtrans}[3]{#1 \vdash #2 ~\hookrightarrow_c~ #3}
\newcommand{\methtrans}[4]{#1~#2 \vdash #3 ~\hookrightarrow_m~ #4}
\newcommand{\statictype}[2]{\xt{static}(#1,#2)}

\newcommand{\code}[1]{{\tt #1}\xspace}

% Copyright
%\setcopyright{none}
%\setcopyright{acmcopyright}
%\setcopyright{acmlicensed}
\setcopyright{rightsretained}
%\setcopyright{usgov}
%\setcopyright{usgovmixed}
%\setcopyright{cagov}
%\setcopyright{cagovmixed}

\lstdefinelanguage{JavaScript}{
  keywords={typeof,new,true,false,instanceof,catch,function,return,null, 
    catch, switch, var, if, in, while, do, else, case, break},
  keywordstyle=\color{darkgray},
  ndkeywords={class,def,interface,export,boolean,throw,extends,implements,import,this},
  ndkeywordstyle=\color{darkgray}\bfseries,
  identifierstyle=\color{black},
  sensitive=false,  comment=[l]{//},  morecomment=[s]{/*}{*/},
  commentstyle=\color{gray}\ttfamily,  stringstyle=\color{gray}\ttfamily,
  morestring=[b]',  morestring=[b]",
  backgroundcolor=\color{vlightgray},
  aboveskip=\medskipamount, %0em,
  belowskip=\smallskipamount, %0em
  escapeinside={(*@}{@*)}
}
\lstset{
  language=JavaScript,  extendedchars=true,  basicstyle=\footnotesize\ttfamily,
  showstringspaces=false,   showspaces=false,  numberstyle=\small,
  numbersep=9pt,  tabsize=2, breaklines=true,  showtabs=false, captionpos=b
}

% DOI
%\acmDOI{10.475/123_4}

% ISBN
%\acmISBN{123-4567-24-567/08/06}

%Conference
\acmConference[FTfJP]{FTfJP}{July 2018}{}
\acmYear{2018}
%\copyrightyear{2016}
%\acmArticle{4}
%\acmPrice{15.00}

\begin{document}
\title{Monotonic Gradual Typing in a Common Calculus}
\subtitle{}

\author{Benjamin Chung}
\affiliation{\institution{Northeastern University}}
\author{Jan Vitek}
\affiliation{\institution{Northeastern and CTU}}
\thanks{We would like to thank Celeste Hollenbeck for assistance with the writing of this paper.}


\begin{abstract} Gradual typing refers to the notion that programs can be
incrementally decorated with type annotations. Languages that support this
approach to software development allow for programs being in various states of
``typedness'' on a scale ranging from entirely untyped to fully statically
typed. Points in the middle of this typed-untyped scale create interactions
between typed and untyped code, which is where gradual type systems differ.
Each gradual type system comes with tradeoffs. Some systems provide strong
guarantees at the expense of vastly degraded performance; others do not impact
the running time of programs, but they do little to prevent type errors. This
paper looks at an intriguing point in the landscape of these systems: the
monotonic semantics. The monotonic semantics is designed to reduce the
overhead of typed field access through a different enforcement mechanism
compared to other gradual type systems. In our previous paper, \cite{us}, we
described four semantics for gradual typing. This paper uses the framework of
that companion paper to present and explore a formulation for the
\emph{monotonic} semantics.  In comparison to the others, the monotonic
semantics is designed to reduce the overhead of typed field access. We
translate a common gradually typed source language to a common statically
typed target language according to the monotonic semantics, allowing easy
comparison to other gradual type systems in our framework.  \end{abstract}

\definecolor{Gray}{gray}{0.9}

\definecolor{vlightgray}{gray}{0.93}

\maketitle

\section{Introduction}

Gradual typing refers to a family of approaches that add types to untyped
programs. These approaches each protect the boundaries between typed and untyped
code in different ways~\cite{SiekTaha06,tf-dls06}. Gradual typing allows
untyped code to pass values into typed code, and the dynamic values need to be
reconciled with the new static types dynamically. The pantheon of gradual type systems
arises from different mechanisms to perform this check of types of untyped values.

Object-oriented languages pose a particular problem for gradual typing. It is
easy to check if a base value has a type, but problems arise from higher-order
behavior (that is, programs that pass around behavior inside values). In a
functional language, this manifests when trying to assert an arrow type on a
lambda, whose return type cannot be checked at the cast site. Object-oriented
languages have this same problem --- trying to assert a typed method on an
object with an untyped version of the method --- but have additional complexity
in the form of subtyping.

\begin{figure}[h]
\begin{lstlisting}
class C { 
  f(x:*):* { x } 
  g(x:*):* { new C() }
}
class D { 
  f(x:D):D { x }
}

new D().f( <*> new C() )
\end{lstlisting}
\caption{A gradually typed program.}
\label{fig:simpleprog}
\end{figure}

Figure~\ref{fig:simpleprog} shows a simple example of a gradually typed
object-oriented program. Typed expressions in our language use a standard
Featherweight Java-like type system with structural subtyping; however, not
all expressions in this program are typed. In the surface language, the type
\any indicates that a variable is dynamic; thus, it can hold a value of any
class; and any method can be invoked, as long as it returns another instance
of \any for that variable to receive

The program uses dynamic behavior (a cast) to pass an instance of class \C as
one of type \D in the call to \f. \C has no guarantee to act like a \D, so
\D-ness must be enforced at runtime. However, deciding the best way to enforce
static types on untyped values is an open problem. The program shown here
could fail when the instance of \C is first passed, fail if \f is called on
\x, or not fail at all if it does not violate the guarantee enforced by the
gradual type system.

In a companion paper, to appear at ECOOP 2018~\cite{us}, we examined four
gradual typing approaches, each of which addresses this problem differently.
In that paper, we discussed the optional, transient, behavioral, and concrete
semantics. We chose these semantics due to their popularity and relative
simplicity; but we elided the monotonic semantics of~\cite{siek14}, due to its
complexity. This paper places the monotonic semantics into the same framework
as the companion paper.

The monotonic semantics sets out to solve a key problem for gradual typing:
speed of field access. For example, the behavioral semantics adds a wrapper
around every field access which checks the value's type, at the cost of
indirection. This can make typed code in a gradually typed language
excruciatingly slow~\cite{popl16}.  Suppose we have an object $\x = \{ \f:\any
= 3 \}$, then alias it as $\xp = \SubCast{\{ \f:\xt{int}   \}}\x$. Despite
\xp's static type, the field dereference $\xp.\f$ could return a value of any
type. If we run \xt{\x.\f = "hello"} at some point, then $\xp.\f$ is now ill-typed 
via the reference. As mentioned previously, other semantics deal with
this by checking the result of operations in typed code, lazily enforcing
soundness of type annotations.

The monotonic semantics is an \emph{eager} mechanism to enforce types,
checking values assigned to fields on assignment rather than dereference. Using
the monotonic semantics, once a value in the heap is typed, it will always remain of that
type. In the example above, once we aliased \x as \xp, \xp's type (e.g. that
\f has type \xt{int}) will be enforced upon usages of \x. Now, our assignment
\xt{\x.\f = "hello"} will fail, because we are assigning a value of improper
type to the field \f of the object. Monotonic uses this mechanism to guarantee
that all values referred to by typed references will be of the correct type.

In this paper, we present a formalization of the monotonic semantics in the
same framework used in our companion paper~\cite{us} for comparing gradual
type systems. We use a common gradually-typed object-oriented source language
and translate it to a common target language (called \kafka) according to the
monotonic semantics. We then use this formalization to discuss the monotonic
semantic properties, and compare it to the gradual type systems presented in
our companion paper.

\section{Monotonic Gradual Types}

The idea behind the monotonic semantics' was originally put forward by Siek and
Vitousek in~\cite{siek14} for their Reticulated Python language
implementation. Kuhlenschmidt, Almahallawi, and Siek later implemented
monotonic again as part of their Grift functional programming language~\cite
{monotonic-grift}, with the goal of demonstrating its performance. For similar
purposes, it was again implemented for TypeScript by Richards, Arteca, and
Turcotte~\cite {richards-bettermono}, leveraging the correspondance between
monotonic's concept of precision and VM object shapes to improve performance.

The core tenet  of the monotonic semantics is that types only ever get more
precise. This design gives rise to its name, as types of values in a monotonic
language's heap monotonically get more static. In an object-oriented language,
each object in the heap has an associated type, as indicated by its run-time
tag. This tag identifies the class of the object, and it is used for
dispatching method invocations. When a reference to an object is cast to a
more precise type, intuitively a type with fewer occurrences of \any, that new
type is attached to the value in the heap.  A type error will occur if untyped
code later tries to read fields, write fields, or call methods in a way that
violates the updated type. For example, an object may start out under type
$\{\Mtype\m\any\any\}$, then become more static when cast to
$\{\Mtype\m{\xt{int}}{\xt{int}}\}$. This new type will be stored in the heap
and enforced on untyped callers. By locking in applied types forevermore,
typed code under the monotonic semantics can assume that mutable state will
remain well-typed.

Through this mechanism, the monotonic semantics is able to provide a stronger
guarantee about values inhabiting mutable typed values when compared to other
gradual type systems. Unlike the optional semantics, the monotonic semantics
provides a soundness guarantee. It also allows the elision of guards at typed
dereference sites, unlike the transient and behavioral semantics. Finally,
the monotonic guarantee is less strict at the original cast site than the
concrete semantics. Additional details about these other semantics can be
found in our companion paper~\cite{us}.

As a result of this stronger soundness guarantee, the monotonic system has the
potential to provide substantial benefits to the performance of gradually
typed languages. Variables are guaranteed to refer to values of their
statically known type, and therefore their types can be used to discover the
memory layout of the underlying object.  Therefore, the objects' memory layout is
known statically and field references made direct. This optimization has the
potential to reduce one of gradual typing's largest overheads.

In the context of object orientation, monotonic's ability to protect mutable
state without the need for wrappers is a significant advantage. Other
gradual type systems need guards, wrappers, or a stronger notion of
soundness in order to guarantee safety of field access. The monotonic
semantics, in contrast, prevents untyped code from assigning ill-typed
values to previously typed references, allowing unchecked dereferencing in
typed code.

\section{Formalism}

We use the framework we set out in our companion paper to provide a simple
presentation of the monotonic semantics in an object-oriented setting. Instead
of using formal languages specially crafted to support each semantics, we use
a common source and target language for every gradual typing semantics. The 
semantics is then encoded into the translation, allowing comparison of gradual 
type systems by examining their translation.

\subsection{Source Language}

Our common source language is similar to Featherweight Java (FJ), with a few
changes. This language has classes but relies on structural subtyping instead
of the nominal subtyping adopted by FJ. For our purposes, there is no need for
inheritance. Structural subtyping is needed to support the evolution of types,
as casts are being applied. The source language also introduces a dynamic
type, \any, and a convertibility relation, allowing gradually typed programs
to be expressed. This same source language and source language type system is
shared between the translation presented here and the four in our main paper,
allowing the same programs to be run under all of our semantics. We replicate
the static semantics for the source language here, in Figure~\ref{slts}. The
dynamic semantics of the source language is given by a translation to a
statically-typed language, which we will present next.

Programs in the source language consist of a class table, \K, and an initial
expression, \e. Classes consist of field and method definitions. Fields have
a name and a type, \f:\t. Method definitions have names, types for their
single argument and return type, and an expression. Expressions consist of
variables, the self-reference \this, field read and write \FRead\f and
\FWrite\f\e, invocation \Call\e\m\e, and new object creation
\New\C{\e[1]..}. Each class has an implicit default constructor that merely
assigns arguments to fields. All operations conform to what they would do in
Featherweight Java, and the type system is straightforward.

The static type system allows the implicit conversion of typed and untyped
terms via the convertibility operator, written $\ConvertE\K s\t\tp$ -- type
$\t$ is convertible to type $\tp$.  The relation is used both for up-casting
and for conversions of \any to non-\any types.  \RuleRef{SUB} allows transparent
up-casting, while \RuleRef{TOA} and \RuleRef{ANYC} allow implicit
conversion to and from the dynamic type.  To avoid collapsing the type
hierarchy, convertibility is not transitive. With a naive semantics, uses
of \RuleRef{TOA} would break soundness, since dynamically typed values would
enter statically typed code without any kind of guard. 

The source language enforces field privacy. Fields can only be accessed
from an object's own methods. Furthermore, fields do not show up in type
signatures. Subtyping between objects is computed solely on methods
signatures.


\begin{figure*}[!t]

\hrulefill

\vspace{4mm}

\begin{tabular}{@{}l@{~}l@{}l@{}l@{}l@{}l@{}l@{}l}
\e~::=~\x\B\this\B\FRead\f\B\FWrite\f\e\B\Call\e\m\e\B\New\C{\e[1]..}\\
\k~::=~ \Class \C {\fd[1]..}{\md[1]..} \qquad
\md~::=~\Mdef\m\x\t\t\e\qquad
\fd~::=~ \Fdef\f\t\qquad
\t~::=~ \any \B \C
\end{tabular}

\begin{mathpar}
\Rule{STG-VAR}{~\\\\
  \HasType \Env\x\t
}{
  \EnvTypeS \Env\K\x\t
}

\Rule{STG-GET}{
  \HasType \Env\this\C \\\\  \Fdef\f\t \in \App\K\C
}{
  \EnvTypeS \Env\K{\FRead\f}\t
}    

\Rule{STG-SET}{
  \HasType \Env\this\C \quad \Fdef\f\t \in \App\K\C \\\\
  \EnvTypeS \Env\K\e\tp \quad  \ConvertE\K{s}\tp\t
}{
  \EnvTypeS \Env\K{\FWrite\f\e}\t
}    

\Rule[width=15em]{STG-CALL}{
  \EnvTypeS \Env\K\e\any \\\\ \EnvTypeS \Env\K\ep\t 
}{
  \EnvTypeS \Env\K{\Call\e\m\ep}{\any}
}    

\Rule[width=15em]{STG-CALL}{
  \EnvTypeS \Env\K\e\C \quad \EnvTypeS \Env\K\ep\t \\\\
  \Mtype \m{\t[1]}{\t[2]}\in \App\K\C  \quad
  \ConvertE\K{s}\t{\t[1]}
}{
  \EnvTypeS \Env\K{\Call\e\m\ep}{\t[2]}
}   

\Rule{STG-NEW}{
  \Ftype{\f[1]}{\t[1]}.. \in \App\K\C \\\\
  \EnvTypeS \Env\K{\e[1]}{\tp[1]}..\quad \ConvertE\K{s}{\tp[1]}{\t[1]}..
}{
  \EnvTypeS \Env\K{\New\C{\e[1]..}}\C
}
\end{mathpar}

\vspace{-5mm}  
  
\begin{mathpar}
\IRule{SUB}{
   \SSub\emptyset\K\t\tp
}{
   \ConvertE\K{s}\t\tp
}
    
\IRule{TOA}{~ }{ \ConvertE\K{s}\t\any}
    
\IRule{ANYC}{~}{ \ConvertE\K{s}\any\t }
\end{mathpar}


\hrulefill
\caption{Surface language syntax and type system (extract).}\label{slts}
\end{figure*}

\begin{figure*}[!t]

\hrulefill

\vspace{4mm}

\small\begin{tabular}{@{}lll}

\begin{minipage}{5.5cm}\begin{tabular}{@{}l@{~}l@{}l@{}l@{}l@{}l@{}l@{}l}
\e\hspace{.1cm} ::= & \hspace{.2cm} \x        
   &\B \this         
   &\B \that      
   &\B \FRead\f  \\   
   & &
   &\B \FWrite\f\e 
   &\B \KCall\e\m\e\t\t \\
   & &   
   &\B \New\C{\e[1]..}  
   &\B \DynCall\e\m\e \\
   & & & \multicolumn{3}{@{}l}{\B \a \B \FReadR\a\f \B \FWriteR\a\f\e} \\
   & & & \B \src{\MonCast\t\e}
\end{tabular}\end{minipage}&
\begin{minipage}{3.7cm}\begin{tabular}{l@{~}l@{}l@{}l}
  \k &::= \Class \C {\fd[1]..}{\md[1]..} \\
 \md &::= ~ \Mdef\m\x\t\t\e \\ 
 \fd &::= ~ \Fdef\f\t \\ 
  \t &::= ~ \any  \B   \C  \\ 
\end{tabular}\end{minipage} &
\begin{minipage}{5.5cm}

\begin{tabular}{llllllllllllllllll}
\EE &::=& ~ \FWriteR\a\f\EE   &\B  
        \KCall\EE\m\e\t\t  &\B\\
        &&\KCall\a\m{\EE}\t\t &\B
        \DynCall\EE\m\e   &\B\\
&       & \DynCall\a\m\EE   &\B
       \src{\MonCast\t\EE}  &\B\\
&&     \New\C{\a[1]..\,\EE\,\e[1]..} &\B 
      \EM{\square}
\end{tabular}
\end{minipage}
\end{tabular}


\begin{mathpar}
\IRule{KT-VAR}{
   ~\\\\
   ~\\\\
   \HasType \Env\x\t
 }{
   \EnvType \Env\s\K\x\t
}

\IRule{KT-SUB}{
  ~\\\\
  \EnvType \Env\s\K\e\tp \\\\
 \StrSub \cdot\K \tp \t
 }{
  \EnvType \Env\s\K\e\t 
}   

\IRule{KT-READ}{
  ~\\\\
  \HasType\Env\this\C \\\\
  \Fdef\f\t \in \App\K\C
}{
  \EnvType \Env\s\K{\FRead\f}\t
}  

\IRule{KT-REFREAD}{
  \s(\a) = \C\{..\} \\\\
  \Fdef\f\t \in \App\K\C
}{
  \EnvType \Env\s\K{\FReadR\a\f}\t
}  

\IRule{KT-WRITE}{
  \HasType\Env\this\C \\\\
  \Fdef\f\t \in \App\K\C \\\\
  \EnvType \Env\s\K\e\t
}{
  \EnvType \Env\s\K{\FWrite\f\e}\t
}    

\IRule[width=12em]{KT-REFWRITE}{
  \s(\a) = \C\{..\} \\\\
  \Fdef\f\t \in \App\K\C \\\\
  \EnvType \Env\s\K\e\t
}{
  \EnvType \Env\s\K{\FWriteR\a\f\e}\t
}  

\IRule[width=16em]{KT-CALL}{
  \EnvType \Env\s\K\e\C \\\\
  \EnvType \Env\s\K\ep\t \\\
  \Mtype\m\t\tp \in \App\K\C 
}{
  \EnvType \Env\s\K{\KCall\e\m\ep\t\tp}\tp
}    

\IRule{KT-DYNCALL}{
  ~\\\\
  \EnvType \Env\s\K\e\any \\\\
  \EnvType \Env\s\K\ep\any
}{
  \EnvType \Env\s\K{\DynCall\e\m\ep}\any
}    

\IRule[width=20em]{KT-NEW}{
  ~\\\\
  \EnvType \Env\s\K{\e[1]}{\t[1]}..\\\\
  \Class \C {\Fdef{\f[1]}{\t[1]}..}{\md[1]..} \in \K
}{
  \EnvType\Env\s\K{\New\C{\e[1]..}}\C
}

\src{
\IRule{KT-MONOCAST}{
  \EnvType \Env\s\K\e\tp
}{
  \EnvType \Env\s\K{\MonCast\t\e}\t
}
}

\IRule{KT-REFTYPE}{
  \s(\a) = \obj\C{\ap[1]..}
}{
  \EnvType \Env\s\K\a\C
}

\IRule{KT-REFANY}{
 }{
   \EnvType \Env\s\K\a\any
}
\end{mathpar}

\vspace{4mm}

\begin{minipage}{0.45\textwidth}
\begin{tabbing}
\K\HS\New\C{\a[1]..}\HS\=\s~\HS\=\Red\HS\=\K\HS\=\ap\HS\hspace{1em}\=\sp\HS\=
   \WHERE\HS\=\fresh\ap \HS\HS\sp ={\Map\s{\Bind\ap{\obj\C{\a[1]..}}}}
\\
\K\HS \FReadR\a{\f[i]} \> \s \>\Red\>  \K\>$\a[i]$ \> \s  \> 
   \WHERE \>\App\s\a=\obj\C{\a[1],..\a[i],\a[n]..}
\\
\K\HS{\FWriteR\a{\f[i]}\ap}\>\s\>\Red\>\K\>\ap\>\sp\>  
   \WHERE \>\App\s\a=\obj\C{\a[1],..\a[i],\a[n]..} 
   \HS\HS \sp=\Map\s{\Bind\a{\obj\C{\a[1],..\ap,\a[n]..}}}
\\
\K\HS{\KCall\a\m\ap\t\tp} \> \s      \>\Red\>     \K \>  \ep \> \s \> 
    \WHERE\> \ep = {[\a/\this~{\ap/\x}]\e} \HS\HS
             \Mdef\m\x{\t_{1}}{\t_{2}}\e\In \App\K\C \HS\HS
             \App\s\a=\obj\C{\a[1]..} \\ \>\>\>\>\>\>\>
               \StrSub {\emptyset}\K\t {\t_{1}} \HS\HS
 \StrSub {\emptyset}\K{\t_{2}} \tp
\\
 \K\HS {\DynCall\a\m\ap}\> \s        \>\Red\>    \K \> \ep \> \s \>  
       \WHERE\> \ep = {[\a/\this~{\ap/\x}]\e}\HS\HS
                \Mdef\m\x\any\any\e \In \App\K\C \HS\HS
                 \App\s\a=\obj\C{\a[1]..} 
\\
\hspace{-1em}\colorbox[gray]{0.89}{\hspace{1.20\textwidth}\vphantom{hello}}\hspace{-1.20\textwidth} \K\HS {\MonCast \t\a} \> \s     \>\Red\>   \Kp \> \ap \> \sp \> \WHERE\> \moncast \a\t\s\K \Kp\ap\sp 
\\
\K \HS \EM{\EE[\e]} \> \s            \>\Red\>   \Kp \> \EM{\EE[\ep]} \> \sp \> \WHERE \> \K~\e~\s \Red~\Kp~\ep~\sp
\end{tabbing}\end{minipage}\vspace{2mm}

\noindent\hrulefill

\caption{\kafka syntax, static and dynamic semantics, and monotonic
  translation.}\label{syn}
\end{figure*}


\subsection{KafKa}

\kafka's design follows the surface language with some differences.
Expressions are extended with \that, a distinguished variable used for
wrapped object reference, typed and untyped method invocations, subtype and
monotonic casts.  Evaluation is likewise standard with a few exceptions,
with an evaluation context consisting of a class table \K, expression
being evaluated \e, and heap \s, mapping from addresses \a to objects,
denoted $\C\{\a\ldots\}$. Due to the need for dynamically generated classes,
the class table \K is part of the state, and not the evaluation context.
\kafka has two invocation forms, the untyped invocation $\DynCall\e\m\ep$
and the typed invocation $\KCall\e\m\ep{\C}{\D}$, both denoting a method
call to method $\m$ with argument \ep. This is unusual for two
reasons. First, most gradual type systems have no explicit untyped
invocation form, and second, the typed form also specifies the argument type
and return type. Our system explicitly indicates which function calls are
statically guaranteed to succeed and which must be dynamically checked. It
also allows for a limited form of overloading, where both typed and untyped
versions of a method may share a name.

We use a gray background to indicate changes to \kafka in figure~\ref{slts} to
support monotonic. The only addition is that of the monotonic cast. The next
section focuses on giving a semantics to that cast.


\section{Monotonic Semantics}

The monotonic semantics is defined by a translation from the surface
language down to \kafka and the meaning of the monotonic cast. 

The translation is syntax directed, each surface class is translated to an
eponymous \kafka class. Types of methods and fields are retained in the
translation. Expressions are translated with $\TRG[M]{\e}\Env$. Variables
are translated to themselves. Calls are translated to either dynamic calls
or static calls depending on the type of the receiver.  If the receiver is
of type \any, then the argument must be of type \any. If the receiver of
some class \C then the argument must be of type \D, where \D is the expected
argument type of the method.  We use a variant of the translation relation
that performs a cast if necessary on its argument expression,
\TAG[M]{\e}{\Env}\t.  The inserted cast is the monotonic cast,
\MonCast{\t}{\ep}.

The semantics of monotonic arises from the implementation of the monocast
metafunction, used in the evaluation of monotonic casts shown in
figure~\ref{moncast}. At a high level, the cast operation computes the
meet of the source types (the most specific join of them), then applies it
by replacing the class associated with the object in memory. This is depicted
in rule \xt{CM}.

The first step in this operation is the meet operator. The meet operation is
akin to the one over a lattice, taking two types and always producing a type
higher on the type lattice. Two types have a meet if they are structurally
the same up to \any types distributed throughout their structure. The meet
itself consists of replacing all \any types in both types with the most
specific known type in that position from either type.

\begin{figure}[h]
\begin{lstlisting}
class C { m(x:*):* { x } n(x:C):C { x } }
class D { m(x:D):D { x } n(x:*):* { x } }
class E { m(x:D):D { x } n(x:C):C { x } }
\end{lstlisting}
\caption{Meet C and D}
\label{fig:meetex}
\end{figure}

\noindent
For example, if we were to compute the meet between \C and \D as  shown in
figure~\ref{fig:meetex}, we would end up with the type \xt{E}. The \any typed
argument to \m in \C is taken to be of type \D from class \D, while the
converse is used for \n in \D. The class \xt{E} then is more specific than
both \C and \D.

This operation relies on the meet metafunction which takes the two target
types, a context (for use in recursive types), and the class table. It then
outputs a new class, the meet of the two types, and an updated classes table
(meet can create multiple types). The context \xt{P}, either empty $\cdot$
or a context extended with a mapping $\P ~{{(\C,\D)}\mapsto\Cp}$, is used to
ensure termination of the meet operation on recursive types by storing the
names of meet-generated classes. Meet is deterministic, so meet of
previously-encountered types will be identical. We similarly extend meet to
operate over method definitions, picking out the comparable method definition
in the target method, and generate a suitable method. When two list of
methods are met, their result is passed into ifcgen, which generates an
``interface''. An interface is trivial class that exists only for the
purpose of giving us a type signature.  It will never be used.  If we were
to use it we would run into trouble.  Suppose we start out with
$\x=\New\C{}$, then alias it with $\xp=\MonCast\D\x$. If we then call
$\DynCall\x\m{\New\D{}}$, and the underlying object is an instance of
\xt{E}, the dynamic call will fail as no dynamically-typed method exists. As
a result of this, our system does not deem \xt{E} to be in a subtyping
relationship with either \C or \D. Moreover, ifcgen makes no attempt to
produce correct method bodies.

\begin{figure}[h]
\begin{lstlisting}
class F { m(x:D):D { x } m(x:*):* { <*>this.m(<D>x) }
          n(x:C):C { x } n(x:*):* { <*>this.n(<C>x) }}
\end{lstlisting}
\caption{wrap result}
\label{fig:monwrap}
\end{figure}

To fix this, we need to add additional guard methods that ensure the
presence of an untyped target method even if cast. The meet of classes \C
and \D in our system is therefore class \xt{F} as shown in
figure~\ref{fig:monwrap}, which has both the typed and untyped methods
specified in both \C and \D. Class \xt{F} is a proper structural subtype of
both \C and \D, and its methods perform semantically correct operations. It
is generated by the wrap metafunction, which produces an equivalent class
from source and target classes. It ensures that any untyped method in the
source is retained in the target, and that typed methods are inserted.

\begin{figure*}[!t]
\center

\begin{center}
\begin{minipage}{10cm}
\begin{align*}
\TRG[M]{\x}\Env &= \x & \\
\TRG[M]{\Call{\e_1}\m{{\e_2}}}\Env &= \DynCall{\ep_1}\m{\ep_2} & 
   \IF\EM{\TypeCk{\K,\Env}{\e[1]}\any ~ \AND ~
    \TRG[M]{\e_1}\Env = \ep_1}
   \EM{~\AND~ \TAG[M]{\e_2}\Env{\any} = \ep_2}\\
\TRG[M]{\Call{\e_1}\m{{\e_2}}}\Env &= \KCall{\ep_1}\m{\ep_2}{\D[1]}{\D[2]} & 
    \IF\EM{\TypeCk{\K,\Env}{\e[1]}\C ~ \AND\; ~ 
    \Mtype\m{\D[1]}{\D[2]}\In\App\K\C}
  ~\EM{\AND ~ \TRG[M]{\e_1}\Env= \ep_1 
   ~\AND~ \TAG[M]{\e_2}\Env{\D[1]} = \ep_2}\\
\TRG[M]{\New\C{{\e_1}..}}\Env &= \New\C{\ep_1..} & 
   \IF\EM{\Ftype{\f[1]}{\t[1]}\In\App\K\C ~\AND~
     \ep[1] = \TAG[M]{\e[1]}\Env{\t[1]} ~..}\\
\TAG[M]{\e}{\Env}\t &= \e' & \IF\EM{\TypeCk{\K,\Env}\e\tp 
     ~\AND~ \K\vdash\tp\Sub\t}\\
\TAG[M]{\e}{\Env}\t &= \MonCast{\t}{\ep} & \IF\EM{\TypeCk{\K,\Env}\e\tp 
    ~\AND~ \K\vdash\tp\not\Sub\t}
\end{align*}
\end{minipage}
\end{center}
\caption{Monotonic Translation}
\end{figure*}
\begin{figure*}
\begin{mathpar}
\IRule{CM}{
  \s(\a) = \C\{\ap..\} \\
  \fresh\Cpp\\
  \md.. = \xt{meths}(\C,\K)\\
  \mt.. = \xt{mtypes}(\C,\K)\\
  \mtp.. = \xt{mtypes}(\Cp,\K)\\\\
  \tmeet \C\t\cdot\K = \Cp~\Kp\\
  \Kpp = \classgen{\C, \md.., \mt.., \mtp.., \Cpp, \K}\, \Kp \\
  \sp = \s[\a \mapsto \Cpp\{\ap..\}]
}{
  \moncast \a\t\s\K \Kpp \sp\\
}
\end{mathpar}

\begin{mathpar}

\IRule{TM1}{ }{\tmeet\C\any\P\K = \C\,\K}

\IRule{TM2}{ }{\tmeet\any\C\P\K = \C\,\K}

\IRule{TM3}{ }{\tmeet\t\t\P\K = \t\,\K}

\IRule{TM5}{
    \P(\C,\D) = \Cp
}{
    \tmeet\C\D\P\K = \Cp\,\K
}

\IRule{TM4}{
  \fresh\Cp\\
  (\C,\D) \not\in\P \\
  \Pp = \P~{{(\C,\D)}\mapsto\Cp} \\
  \mmeet{\xt{mtypes}(\C,\K)}{\xt{mtypes}(\D,\K)}\Pp\K = \mtpp..\,\Kp\\
  \Kpp = \Kp~\typegen{\mtpp..}\Cp\\
}{
    \tmeet\C\D\P\K = \Cp\,\Kpp
}

\end{mathpar}


\begin{mathpar}
\IRule{MM1}{}{
  \mmeet{\mt ..}{\cdot}\Env\K =\mt.. ~\K
}

\IRule{MM2}{}{
  \mmeet{\cdot}{\mt ..}\Env\K ={\mt ..} ~\K
}

\IRule{MM6}{
  \Mtype\m{\t_3}{\t_4} \in {\mt_2} \\
  \tmeet{\t_3}{\t_1}\Env\K = \t_5~\Kp \\
  \tmeet{\t_2}{\t_4}\Env\Kp = {\t_6}~{\Kpp} \\
  \mmeet{{\mt_1 ..}}{{\mt_2 ..}}\Env\Kp = {\mt_3 ..}~\Kpp
}{
  \mmeet{\Mtype\m{\t_1}{\t_2}~{\mt_1..}}{{\mt_2..}}\Env\K =\Mtype\m 
   {\t_5}{\t_6} ~{\mt_3 ..} ~\Kpp
}
\end{mathpar}

\vspace{5mm}

\begin{center}\begin{minipage}{12cm}
\noindent\[\begin{array}{l@{~}r@{}r@{~}r@{~}r@{~}rr}
\arrayrulecolor{white}
\typegen{\mt..}{\D} = \\
\SP \class~\D~\{
\\[1mm]
\SPP \Mdef\m\x\t\tp {{\MonCast\tp{\x}}} 
&
\qquad \All m \Mtype\m\t\tp\in\mt..
\\[1mm]
\SP\}
\end{array}
\]

\vspace{5mm}

\[\begin{array}{l@{~}r@{}r@{~}r@{~}r@{~}rr}
\arrayrulecolor{white}
\classgen{\C, \md.., \mt.., \mtp.., \D, \K}= \\
  \SP \class~\D~\{ \\
    \SPP \Fdef\f\t   &
        \All{\f\!:\t\in\text{fields}(\K,\C)} \\[1mm]
    \SPP \Mdef\m\x\any\any{~\MonCast\any{\KCall\this\m{\MonCast{\t[1]}\x}{\t[1]}{\t[2]}}~}\HS\HS & 
          \All\m \Mdef\m\x{\any}{\any}\e\in\md..  ~\AND~ \Mtype\m{\C_1}{\C_2}\in{\mtp..}
\\[1mm]\hline
\SPP \Mdef\m\x{\C_1}{\C_2} {\MonCast{\C_2}{[{(\MonCast{\any}\x)}/\x]\e}~}
&     \All \m \Mdef\m\x{\any}{\any}\e\in\md.. ~\AND~ \Mtype\m{\C_1}{\C_2}\in\mtp.. 
\\[1mm]\hline
\SPP \Mdef\m\x{\C_1}{\C_2} {\e}~
&     \All \m \Mdef\m\x{\C_1}{\C_2}\e\in\md.. ~\AND~ \Mtype\m{\C_1}{\C_2}\in\mtp.. 
\\[1mm]\hline
\SPP \Mdef\m\x{\any}{\any} {\e}~
&     \All \m \Mdef\m\x{\any}{\any}\e\in\md.. ~\AND~ \Mtype\m{\C_1}{\C_2}\not\in\mtp.. 
\\[1mm]\hline
\SP\}
\end{array}
\]\end{minipage}\end{center}
\caption{Monotonic Cast Operations}
\label{moncast}
\end{figure*}

\section{Example}

To illustrate the operation of our semantics, we present a fully worked
example of a simple program running under the monotonic semantics. This
program, shown in figure~\ref{fig:monoex}, is first translated from source
to \kafka and then executed according to the \kafka semantics described
above.

This program is designed to illustrate the cast semantics of monotonic. The
method \xt{alias} aliases its argument, casting it to both \C and \D. The
value cast to \C is then retrieved and its method \xt{check} is then called
(in untyped code) with an \xt{I}. The reference is identical to the one
originally passed to \xt{alias} and no wrapper has been applied. One would
expect then that the untyped call to \xt{check} would pass, since the body
of \xt{check} is the identity function, but under the monotonic semantics it
fails.

Before explaining the semantics of this program in more detail, we will first
describe the top level type-driven transformation. First, casts are inserted
wherever a statically typed value is passed to the dynamic type \any and vice versa. Then, typed
methods have untyped shims generated, so they can be called in a protected
manner from untyped code. 

Execution starts with the expression \code{new E().alias(<*>new
C()).check(<*>new C())}. \code{new E()} evaluates to a reference \a to an
object $\xt{E}\{\}$, upon which \xt{alias} is invoked with argument \New\C{}
(which evaluates to $\a[1]$, mapping to $\C\{\}$). \xt{alias} then expands to
$\MonCast\any{\New{\xt{T}}{\MonCast\C{\a[1]},\MonCast\D{\a[1]}}}$. The first
$\MonCast\C{\a[1]}$ is a no-op, as ${\a[1]}$ is already of type \C, but the second
$\MonCast\D{\a[1]}$ modifies the runtime type of the value stored at ${\a[1]}$.

\D defines \xt{check} to take and return instances of \xt{I}, while \C  has
\xt{check} taking and returning values of type \any. As \D's definition is
more precise, it will take priority. As a result, the result of meet will be
\Cp, with \Class\Cp{}{\Mdef{\xt{check}}\x{\xt{I}}{\xt{I}}\x} appended to the
class table. monWrap is then applied, producing the final new class \Cpp from
\Cp, consisting of \Class\Cpp{}{\Mdef{\xt{check}}\x{\xt{I}}{\xt{I}}{\MonCast{\xt{I}}{\MonCast{\any}{\x}}} ~ \Mdef{\xt{check}}\x\any\any{\MonCast{\any}{\KCall\this{\xt{check}}{\MonCast{\xt{I}}{\x}}{\xt{I}}{\xt{I}} }}}. This is then used to overrite ${\a[1]}$'s class in the heap. ${\a[1]}$ originally
referred to $\C\{\}$, but it will now refer to $\Cpp\{\}$ after the cast.

At this point, the initial expression has been reduced to
\DynCall{\MonCast\any{\KCall{\New{\xt{T}}{{\a[1]}, {\a[1]}}}{\xt{first}}\any\any
}}{\xt{check}}{\MonCast\any{\New\C{}}}. The inner expression
\KCall{\New{\xt{T}}{{\a[1]}, {\a[1]}}}{\xt{first}}{{\a[1]}}\any\any evaluates to
\KCall{\a[2]}{\xt{first}}{{\a[1]}}\any\any where \a[2] refers to $\xt{T}\{{\a[1]},{\a[1]}\}$
in the heap. The \xt{first} function then is called, returning {\a[1]} under type
\any.

This leaves us with \DynCall{{\a[1]}}{\xt{check}}{\MonCast\any{\New\C{}}}. We
evaluate \New\C{} to \a[3] (mapped to $\C\{\}$), and make the invocation. A
method \xt{check} under type \any and \any exists, so the program does not get
stuck. \app is an instance of \Cpp, so it is dispatched to the generated
method \xt{check} made by the monocast operation. As a result, our new term is
\MonCast{\any}{\KCall{{\a[1]}}{\xt{check}}{\MonCast{\xt{I}}{\a[3]}}{\xt{I}}{\xt{I}}}.
However, at this point, we get stuck, as no meet exists between the runtime type of
\a[3] (mapped to $\C\{\}$) and \xt{I}. 

The example illustrates an interesting property of the monotonic semantics.
Despite this program working perfectly in an untyped language, the monotonic
semantics caused it to fail. More interestingly, no other gradual typing semantics
would experience a cast error on this program. This
is because none will retain the type cast of the first alias to \a[3] in the
underlying object, since the alias is immediately thrown away. 
 
\begin{figure}[!h]
\begin{tabular}{cc}
\begin{lstlisting}[linewidth=0.22\textwidth]
class C{ check(x:*) {x}}
class D{ check(x:I):I {x}}

  

class I{ dif(x:*):* {x}}

class T { 
 f1:C f2:D 
 first(x:*):* {this.f1}}

class E { 
 alias(x:*):* { 
  new T(x,x).first(x) }}


new E().alias(new C())
       .check(new C())
\end{lstlisting}  &
\begin{lstlisting}[linewidth=0.22\textwidth]
class C { check(x:*) {x}}
class D { 
 check(x:I):I {x}
 check(x:*):* { 
    <*>this.m(<I>x) }}
class I { dif(x:*):* {x}}

class T { 
 f1:C f2:D 
 first(x:*):* {
    <*>this.f1}}
class E { 
 alias(x:*):* { 
    <*>new T(<C>x,<D>x)
            .first(x)}}

new E().alias(<*>new C())
       @check(<*>new C())
\end{lstlisting}\\
Source & \kafka 
\end{tabular}
\caption{Monotonic Semantics Example}
\label{fig:monoex}
\end{figure}

\footnotesize
\normalsize

\section{Conclusion}

This paper has presented the monotonic semantics for gradual types of object
oriented programs as a translation from source programs to \kafka.  One of
the key properties of that translation is that the monotonic casts attempts
to compute the meet of the class of the target object and the requested
type.  If the two cannot be reconcilled the cast fails, otherwise the type
of the object is updated in place to reflect the new type. This allows
monotonic programs to avoid the potentially unbounded layers of wrappers of
previous approaches. The monotonicity property ensure that there can be at
most as many wrappers as there are occurrences of \any in the
type. Practical implementation can have a single wrapper.

One simplification that we have made here comes from the design of
\kafka. \kafka does not reflect fields in the types of objects, they are
fully encapsulated.  This means that when computing the meet we do not have
to update the type of fields. A language that makes fields accessible
outside of objects would have to also meet the values referenced by fields,
which may mean that a meet operation could update multiple objects.

Given the complexity of the operations performed by meet, more experience is
needed to determine if the approach is viable. The implementation of
Monotonic Reticulated Python has not yet been widely adopted, so few programs 
have been written in this style.

\bibliographystyle{ACM-Reference-Format}
\bibliography{../../bib/jv,../../bib/all}

\end{document}
