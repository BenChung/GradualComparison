\documentclass[]{article}

%opening
\title{README for Kafka artifact}
\author{}
\date{}

\begin{document}

\maketitle

\section{Kafka's .Net implementation and what it means}

The Kafka implementation is written in F$\#$. The implementation follows the same syntax and semantics as those presented in figure 3 and 4 of the paper. Explain how to build it, how to run it, which tests to run, and what the result of the tests mean. The litmus tests have been translated into each of the gradual semantics (optional, behavioral, transient, concrete).

Build the application for the implementation, build and submit the DLL. Need to define equality for expression to not include the parsec position. 


\section{Coq and what it means}



\section{How to install each naive languages}



\subsection{TypeScript}

https://www.typescriptlang.org/


\subsection{Thorn}



\subsection{Typed Racket}



\subsection{Reticulated Python}



\section{Litmus tests in naive languages}

Below we present source code for each of the litmus tests.

\subsection*{Concrete}

The code for the litmus tests in Thorn. \\ 

\noindent
\textbf{Litmus Test 1}:
\begin{verbatim}
class A() { def m(x:A):A = this; }
class I() { def n(x:I):I = this; }
class T() {
def s(x:I):T = this;
def t(x:dyn):dyn = this.s(x);
}
T().t(A());
\end{verbatim}

\noindent
\textbf{Litmus Test 2}:
\begin{verbatim}
class Q() { def n(x: Q): Q = this;}
class A() { def m(x:A): A = this;}
class I() { def m(x:Q):I = this;}
class T() {
def s(x:I):T = this; 
def t(x:dyn):dyn = this.s(x);
}
T().t(A());   
\end{verbatim}


\noindent\textbf{Litmus Test 3}:
\begin{verbatim}
class C() { def m(x:C):C = x; }
class D() { def n(x:D):D = x; }
class E() { def m(x:D):D = x; }      
class F() {
def m(x:E):E = x;
def n(x:dyn):dyn = this.m(x);
} 
F().n(C()).m(C());
\end{verbatim}

\subsection*{Optional}

The code for the litmus tests in TypeScript. \\

\noindent\textbf{Litmus Test 1}:
\begin{verbatim}
class A { m(x: A): A { return this } }
class I { n(x:I):I { return this } }
class T {
s(x: I): T { return this }
t(x: any): any { return this.s(x) }
}
new T().t(new A())
\end{verbatim}

\noindent\textbf{Litmus Test 2}:
\begin{verbatim}
class Q { n(x: Q): Q { return this } }
class A { m(x: A): A { return this } }
class I { m(x:Q):I { return this } }
class T {
s(x: I): T { return this }
t(x: any): any { return this.s(x) }
}
new T().t(new A())
\end{verbatim}

\noindent\textbf{Litmus Test 3}:
\begin{verbatim}
class C { m(x: C): C { return x } }
class D { n(x: D): D { return x } }
class E { m(x: D): D { return x } }
class F {
m(x: E): E { return x }
n(x: any): any { return this.m(x) }
}
new F().n(new C()).m(new C())
\end{verbatim}

\subsection*{Behavioral}

The code for the litmus tests in Typed Racket. \\

\noindent\textbf{Litmus Test 1}:

\begin{verbatim}
#lang racket
(module u racket
(define Tp% (class object%
(super-new)
(define/public (t x) (send this s x))))
(provide Tp%))
(module t typed/racket
(require/typed (submod ".." u) [Tp% (Class [t (-> Any Any)])])
(define-type A (Instance (Class (m (-> A A)))))
(define-type I (Instance (Class (n (-> I I)))))
(define-type T (Instance (Class (s (-> I T)))))
(define T% (class Tp%
(super-new)
(: s (-> I T))
(define/public (s x) this)))
(define A% (class object%
(super-new)
(: m (-> A A))
(define/public (m x) this)))
(provide T% A%))
(require 't)
(send (new T%) t (new A%))
\end{verbatim}

\noindent\textbf{Litmus Test 2}:

\begin{verbatim}
#lang racket
(module u racket
(define Tp% (class object%
(super-new)
(define/public (t x) (send this s x))))
(provide Tp%))
(module t typed/racket
(require/typed (submod ".." u) [Tp% (Class [t (-> Any Any)])])
(define-type Q (Instance (Class (n (-> Q Q)))))
(define-type A (Instance (Class (m (-> A A)))))
(define-type I (Instance (Class (m (-> Q I)))))
(define-type T (Instance (Class (s (-> I T)))))
(define T% (class Tp%
(super-new)
(: s (-> I T))
(define/public (s x) this)))
(define A% (class object%
(super-new)
(: m (-> A A))
(define/public (m x) this)))
(provide T% A%))
(require 't)
(send (new T%) t (new A%))
\end{verbatim}

\noindent\textbf{Litmus Test 3}:

\begin{verbatim}
#lang racket
(module u racket
(define Fp% (class object%
(super-new)
(define/public (n x) (send this m x))))
(provide Fp%))
(module t typed/racket
(require/typed (submod ".." u) [Fp% (Class [n (-> Any Any)])])
(define-type C (Instance (Class (m (-> C C)))))
(define-type E (Instance (Class (m (-> D D)))))
(define-type D (Instance (Class (n (-> D D)))))
(define F% (class Fp%
(super-new)
(: m (-> E E))
(define/public (m x) x)))
(define C% (class object%
(super-new)
(: n (-> C C))
(define/public (n x) x)))
(provide F% C%))
(require 't)
(send (send (new F%) n (new C%)) m (new C%))
\end{verbatim}


\subsection*{Transient}

The code for the litmus tests in Transient Reticulated Python. \\

\noindent\textbf{Litmus Test 1}:
\begin{verbatim}
class A:
def m(self, x:A) -> A:
return self
class I:
def n(self, x:I) -> I:
return self
class T:
def s(self, x:I) -> T:
return self
def t(self, x:Dyn) -> Dyn:
return self.s(x)
T().t(A())
\end{verbatim}

\noindent
\textbf{Litmus Test 2}:
\begin{verbatim}
class C:
def n(self, x:C) -> C:
return self
class Q:
def m(self, x:Q) -> Q:
return self     
class A:
def m(self, x:A) -> A:
return self
class I:
def m(self, x:Q) -> I:
return self
class T:
def s(self, x:I) -> T:
return self
def t(self, x:Dyn) -> Dyn:
return self.s(x)
T().t(A())
\end{verbatim}

\noindent\textbf{Litmus Test 3}:
\begin{verbatim}
class C:
def m(self, x:C) -> C:
return x
class D:
def n(self, x:D) -> D:
return x
class E:
def m(self, x:D) -> D:
return x
class F:
def m(self, x:E) -> E:
return x   
def n(self, x:Dyn) -> Dyn:
return self.m(x)
F().n(C()).m(C())
\end{verbatim}


\end{document}
