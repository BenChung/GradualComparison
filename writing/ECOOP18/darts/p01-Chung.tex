\documentclass[a4paper,UKenglish]{darts-v2018}
%This is a template for producing DARTS artifact descriptions. 
%for A4 paper format use option "a4paper", for US-letter use option "letterpaper"
%for british hyphenation rules use option "UKenglish", for american hyphenation rules use option "USenglish"
% for section-numbered lemmas etc., use "numberwithinsect"
 
\usepackage{microtype}%if unwanted, comment out or use option "draft"

%\graphicspath{{./graphics/}}%helpful if your graphic files are in another directory

%\nolinenumbers to disable line numbers

\bibliographystyle{plainurl}% the recommended bibstyle


% Commands for artifact descriptions
% Written by Camil Demetrescu and Erik Ernst
% April 8, 2014

\newenvironment{scope}{\section{Scope}}{}
\newenvironment{content}{\section{Content}}{}
\newenvironment{getting}{\section{Getting the artifact} The artifact 
endorsed by the Artifact Evaluation Committee is available free of 
charge on the Dagstuhl Research Online Publication Server (DROPS).}{}
\newenvironment{platforms}{\section{Tested platforms}}{}
\newcommand{\license}[1]{{\section{License}#1}}
\newcommand{\mdsum}[1]{{\section{MD5 sum of the artifact}#1}}
\newcommand{\artifactsize}[1]{{\section{Size of the artifact}#1}}

\newcommand{\orcid}[1]{\url{http://orcid.org/#1}}
\newcommand{\email}[1]{\href{mailto:#1}{\texttt{#1}}}

% Author macros::begin %%%%%%%%%%%%%%%%%%%%%%%%%%%%%%%%%%%%%%%%%%%%%%%%
\title{KafKa: Gradual Typing for Objects (Artifact)}%\footnote{This work was partially supported by someone.}}
\titlerunning{Gradual Typing for Objects} %optional, in case that the title is too long; the running title should fit into the top page column

% ARTIFACT: Authors may not be exactly the same as the related scholarly paper, e.g., you may want to include authors who contributed to the preparation of the artifact, but not to the companion paper

\author{Benjamin Chung}{Northeastern University}{}{}{}%mandatory, please use full name; only 1 author per \author macro; first two parameters are mandatory, other parameters can be empty.

\author{Paley Li}{Czech Technical University}{}{}{}

\author{Francesco Zappa Nardelli }{INRIA}{}{}{}

\author{Jan Vitek}{Northeastern University}{}{}{}

\authorrunning{B. Chung, P. Li, F. Z. Nardelli, J. Vitek} %mandatory. First: Use abbreviated first/middle names. Second (only in severe cases): Use first author plus 'et. al.'

\Copyright{B. Chung, P. Li, F. Z. Nardelli, J. Vitek}%mandatory, please use full first names. DARTS license for the artifact description is "CC-BY";  http://creativecommons.org/licenses/by/3.0/

\subjclass{F.3.3 Studies of Program Constructs}% mandatory: Please choose ACM 2012 classifications from https://www.acm.org/publications/class-2012 or https://dl.acm.org/ccs/ccs_flat.cfm . E.g., cite as "General and reference $\rightarrow$ General literature" or \ccsdesc[100]{General and reference~General literature}. 

\keywords{Gradual typing, object-orientation, language design, type systems}% mandatory: Please provide 1-5 keywords
% Author macros::end %%%%%%%%%%%%%%%%%%%%%%%%%%%%%%%%%%%%%%%%%%%%%%%%%

% Please provide information to the related scholarly information
\RelatedArticle{Benjamin Chung and Paley Li and Francesco Zappa Nardelli and Jan Vitek, ``KafKa: Gradual Typing for Objects '', in Proceedings of the 32nd European Conference on Object-Oriented Programming (ECOOP 2018), LIPIcs, Vol.~0, pp.~0:1--0:2, 2018.\newline \url{http://dx.doi.org/10.4230/LIPIcs.xxx.xxx.xxx}}

%Editor-only macros:: begin (do not touch as author)%%%%%%%%%%%%%%%%%%%%%%%%%%%%%%%%%%
\Volume{4}
\Issue{3}
\Article{1}
\RelatedConference{32nd European Conference on Object-Oriented Programming (ECOOP 2018), July 19--21, 2018, Amsterdam, Netherlands}
% Editor-only macros::end %%%%%%%%%%%%%%%%%%%%%%%%%%%%%%%%%%%%%%%%%%%%%%%

\begin{document}

\maketitle

\begin{abstract}
A wide range of gradual type systems have been proposed, providing many
languages with the ability to mix typed and untyped code. However, hiding
under language details, these gradual type systems have fundamentally
different ideas of what it means to be typed. In this paper, we show that
four of the most common gradual type systems provide distinct guarantees,
and we give a formal framework for comparing gradual type systems for
object-oriented languages. First, we show that the different gradual type
systems are practically distinguishable via a three-part litmus test. Then,
we present a formal framework for defining and comparing gradual type
systems. Within this framework, different gradual type systems become
translations between a common source and target language, allowing for
direct comparison of semantics and guarantees.
\end{abstract}

% ARTIFACT: please stick to the structure of 7 sections provided below

% ARTIFACT: section on the scope of the artifact (what claims of the paper are intended to be backed by this artifact?)

\begin{scope}
%What is the scope of the artifact? What claims of the related scholarly paper are intended to be backed by this artifact?
The scope of the artifact includes a complete implementation of the Kafka vm, the translation from native code, and a complete Coq proof of Kafka.

Our Kafka and native translation implementations are written in F$\#$. The folder kafkaimpl contains the translation from the surface source language to kafka, the translation from kafka to CIL, and the type checker for Kafka. The folder kafkatests contains unit tests for the kafka implementation. The implementation of Kafka follows the same syntax and semantics as those presented in figure 3 and 4 of the paper. The implementation of our translation from the source language to Kafka follows the rules presented in section 5 of the paper. We strongly recommend loading, building, and running this implementation within Microsoft Visual Studio, it is by far the simplest and easiest option. 

To run this implementation, it takes two command line variables. The gradual semantics you wish to use and the name path of the file which contains the code that you wish to run. 'beh' is the option for the behavioral semantics, 'opt' for the optional semantics, 'tra' for the transient semantics, and 'con' for the concrete semantics. The three litmus tests presented in the paper are readily available to be ran, and they are in the files: litmus1.txt, litmus2.txt, litmus3.txt. It is possible for the user to pass in a file of their own design.

The result of each litmus test under each gradual semantics reflect the behavior expressed in section 3 of our paper.

We have proved type soundness for Kafka in Coq, the kafka.v file in the impl folder is the self-contained Coq 8.6 proof for KafKa. It is important that the version of Coq must be 8.6, otherwise any Coq IDE will work including the command line.

%Theorem 1 (page 14) in the paper corresponds to the Coq theorem \\

%Soundness: $\forall$ $\ensuremath{\sf{K}}$ $\ensuremath{\sf{e}}$ $\ensuremath{\sf{s}}$ $\t$,

%WellFormedState $\ensuremath{\sf{K}}$ $\ensuremath{\sf{e}}$ $\ensuremath{\sf{s}}$ $\rightarrow$ HasType \texttt{nil} $\ensuremath{\sf{s}}$ $\ensuremath{\sf{k}}$ $\ensuremath{\sf{e}}$ $\t$ $\rightarrow$ is$\_$sound $\ensuremath{\sf{k}}$ $\ensuremath{\sf{s}}$ $\ensuremath{\sf{e}}$ $\t$ .

%HasType \texttt{nil} $\ensuremath{\sf{s}}$ $\ensuremath{\sf{k}}$ $\ensuremath{\sf{e}}$ $\t$ \textit{corresponds to} \texttt{nil} $\ensuremath{\sf{s}}$ $\ensuremath{\sf{k}}$ $\vdash$ $\ensuremath{\sf{e}}$ : $\t$ \\

%The WellFormedState statement combines well-typedness of the class table and the running expression with heap typing. \\

There are three holes in the proof, in kafka.v:
\begin{itemize}
	\item Transitivity of structural recursive subtyping (line number: 342)
	\item Soundness of subtyping (line number: 350)
	\item That subtyping still holds when the class table is expanded (line number: 642)
\end{itemize}

The first two components are well-known prior work (e.g. A Mechanical Soundness Proof for Subtyping Over Recursive Types, Jones 2016). 
The third property simply requires that pre-existing subtyping judgments still hold when the class table is expanded.
\end{scope}

% ARTIFACT: section on the contents of the artifact (code, data, etc.)
\begin{content}
The artifact package includes:
\begin{itemize}
\item Kafka's .Net implementation
\item Kafka's Coq proof
\item Litmus tests in each of the four native languages
\end{itemize}
\end{content}

% ARTIFACT: section containing links to sites holding the
% latest version of the code/data, if any
\begin{getting}
% leave empty if the artifact is only available on the DROPS server.
% otherwise, provide links to the latest version of the artifact (e.g., on github)
In addition, the artifact is also available at:
\url{https://github.com/BenChung/GradualComparison}.
\end{getting}

% ARTIFACT: section specifying the platforms on which the artifact is known to
% work, including requirements beyond the operating system such as large
% amounts of memory or many processor cores
\begin{platforms}
Please specify the platforms on which the artifact is known to
work, including requirements beyond the operating system such as large
amounts of memory or many processor cores.

In this section, we discuss how to install each of the native language that is required to run the native litmus tests.

\subsection{TypeScript}

For TypeScript, all the information regarding the language and the process of installment can be found at https://www.typescriptlang.org/.

\subsection{Thorn}

Unfortunately, there does not exist a public implementation of Thorn that is readily available. The Thorn skeleton, which we obtained through private means, was heavily savaged from bit rot and decay, and was not easily installable nor it contain all the necessary components, such as the thorn type checker.

\subsection{Typed Racket}

For Typed Racket, first you would need to install the Racket IDE called DrRacket, which can be found at https://docs.racket-lang.org/getting-started/index.html. In order to write a Typed Racket module within DrRacket you would be required to follow the three steps outlined at: https://docs.racket-lang.org/ts-guide/quick.html.


\subsection{Reticulated Python}

The information regarding the installment and running of Reticulated Python can be found at: https://github.com/mvitousek/reticulated. It requires Python version 3.5 or older.
\end{platforms}

% ARTIFACT: section specifying the license under which the artifact is
% made available
\license{The artifact is available under license \dots.}

% ARTIFACT: section specifying the md5 sum of the artifact master file
% uploaded to the Dagstuhl Research Online Publication Server, enabling 
% downloaders to check that the file is the expected version and suffered 
% no damage during download.
\mdsum{xxx}

% ARTIFACT: section specifying the size of the artifact master file uploaded
% to the Dagstuhl Research Online Publication Server
\artifactsize{x.xx GiB}

\subparagraph*{Acknowledgements.}

I want to thank \dots

% ARTIFACT: optional appendix

% ARTIFACT: include here any additional references, if needed...

%%
%% Bibliography
%%

%% Either use bibtex (recommended), 

\bibliography{darts-v2018-sample-article}

%% .. or use the thebibliography environment explicitely



\end{document}
